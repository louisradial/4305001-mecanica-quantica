% vim: spl=pt
\begin{lemma}{Desigualdade de Cauchy-Schwarz}{cauchy_schwarz}
    Em um espaço de produto interno \((V, \inner{\noarg}{\noarg})\) sobre o corpo\footnote{Estamos considerando \(\mathbb{K} = \mathbb{C}\) ou \(\mathbb{K} = \mathbb{R}.\)} \(\mathbb{K}\) vale
    \begin{equation*}
        \abs{\inner{x}{y}}^2 \leq \inner{x}{x} \inner{y}{y}
    \end{equation*}
    para todos \(x, y \in V.\) Ainda, a igualdade se dá se e somente se \(x\) e \(y\) são linearmente dependentes.
\end{lemma}
\begin{proof}
    A desigualdade é trivialmente satisfeita se pelo menos um dos vetores é o vetor nulo. Podemos assumir então que \(x, y \in V \setminus\set{0}\) e consideramos a inequação \(\inner*{\alpha x + y}{\alpha x + y} \geq 0,\) que vale para todo \(\alpha \in \mathbb{K}\), pela positividade da forma sesquilinear. Expandindo o produto interno com suas propriedades de sesquilinearidade, obtemos
    \begin{align*}
        \inner{\alpha x + y}{\alpha x + y} &= \alpha\inner{\alpha x + y}{x} + \inner*{\alpha x + y}{y}\\
                                           &= \alpha \left(\conj{\alpha} \inner{x}{x} + \inner{y}{x}\right) + \conj{\alpha}\inner{x}{y} + \inner{y}{y}\\
                                           &= \alpha\conj{\alpha}\inner{x}{x} + \alpha \inner{y}{x} + \conj{\alpha}\inner{x}{y} + \inner{y}{y}\\
                                           &= \abs{\alpha}^2 \inner{x}{x} + 2\Re\left(\conj{\alpha}\inner{x}{y}\right) + \inner{y}{y}.
    \end{align*}
    Tomando \(\alpha = - \frac{\inner{x}{y}}{\inner{x}{x}}\), segue da simetria por conjugação que
    \begin{equation*}
        \inner{\alpha x + y}{\alpha x + y} = \inner{y}{y} - \frac{\abs*{\inner{x}{y}}^2}{\inner{x}{x}} \geq 0,
    \end{equation*}
    isto é, \(\abs*{\inner{x}{y}}^2 \leq \inner{x}{x} \inner{y}{y}\).

    Suponha \(\set{x,y} \subset V\) linearmente dependente. Se um dos vetores é o vetor nulo, então a igualdade é satisfeita trivialmente. Podemos assumir sem perda de generalidade que \(x, y \in V\setminus\set{0}\), então existe \(\alpha \in \mathbb{K}\setminus\set{0}\) tal que \(\alpha x = y\). Assim, temos
    \begin{equation*}
        \abs*{\inner{x}{y}}^2 = \inner{x}{y}\inner{y}{x} = \inner{x}{\alpha x}\inner*{y}{\frac1\alpha y} = \inner{x}{x} \inner{y}{y},
    \end{equation*}
    isto é, a igualdade é satisfeita.

    Suponha que \(\abs*{\inner{x}{y}} = \inner{x}{x} \inner{y}{y}\). Novamente, se algum dos dois vetores é o vetor nulo, então \(\set{x,y}\) é trivialmente linearmente dependente, portanto podemos assumir sem perda de generalidade que \(x, y \in V \setminus\set{0}\). Consideremos
    \begin{equation*}
        \inner*{y - \frac{\inner{x}{y}}{\inner{x}{x}}x}{y - \frac{\inner{x}{y}}{\inner{x}{x}}x} = \inner{y}{y} - \frac{\abs*{\inner{x}{y}}}{\inner{x}{x}}.
    \end{equation*}
    Por hipótese, o lado direito é nulo, portanto pelo produto interno ser positivo definido temos \(y - \frac{\inner{x}{y}}{\inner{x}{x}}x = 0\), logo \(\set{x,y}\) é linearmente dependente.
\end{proof}

\begin{lemma}{Relação de incerteza}{incerteza}
    O desvio padrão de um observável \(X\) em um estado puro \(\ket{\phi}\) é a quantidade
    \begin{equation*}
        \Delta_\phi X = \sqrt{\mean{X^2}_\phi - \mean{X}^2_\phi}.
    \end{equation*}
    Em um estado puro \(\ket{\phi},\) vale
    \begin{equation*}
        \Delta_\phi A \Delta_\phi B \geq \frac12 \abs{\mean{\commutator{A}{B}}_\phi}^2
    \end{equation*}
    para quaisquer observáveis \(A\) e \(B\).
\end{lemma}
\begin{proof}
    Para um observável \(X\) e um vetor \(\ket{\psi}\), denotaremos \(\ket{\psi_X} = (X - \mean{X}_\psi)\ket{\psi}.\) Se \(Y\) é também um observável, notemos que
    \begin{align*}
        \braket{\psi_X}{\psi_Y} &= \bra{\psi}(X - \mean{X}_\psi)(Y - \mean{Y}_\psi)\ket{\psi}\\
                                &= \frac12 \bra{\psi} \commutator{X - \mean{X}_\psi}{Y - \mean{Y}_\psi}\ket\psi + \frac12 \bra{\psi}\anticommutator{X - \mean{X}_\psi}{Y - \mean{Y}_\psi}\ket{\psi}\\
                                &= \frac12 \bra\psi \commutator{X}{Y} \ket\psi + \frac12 \mean{\anticommutator{X - \mean{X}_\psi}{Y - \mean{Y}_\psi}}_{\psi}\\
                                &= \frac12 \mean*{\commutator{X}{Y}}_\psi + \frac12 \mean*{\anticommutator{X - \mean{X}_\psi}{Y - \mean{Y}_\psi}}_\psi
    \end{align*}
    portanto no caso particular \(Y = X,\)
    \begin{align*}
        \braket{\psi_X}{\psi_X} &= \mean*{X^2 - 2\mean{X}_\psi X + \mean{X}_\psi^2}_\psi\\
                                &= \mean{X^2}_\psi - \mean{X}_\psi^2\\
                                &= (\Delta_\psi X)^2
    \end{align*}
    Assim, para observáveis \(A\) e \(B\) e um estado puro \(\phi\), temos
    \begin{equation*}
        (\Delta_\phi A)^2(\Delta_\phi B)^2 = \braket{\phi_A}{\phi_A}\braket{\phi_B}{\phi_B} \geq \abs{\braket{\phi_A}{\phi_B}}^2 = \abs*{\frac12 \mean{\commutator{A}{B}}_\phi + \frac12 \mean{\anticommutator{A - \mean{A}_\phi}{B - \mean{B}_\phi}}_\phi}^2
    \end{equation*}
    pela \nameref{lem:cauchy_schwarz}. Relembremos que se \(X\) e \(Y\) são hermitianos, os operadores \(i\commutator{X}{Y}\) e \(\anticommutator{X}{Y}\) são hermitianos, pois temos
    \begin{align*}
        \herm{\commutator{X}{Y}} &= \herm{(XY)} - \herm{(YX)}&
        \herm{\anticommutator{X}{Y}} &= \herm{(XY)} - \herm{(YX)}\\
                                     &= \herm{Y} \herm{X} - \herm{X} \herm{Y}&
                                 &= \herm{Y} \herm{X} + \herm{X} \herm{Y}\\
                                 &= YX - XY&
                                 &= YX + XY\\
                                 &= -\commutator{X}{Y}&
                                 &= \anticommutator{X}{Y}.
    \end{align*}
    Assim, como \(i \commutator{A}{B}\) é hermitiano, o valor esperado \(\mean{\commutator{A}{B}}_\phi\) é puramente imaginário e então
    \begin{equation*}
        (\Delta_\phi A)^2 (\Delta_\phi B)^2 \geq \frac14 \abs*{\mean{\commutator{A}{B}}_\phi}^2 + \frac14 \mean*{\anticommutator{A - \mean{A}_\phi}{B - \mean{B}_\phi}}_\phi^2 \geq \frac14 \abs*{\mean{\commutator{A}{B}}_\phi}^2,
    \end{equation*}
    e obtemos \((\Delta_\phi A) (\Delta_\phi B) \geq \frac12 \abs*{\mean{\commutator{A}{B}}_\phi}\).
\end{proof}

\begin{exercício}{Condição de igualdade para a relação de incerteza}{exercício01}
    Mantendo a notação do \cref{lem:incerteza}, mostre que a relação de incerteza \(\Delta_\phi A \Delta_\phi B \geq \frac12\abs{\mean{\commutator{A}{B}}_\phi}\) torna-se uma igualdade quando \(\ket{\phi}\) é tal que \(\ket{\phi_A} = \lambda \ket{\phi_B}\), com \(\lambda = \frac{\mean{\commutator{A}{B}}}{2 (\Delta_\phi B)^2}.\)
\end{exercício}
\begin{proof}[Resolução]
    Dos \cref{lem:cauchy_schwarz,lem:incerteza}, temos a igualdade
    \begin{equation*}
        (\Delta_\phi A)^2(\Delta_\phi B)^2 = \braket{\phi_A}{\phi_A}\braket{\phi_B}{\phi_B} = \abs{\braket{\phi_A}{\phi_B}}^2
    \end{equation*}
    se e somente se \(\ket{\phi_A}\) e \(\ket{\phi_B}\) forem linearmente dependentes. Então, se \(\ket{\phi_A} = \lambda \ket{\phi_B},\) temos
    \begin{equation*}
        (\Delta_\phi A) (\Delta_\phi B) = \abs*{\braket{\phi_A}{\phi_B}} = \abs{\lambda} \abs*{\braket{\phi_B}{\phi_B}} = \abs{\lambda} (\Delta_{\phi}B)^2
    \end{equation*}
    logo se \(\lambda = \frac{\abs{\mean{\commutator{A}{B}}_\phi}}{2 (\Delta_\phi B)^2}\), obtemos a igualdade
    \begin{equation*}
        (\Delta_\phi A)(\Delta_\phi B) = \frac12 \abs{\mean*{\commutator{A}{B}}_\phi},
    \end{equation*}
    como desejado.
\end{proof}
