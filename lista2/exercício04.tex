% vim: spl=pt
\begin{lemma}{Comutador de um produto}{baccab}
    Vale
    \begin{equation*}
        \commutator{A}{BC} = \commutator{A}{B}C + B \commutator{A}{C}
    \end{equation*}
    para quaisquer operadores \(A, B, C.\)
\end{lemma}
\begin{proof}
    Temos
    \begin{equation*}
        \commutator{A}{BC} = ABC - BAC + BAC - BCA = (AB - BA)C + B(AC - CA) = \commutator{A}{B}C + B\commutator{A}{C}
    \end{equation*}
    como desejado.
\end{proof}
\begin{lemma}{Relações de comutação canônica e funções de momento e posição}{ccr}
    Vale
    \begin{equation*}
        \commutator{x}{g(p)} = i \hbar \diffp{g}{p}
        \quad\text{e}\quad
        \commutator{f(x)}{p} = i \hbar \diffp{f}{p}
    \end{equation*}
    para funções analíticas \(f\) e \(g\).
\end{lemma}
\begin{proof}
    Mostremos que
    \begin{equation*}
        \commutator{x}{p^n} = i \hbar n p^{n - 1}
        \quad\text{e}\quad
        \commutator{x^n}{p} = i \hbar n x^{n-1}
    \end{equation*}
    para todo \(n \in \mathbb{N}\). Evidentemente, as expressões são válidas para \(n = 1\). Supondo válidas para algum \(m \in \mathbb{N}\), temos pelo \cref{lem:baccab} que
    \begin{align*}
        \commutator{x}{p^{m+1}} &= \commutator{x}{p^m}p + p^m\commutator{x}{p}&
        \commutator{x^{m+1}}{p} &= -\commutator{p}{x^m}x - x^m\commutator{p}{x}\\
                                &= i \hbar m p^m + i\hbar p^m&
                                &= i \hbar m x^m + i \hbar x^m\\
                                &= i \hbar (m+1) p^m&
                                &= i \hbar(m + 1) x^m,
    \end{align*}
    isto é, são válidas para \(m+1.\) Pelo princípio de indução finita, concluímos que as expressões seguem para todo \(n \in \mathbb{N}\).
    
    Como \(f\) e \(g\) são analíticas, temos
    \begin{equation*}
        f(x) = f_0 \unity + \sum_{n \in \mathbb{N}} f_n x^n
        \quad\text{e}\quad
        g(p) = g_0 \unity + \sum_{n \in \mathbb{N}} g_n p^n,
    \end{equation*}
    com \(f_n, g_n \in \mathbb{C}\). Assim, obtemos
    \begin{align*}
        \commutator{x}{g(p)} &= \commutator*{x}{g_0 \unity + \sum_{n \in \mathbb{N}} g_n p^n} &
        \commutator{f(x)}{p} &= \commutator*{f_0 \unity + \sum_{n \in \mathbb{N}} f_n x^n}{p}\\
                             &= \commutator{x}{\sum_{n \in \mathbb{N}} g_n p^n} &
                             &= \commutator{\sum_{n \in \mathbb{N}} f_n x^n}{p} \\
                             &= \sum_{n \in \mathbb{N}} g_n \commutator{x}{p^n} &
                             &= \sum_{n \in \mathbb{N}} f_n \commutator{x^n}{p} \\
                             &= i\hbar \sum_{n \in \mathbb{N}} g_n n p^n &
                             &= i\hbar \sum_{n \in \mathbb{N}} f_n n x^n \\
                             &= i \hbar \diffp{g}{p}&
                             &= i \hbar \diffp{f}{p},
    \end{align*}
    como desejado.
\end{proof}
\begin{exercício}{Evolução temporal de uma partícula livre em uma dimensão}{exercício04}
    Considere uma partícula livre em uma dimensão.
    \begin{enumerate}[label=(\alph*)]
        \item Encontre e resolva as equações de movimento de Heisenberg para os operadores \(x(t)\) e \(p(t)\).
        \item Calcule \(\Delta x(t)\).
        \item Calcule \(\commutator{x(t)}{x(0)}\).
    \end{enumerate}
\end{exercício}
\begin{proof}[Resolução]
    Para uma partícula livre, temos \(H = \frac{1}{2m}p^2,\) portanto
    \begin{align*}
        \diff{x}{t} &= \frac{1}{i \hbar}\commutator{x}{H}&
        \diff{p}{t} &= \frac{1}{i \hbar}\commutator{p}{H}\\
                    &= \frac{1}{2i \hbar m}\commutator{x}{p^2}&
                    &= \frac{1}{2i \hbar m} \commutator{p}{p^2}\\
                    &= \frac{1}{m}p&
                    &= 0
    \end{align*}
    são as equações de movimento. Como \(p\) é uma constante de movimento, temos \(p(t) = p(0) = p_0\) e \(x(t) = x_0 + \frac{t}{m}p_0,\) com \(x(0) = x_0\). Ainda, temos \(x^2(t) = x_0^2 + \frac{t^2}{m^2} p_0^2 + \frac{t}{m}\anticommutator{x_0}{p_0}\) e então
    \begin{equation*}
        \mean{x(t)}^2 = \mean{x_0}^2 + \frac{t^2}{m^2} \mean{p_0}^2 + \frac{2 t}{m} \mean{x_0} \mean{p_0}
        \quad\text{e}\quad
        \mean{x^2(t)} = \mean{x_0^2} + \frac{t^2}{m^2} \mean{p_0^2} + \frac{t}{m} \mean{\anticommutator{x_0}{p_0}},
    \end{equation*}
    logo
    \begin{align*}
        \Delta x(t) = \sqrt{\mean{x^2(t)} - \mean{x(t)}^2} &= \sqrt{(\Delta x_0)^2 + \frac{t^2}{m^2} (\Delta p_0)^2 + \frac{t}{m} \left[\mean{\anticommutator{x_0}{p_0}} - 2\mean{x_0}\mean{p_0}\right]}\\
                                                           &=\sqrt{(\Delta x_0)^2 + \frac{t^2}{m^2} (\Delta p_0)^2 + \frac{2t}{m} \left[i \hbar + \mean{p_0 x_0} - \mean{x_0}\mean{p_0}\right]}
    \end{align*}
    é a evolução temporal do desvio padrão para o operador \(x\). Notemos que \(x\) não comuta consigo em tempos distintos, pois temos
    \begin{equation*}
        \commutator{x(t)}{x(s)} = \commutator{x_0 + \frac{t}{m} p_0}{x_0 + \frac{s}{m} p_0} = \frac{t}{m}\commutator{p_0}{x_0}  +\frac{s}{m} \commutator{x_0}{p_0}= \frac{i\hbar (s - t)}{m}
    \end{equation*}
    e \(\commutator{x(t)}{x(0)} = -\frac{i \hbar t}{m}\), por exemplo.
\end{proof}
