% vim: spl=pt
\begin{exercício}{}{exercício03}
    Assuma que \(\ket{a_1}\) e \(\ket{a_2}\) são autokets de um operador hermitiano \(A\), cujos autovalores são \(a_1\) e \(a_2,\) respectivamente. O hamiltoniano do sistema é dado por
    \begin{equation*}
         H = \delta \ket{a_1}\bra{a_2} + \delta \ket{a_2}\bra{a_1},
    \end{equation*}
    onde \(\delta\) é um número real.
    \begin{enumerate}[label=(\alph*)]
        \item Obtenha os autovalores e respectivos autovetores de \(H\).
        \item Se no instante inicial o sistema encontra-se no estado \(\ket{a_1}\), obtenha o estado do sistema para \(t > 0\) na representação de Schrödinger.
        \item Nas condições do item anterior, qual a probabilidade do sistema ser encontrado no estado \(\ket{a_2}\) para \(t > 0\)?
    \end{enumerate}
\end{exercício}
\begin{proof}[Resolução]
    
\end{proof}
