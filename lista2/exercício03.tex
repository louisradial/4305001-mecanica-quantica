% vim: spl=pt
\begin{exercício}{Evolução temporal de um sistema de dois níveis}{exercício03}
    Assuma que \(\ket{a_1}\) e \(\ket{a_2}\) são autokets de um operador hermitiano \(A\), cujos autovalores são \(a_1\) e \(a_2,\) respectivamente, com \(a_1 \neq a_2\). O hamiltoniano do sistema é dado por
    \begin{equation*}
         H = \delta \ket{a_1}\bra{a_2} + \delta \ket{a_2}\bra{a_1},
    \end{equation*}
    onde \(\delta\) é um número real.
    \begin{enumerate}[label=(\alph*)]
        \item Obtenha os autovalores e respectivos autovetores de \(H\).
        \item Se no instante inicial o sistema encontra-se no estado \(\ket{a_1}\), obtenha o estado do sistema para \(t > 0\) na representação de Schrödinger.
        \item Nas condições do item anterior, qual a probabilidade do sistema ser encontrado no estado \(\ket{a_2}\) para \(t > 0\)?
    \end{enumerate}
\end{exercício}
\begin{proof}[Resolução]
    Definamos os estados
    \begin{equation*}
        \ket{\pm} = \frac1{\sqrt{2}} \ket{a_1} \pm \frac1{\sqrt{2}} \ket{a_2},
    \end{equation*}
    então como \(\braket{a_1}{a_2} = 0\) pela não degenerescência do espectro de \(A\), temos
    \begin{equation*}
        H\ket{\pm} = \delta \ket{a_1} \braket{a_2}{\pm} + \delta \ket{a_1} \braket{a_1}{\pm} = \pm\frac{\delta}{\sqrt{2}} \ket{a_1} + \frac{\delta}{\sqrt{2}}\ket{a_2} = \pm \delta \ket{\pm},
    \end{equation*}
    logo \(\sigma(H) = \set{\delta, -\delta}\) é o espectro do hamiltoniano e os autovetores de \(H\) são \(\ket{+}\) e \(\ket{-}\).

    Seja \(\ket{\Psi} = \ket{a_1}\) o estado inicial, então 
    \begin{equation*}
        \ket{\Psi} = \ket{+}\braket{+}{a_1} + \ket{-}\braket{-}{a_1} = \frac{1}{\sqrt{2}} \ket{+} + \frac{1}{\sqrt{2}}\ket{-}.
    \end{equation*}
    Sua evolução temporal é dada por
    \begin{align*}
        \ket{\Psi; t} = \exp\left(\frac{t}{i\hbar} H\right)\ket{\Psi} 
        &= \frac{1}{\sqrt{2}}\exp\left(\frac{t}{i\hbar} H\right)\ket{+} + \frac{1}{\sqrt{2}} \exp\left(\frac{t}{i \hbar} H\right)\ket{-}\\
        &= \frac{1}{\sqrt{2}} \exp\left(\frac{\delta t}{i \hbar}\right)\ket{+} + \frac{1}{\sqrt{2}} \exp\left(-\frac{\delta t}{i \hbar}\right)\ket{-}\\
        &= \frac{1}{\sqrt{2}} \exp\left(\frac{\delta t}{i \hbar}\right)\left(\frac1{\sqrt{2}}\ket{a_1} + \frac1{\sqrt{2}}\ket{a_2}\right) + \frac{1}{\sqrt{2}} \exp\left(-\frac{\delta t}{i \hbar}\right)\left(\frac1{\sqrt{2}}\ket{a_1} - \frac1{\sqrt{2}}\ket{a_2}\right)\\
        &= \frac12\left[\exp\left(-\frac{i\delta t}{\hbar}\right) + \exp\left(\frac{i\delta t}{\hbar}\right)\right]\ket{a_1} - \frac12\left[\exp\left(\frac{i \delta t}{i \hbar}\right) - \exp\left(-\frac{i \delta t}{i \hbar}\right)\right]\ket{a_2}\\
        &= \cos\left(\frac{\delta t}{\hbar}\right) \ket{a_1} - i \sin\left(\frac{\delta t}{\hbar}\right) \ket{a_2}.
    \end{align*}
    Desta forma, vemos que  \(\abs{\braket{a_2}{\Psi; t}}^2 = \sin^2\left(\frac{\delta t}{\hbar}\right)\) é a probabilidade de encontrar o sistema no estado \(\ket{a_2}\) no instante \(t\).
\end{proof}
