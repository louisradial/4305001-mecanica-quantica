% vim: spl=pt
\begin{lemma}{Operador de translação espacial}{translação}
    O operador de translação espacial \(T_a = \exp\left(- \frac{i p a}{\hbar}\right)\) satisfaz
    \begin{equation*}
        \herm{T}_a x T_a = x + a
        \quad\text{e}\quad
        T_a \ket{x'} = \ket{x' + a}
    \end{equation*}
    para todo \(a \in \mathbb{R}\) e todo \(x' \in \sigma(x)\).
\end{lemma}
\begin{proof}
    Pelo \cref{lem:ccr}, temos
    \begin{equation*}
        x T_a = T_a x + \commutator{x}{T_a} = T_a x + i \hbar \diffp{T_a}{p} = T_a x + a T_a = T_a (x + a),
    \end{equation*}
    portanto da unitariedade de \(T_a\), temos \(\herm{T}_a x T_a = x + a.\) Seja \(\ket{x'}\) tal que \(x \ket{x'} = x' \ket{x'},\) então
    \begin{equation*}
        x T_a \ket{x'} = T_a (x + a) \ket{x'} = (x' + a) T_a \ket{x'},
    \end{equation*}
    isto é, \(T_a \ket{x'} = \ket{x' + a},\) como desejado.
\end{proof}
\begin{lemma}{Operador de número}{oscilador}
    Sejam operadores \(b, \herm{b}\) tais que \(\commutator{b}{\herm{b}} = 1\) e que \(\herm{b}b\) admita zero como autovalor não degenerado\footnote{Deveríamos também exigir que \(b\) seja um operador fechado ou, equivalentemente, que \(\herm{b}b\) seja hermitiano, o que não é imediato como costumamos tomar na literatura. Ainda, assumimos que seu espectro é igual ao conjunto de autovalores, isto é, que o espectro residual e espectro contínuo são tomados como conjuntos vazios.}. O operador de número \(\herm{b}b\) tem espectro não degenerado \(\sigma(N) = \mathbb{N}_0\), os números inteiros não negativos, e os seus autovetores \(\setc{\ket{n}}{n \in \mathbb{N}_0}\) podem ser tomados de forma que
    \begin{equation*}
        b\ket{0} =0,\quad b \ket{n+1} = \sqrt{n+1} \ket{n},\quad\text{e}\quad \herm{b}\ket{n} = \sqrt{n+1} \ket{n+1}
    \end{equation*}
    para todo \(n \in \mathbb{N}_0.\)
\end{lemma}
\begin{proof}
    É claro que o operador hermitiano \(\herm{b}b\) é um operador positivo, pois temos
    \begin{equation*}
        \bra{\psi}\herm{b}b\ket{\psi} = \norm{b \ket{\psi}}^2 \geq 0
    \end{equation*}
    para todo \(\ket{\psi},\) logo seu espectro é não negativo. Para um autovetor \(\ket{\lambda},\) com \(\herm{b}b \ket{\lambda} = \lambda \ket{\lambda},\) temos
    \begin{equation*}
        \norm{b \ket{\lambda}}^2 = \bra{\lambda}\herm{b}b\ket{\lambda} = \lambda\braket{\lambda}{\lambda} = \lambda,
    \end{equation*}
    portanto como zero é um autovalor do operador de número temos \(b \ket{0} = 0\) e se \(\lambda > 0\) temos \(b \ket{\lambda} \neq 0\). De forma análoga, temos
    \begin{equation*}
        \norm{\herm{b} \ket{\lambda}}^2 = \bra{\lambda} b \herm{b} \ket{\lambda} = \bra{\lambda} \herm{b} b + 1\ket{\lambda} = \lambda + 1 > 0,
    \end{equation*}
    isto é \(\herm{b} \ket{\lambda} \neq 0\) para todo autovalor \(\lambda\). Notemos que
    \begin{align*}
        \commutator{\herm{b}b}{b} = - \commutator{b}{\herm{b}}b - \herm{b}\commutator{b}{b} = - b
        \quad\text{e}\quad
        \commutator{\herm{b}b}{\herm{b}} = - \commutator{\herm{b}}{\herm{b}}b - \herm{b}\commutator{\herm{b}}{b} = \herm{b},
    \end{align*}
    portanto para um autovalor \(\lambda > 0\) temos
    \begin{equation*}
        b \ket{\lambda} = \commutator{b}{\herm{b}b} \ket{\lambda} = b \herm{b}b \ket{\lambda} - \herm{b} b b\ket{\lambda} = \lambda b \ket{\lambda} - \herm{b} b b\ket{\lambda} \implies \herm{b} b b \ket{\lambda} = (\lambda - 1) b\ket{\lambda},
    \end{equation*}
    portanto \(b \ket{\lambda}\) é autovetor associado ao autovalor \(\lambda - 1,\) donde segue que \(\lambda \geq 1\). Semelhantemente, para um autovalor \(\lambda\) arbitrário, temos
    \begin{equation*}
        \herm{b}\ket{\lambda} = \commutator{\herm{b}b}{\herm{b}} \ket{\lambda} = \herm{b}b \herm{b}\ket{\lambda} - \herm{b} \herm{b}b\ket{\lambda} = \herm{b}b \herm{b}\ket{\lambda} - \lambda\herm{b}\ket{\lambda} \implies \herm{b}b \herm{b} \ket{\lambda} = (\lambda + 1) \herm{b}\ket{\lambda},
    \end{equation*}
    logo \(\herm{b}\ket{\lambda}\) é autovetor associado ao autovalor \(\lambda + 1.\) Em resumo, \(\herm{b}\) aumenta o autovalor em uma unidade e \(b\) reduz o autovalor em uma unidade, com \(b\ket{0} = 0\). Como zero é autovalor e como \(\herm{b}\) pode elevar o autovalor arbitrariamente, é claro que \(\mathbb{N}_0 \subset \sigma(\herm{b}b)\). Para mostrar a igualdade destes conjuntos, consideramos um autovalor não inteiro \(\lambda\) e notamos que nunca teremos \(a^k \ket{\lambda} = 0,\) pois \(\lambda - k\) jamais é inteiro. Entretanto, aplicando-se \(k = \ceil{\lambda} + 1\) vezes o operador \(b,\) obtemos um autovetor associado ao autovalor \(\lambda - \ceil{\lambda} - 1 < 0,\) o que contradiz a positividade de \(\herm{b}b.\) Assim, todos os seus autovalores são inteiros não negativos.

    Mostremos por indução que o seu espectro é não degenerado. Por hipótese, o autovalor zero não é degenerado. Assumindo que não é degenerado para o autovalor \(n \in \mathbb{N},\) supomos por contradição que o autoespaço associado ao autovalor \(n + 1\) é degenerado com dimensão \(k > 1,\) isto é, existem \(k\) vetores \(\ffamily{\ket{n + 1_j}}{j = 1}{k}\) ortogonais tais que \(\herm{b}b\ket{n+1_j} = (n+1)\ket{n+1_j}.\) Assim, 
    \begin{equation*}
        \herm{b}\ket{n} = \sum_{j = 1}^k \alpha_j \ket{n + 1_j}
        \quad\text{com}\quad
        \sum_{j = 1}^k \abs{\alpha_j}^2 = \norm{\herm{b}\ket{n}}^2,
    \end{equation*}
    mas para um \(j \in \set{1, \dots, k}\) temos
    \begin{equation*}
        (n + 1) \ket{n+1_j} = \herm{b}b\ket{n+1_j} = \herm{b}\ket{n} = \sum_{\ell = 1}^k \alpha_{\ell} \ket{n+1_\ell},
    \end{equation*}
    donde segue que \(\alpha_{\ell} = (n+1) \delta_{j \ell}\), e então \(\herm{b}\ket{n} = (n+1) \ket{n + 1_j}\). Como este \(j\) é arbitrário, teríamos a mesma expressão para \(j = 1\) e \(j = 2,\) isto é, \(\ket{n+1_1} = \ket{n+1_2},\) o que contradiz a hipótese de que o autoespaço tem dimensão \(k > 1,\) e concluímos que o autoespaço associado ao autovalor \(n + 1\) é não degenerado. Pelo princípio da indução finita, inferimos que o espectro de \(\herm{b}b\) é não degenerado. Com isso, concluímos que
    \begin{equation*}
        \herm{b} \ket{n} = \beta_n\ket{n+1}
        \quad\text{e}\quad
        b \ket{n+1} = \gamma_{n+1} \ket{n}
    \end{equation*}
    para todo \(n \in \mathbb{N}_0,\) onde \(\beta_n, \gamma_{n+1} \in \mathbb{C}.\) Como temos as normas \(\norm{b \ket{n}}^2 = n\) e \(\norm{\herm{b}\ket{n}}^2 = n+1,\) podemos tomar \(\beta_n = \sqrt{n}\) e \(\gamma_{n+1} = \sqrt{n+1}\), isto é, 
    \begin{equation*}
        b\ket{0} = 0,\quad
        b\ket{n+1} = \sqrt{n+1} \ket{n}
        \quad\text{e}\quad
        \herm{b} \ket{n} = \sqrt{n} \ket{n+1}
    \end{equation*}
    para todo \(n \in \mathbb{N}_0\), como desejado.
\end{proof}
\begin{exercício}{Evolução temporal para o oscilador harmônico}{exercício05}
    Considere uma partícula sob a ação do potencial harmônico unidimensional
    \begin{equation*}
        V(x) = \lambda x + \frac12 m \omega^2 x^2.
    \end{equation*}
    \begin{enumerate}[label=(\alph*)]
        \item Utilizando a representação de Heisenberg obtenha a evolução temporal de \(p(t)\) e \(x(t)\).
        \item Se em \(t = 0\) o sistema está no estado \(\exp\left(-\frac{ipa}{\hbar}\right)\ket{0},\) onde \(a \in \mathbb{R}\) e \(\ket{0}\) é o estado fundamental do oscilador harmônico. Utilizando a representação de Heisenberg, obtenha \(\mean{x(t)}\) para \(t > 0\).
        \item A função de correlação é definida por \(C(t) = \mean{x(t)x(0)},\) onde \(x(t)\) é o operador de posição na representação de Heisenberg. Obtenha \(C(t)\) para o estado fundamental do oscilador harmônico.
    \end{enumerate}
\end{exercício}
\begin{proof}[Resolução]
    Em termo dos operadores adimensionais
    \begin{equation*}
        \tilde{p} = \sqrt{\frac{1}{m \hbar \omega}} p
        \quad\text{e}\quad
        \tilde{x} = \sqrt{\frac{m \omega}{\hbar}} \left(x + \frac{\lambda}{m \omega^2}\right),
    \end{equation*}
    temos
    \begin{equation*}
        \commutator{\tilde{x}}{\tilde{p}} = \frac{1}{\hbar} \commutator*{x + \frac{\lambda}{m \omega^2}}{p} = i
    \end{equation*}
    e
    \begin{equation*}
        \frac{\hbar \omega}{2}\tilde{x}^2 = \frac{m \omega^2}{2} \left(x^2 + \frac{2\lambda x}{m \omega^2}+ \frac{\lambda^2}{m^2 \omega^4}\right) = V(x) + \frac{\lambda^2}{2 m \omega^2}.
    \end{equation*}
    Assim, o hamiltoniano é dado por
    \begin{equation*}
        H = \frac{\hbar \omega}{2} \left(\tilde{p}^2 + \tilde{x}^2\right) - \frac{\lambda^2}{2m \omega^2},
    \end{equation*}
    com as relações de comutação
    \begin{equation*}
        \commutator{\tilde{x}}{H} = \frac{\hbar \omega}{2} \commutator{\tilde{x}}{\tilde{p}^2} = i \hbar \omega \tilde{p}
    \end{equation*}
    e
    \begin{equation*}
        \commutator{\tilde{p}}{H} = -\frac{\hbar \omega}{2} \commutator{\tilde{x}^2}{\tilde{p}} = - i\hbar \omega \tilde{x},
    \end{equation*}
    logo
    \begin{equation*}
        \diff{\tilde{x}}{t} = \omega \tilde{p}\quad\text{e}\quad\diff{\tilde{p}}{t} = - \omega \tilde{x}
    \end{equation*}
    são as equações de movimento do sistema. Consideramos os operadores de aniquilação e de criação\footnote{Poderíamos aplicar o \cref{lem:pauli} como feito no \cref{ex:exercício02}, mas os operadores de aniquilação e criação são mais convenientes para o restante deste exercício.}
    \begin{equation*}
        b = \frac{1}{\sqrt{2}}(\tilde{x} + i \tilde{p})
        \quad\text{e}\quad
        \herm{b} = \frac{1}{\sqrt{2}}(\tilde{x} - i \tilde{p})
    \end{equation*}
    e notemos que
    \begin{align*}
        \diff{b}{t} &= \frac{1}{\sqrt{2}}\diff*{(\tilde{x} + i\tilde{p})}{t}&
        \diff{\herm{b}}{t} &= \frac{1}{\sqrt{2}}\diff*{(\tilde{x} - i\tilde{p})}{t}\\
                           &= \frac{\omega}{\sqrt{2}}(\tilde{p} - i \tilde{x})&
                           &= \frac{\omega}{\sqrt{2}}(\tilde{p} + i \tilde{x})\\
                           &= -\frac{i \omega}{\sqrt{2}}(\tilde{x} + i \tilde{p})&
                           &= \frac{i\omega}{\sqrt{2}}(\tilde{x} - i \tilde{p})\\
                           &= -i \omega b&
                           &= i \omega \herm{b}
    \end{align*}
    logo
    \begin{equation*}
        b(t) = e^{-i \omega t} b_0 \quad\text{e}\quad \herm{b}(t) = e^{i \omega t} \herm{b}_0 \implies \tilde{x}(t) + i \tilde{p}(t) = e^{-i \omega t} (\tilde{x}_0 + i \tilde{p}_0) \quad\text{e}\quad \tilde{x}(t) - i \tilde{p}(t) = e^{i \omega t} (\tilde{x}_0 - i \tilde{p}_0) 
    \end{equation*}
    e obtemos
    \begin{equation*}
        \tilde{x}(t) = \cos(\omega t) \tilde{x}_0 + \sin(\omega t) \tilde{p}_0
        \quad\text{e}\quad
        \tilde{p}(t) = - \sin(\omega t) \tilde{x}_0 + \cos(\omega t) \tilde{p}_0,
    \end{equation*}
    onde \(\tilde{x}_0 = \tilde{x}(0)\) e \(\tilde{p}_0 = \tilde{p}(0).\)
    Retornando aos operadores originais, temos
    \begin{equation*}
        x(t) = \frac{\sin(\omega t)}{m \omega} p_0+ \cos(\omega t) \left(x_0 + \frac{\lambda}{m \omega^2}\right) - \frac{\lambda}{m \omega^2}
        \quad\text{e}\quad
        p(t) = \cos(\omega t) p_0 - \sin(\omega t) \left( m \omega x_0 + \frac{\lambda}{\omega}\right)
    \end{equation*}
    como suas expressões no instante \(t\), com \(x(0) = x_0\) e \(p(0) = p_0\).

    Notemos que 
    \begin{align*}
        \commutator{b}{\herm{b}} &= \frac12(\tilde{x} + i \tilde{p})(\tilde{x} - i \tilde{p}) - \frac12 (\tilde{x} - i \tilde{p})(\tilde{x} + i \tilde{p})\\
                                  &= \frac12(\tilde{x}^2 + \tilde{p}^2 - i \commutator{\tilde{x}}{\tilde{p}}) - \frac12(\tilde{x}^2 + \tilde{p}^2 + i \commutator{\tilde{x}}{\tilde{p}})\\
                                  &= -i \commutator{\tilde{x}}{\tilde{p}}\\
                                  &= 1
    \end{align*}
    e, analogamente,
    \begin{align*}
        \anticommutator{b}{\herm{b}} &= \frac12(\tilde{x} + i \tilde{p})(\tilde{x} - i \tilde{p}) + \frac12 (\tilde{x} - i \tilde{p})(\tilde{x} + i \tilde{p})\\
                                  &= \frac12(\tilde{x}^2 + \tilde{p}^2 - i \commutator{\tilde{x}}{\tilde{p}}) + \frac12(\tilde{x}^2 + \tilde{p}^2 + i \commutator{\tilde{x}}{\tilde{p}})\\
                                  &= \tilde{x}^2 + \tilde{p}^2,
    \end{align*}
    portanto
    \begin{equation*}
        H = \frac12 \hbar \omega \anticommutator{b}{\herm{b}} - \frac{\lambda^2}{2m \omega^2} = \hbar \omega \herm{b} b + \frac12 \hbar \omega \commutator{b}{\herm{b}} - \frac{\lambda^2}{2m \omega^2} = \hbar \omega \herm{b} b + \frac12 \left(\hbar \omega - \frac{\lambda^2}{m \omega^2}\right),
    \end{equation*}
    isto é, o hamiltoniano é um polinômio do operador \(\herm{b}b\). Notemos que existe um estado \(\ket{0}\) tal que \(b\ket{0} = 0,\) uma vez que, em representação de coordenadas, esta equação se traduz em uma equação diferencial de primeira ordem com uma única solução independente, a saber
    \begin{equation*}
        b\ket{0} = 0 \implies \left(\tilde{x} + \diff*{}{\tilde{x}}\right) \braket{\tilde{x}}{0} = 0\implies \braket{\tilde{x}}{0} \sim e^{-\frac12 \tilde{x}^2},
    \end{equation*}
    portanto \(\herm{b}b\ket{0} = 0\) e inferimos que \(\ket{0}\) gera o núcleo do operador \(\herm{b}b\). Concluímos que, na base definida pelo \cref{lem:oscilador}, temos
    \begin{equation*}
        H \ket{n} = \left[\hbar \omega \left(n + \frac12 \right) - \frac{\lambda^2}{2m \omega^2}\right] \ket{n}
    \end{equation*}
    para todo \(n \in \mathbb{N}_0\). Logo, o estado fundamental do sistema é dado por \(\ket{0}\). Nesse estado, temos
    \begin{equation*}
        \bra{0} \left(b \pm  \herm{b}\right) \ket{0} = \pm \braket{0}{1} = 0,
    \end{equation*}
    isto é, \(\bra{0}\tilde{x}(t)\ket{0} = \bra{0}\tilde{p}(t)\ket{0} = 0\). Disso, sabemos que \(\bra{0}x_0\ket{0} = - \frac{\lambda}{m \omega^2}\), \(\bra{0} p_0\ket{0} = 0\) e
    \begin{equation*}
        \bra{0}x(t)\ket{0} = \bra{0}\left[\sqrt{\frac\hbar{m \omega}} \tilde{x}(t) - \frac{\lambda}{m \omega^2}\right]\ket{0} = - \frac{\lambda}{m \omega^2}.
    \end{equation*}
    Temos também a função de correlação para o estado fundamental do oscilador harmônico
    \begin{align*}
        C(t) = \bra{0}x(t)x_0\ket{0} &= \bra{0}\left(\sqrt{\frac{\hbar}{m \omega}} \tilde{x}(t) - \frac{\lambda}{m \omega^2}\right)\left(\sqrt{\frac{\hbar}{m \omega}} \tilde{x}_0 - \frac{\lambda}{m \omega^2}\right)\ket{0}\\
                              &= \frac{\hbar}{m \omega} \bra{0} \tilde{x}(t) \tilde{x}_0 \ket{0} -\frac{\lambda}{m\omega^2} \left[\bra{0}\tilde{x}(t)\ket{0}+ \bra{0}\tilde{x}_0\ket{0}\right] + \frac{\lambda^2}{m^2 \omega^4}\\
                              &= \frac{\hbar}{2m \omega} \bra{0}(b + \herm{b})(b_0 + \herm{b}_0)\ket{0} + \frac{\lambda^2}{m^2 \omega^4}\\
                              &= \frac{\hbar e^{-i \omega t}}{2m \omega} \bra{0} b_0 \herm{b}_0 \ket{0} + \frac{\lambda^2}{m^2 \omega^4}\\
                              &= \frac{\hbar e^{-i \omega t}}{2 m \omega} + \frac{\lambda^2}{m^2 \omega^4}.
    \end{align*}

    Pelo \cref{lem:translação}, segue que
    \begin{equation*}
        \exp\left(\frac{i p_0 a}{\hbar}\right)x(t) \exp\left(-\frac{i p_0 a}{\hbar}\right) = \sin(\omega t) \frac{p_0}{m \omega}+ \cos(\omega t) \left(x_0 + a + \frac{\lambda}{m \omega^2}\right) - \frac{\lambda}{m \omega^2} = x(t) + a \cos(\omega t),
    \end{equation*}
    então se \(\ket{\psi} = \exp\left(-\frac{i p_0 a}{\hbar}\right)\ket{0}\) é o estado inicial do sistema, obtemos
    \begin{align*}
        \mean{x(t)} = \bra{\psi}x(t)\ket{\psi} &= \bra{0} \exp\left(\frac{i p_0 a}{\hbar}\right) x(t) \exp\left(-\frac{i p_0 a}{\hbar}\right)\ket{0}\\
                                               &= \bra{0} \left[x(t) + a\right]\ket{0}\\
                                               &= \bra{0} x(t) \ket{0} + a \cos(\omega t)\\
                                               &= a \cos(\omega t) - \frac{\lambda}{m \omega^2}
    \end{align*}
    como a evolução temporal do valor esperado de \(x\) para este estado inicial. 
    % Ainda,
    % \begin{align*}
    %     C(t) = \bra{\psi}x(t) x(0)\ket{\psi} &= \bra{0} \exp\left(\frac{i p_0 a}{\hbar}\right) x(t) \exp\left(-\frac{i p_0 a}{\hbar}\right) \exp\left(\frac{i p_0 a}{\hbar}\right) x_0 \exp\left(-\frac{i p_0 a}{\hbar}\right)\ket{0}\\
    %                                          &= \bra{0} \left[x(t) + a \cos(\omega t)\right](x_0 + a)\ket{0}\\
    %                                          &= \bra{0} x(t) x_0 \ket{0} + a \bra{0} x(t) \ket{0} + a \cos(\omega t) \bra{0} (x_0 + a) \ket{0}\\
    %                                          &= \bra{0}x(t) x_0 \ket{0} - \frac{a\lambda}{m \omega^2} + a \left(a - \frac{\lambda}{m \omega^2}\right) \cos(\omega t)\\
    %                                          &= \frac{\hbar e^{- i \omega t}}{ 2 m \omega} + \frac{\lambda^2}{m^2 \omega^4} - \frac{a \lambda}{m \omega^2} + a \left(a - \frac{\lambda}{m \omega^2}\right)\cos(\omega t)
    % \end{align*}
    % é a função de correlação para o estado inicial considerado.
\end{proof}
