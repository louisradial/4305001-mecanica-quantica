% vim: spl=pt
\begin{exercício}{Evolução temporal para o oscilador harmônico}{exercício05}
    Considere uma partícula sob a ação do potencial harmônico unidimensional
    \begin{equation*}
        V(x) = \lambda x + \frac12 m \omega^2 x^2.
    \end{equation*}
    \begin{enumerate}[label=(\alph*)]
        \item Utilizando a representação de Heisenberg obtenha a evolução temporal de \(p(t)\) e \(x(t)\).
        \item Se em \(t = 0\) o sistema está no estado \(\exp\left(-\frac{ipa}{\hbar}\right)\ket{0},\) onde \(a \in \mathbb{R}\) e \(\ket{0}\) é o estado fundamental do oscilador harmônico. Utilizando a representação de Heisenberg, obtenha \(\mean{x(t)}\) para \(t > 0\).
        \item A função de correlação é definida por \(C(t) = \mean{x(t)x(0)},\) onde \(x(t)\) é o operador de posição na representação de Heisenberg. Obtenha \(C(t)\) para o estado fundamental do oscilador harmônico.
    \end{enumerate}
\end{exercício}
\begin{proof}[Resolução]
    Em termo dos operadores adimensionais
    \begin{equation*}
        \tilde{p} = \sqrt{\frac{1}{m \hbar \omega}} p
        \quad\text{e}\quad
        \tilde{x} = \sqrt{\frac{m \omega}{\hbar}} \left(x + \frac{\lambda}{m \omega^2}\right),
    \end{equation*}
    temos
    \begin{equation*}
        \commutator{\tilde{x}}{\tilde{p}} = \frac{1}{\hbar} \commutator*{x + \frac{\lambda}{m \omega^2}}{p} = i
    \end{equation*}
    e
    \begin{equation*}
        \frac{\hbar \omega}{2}\tilde{x}^2 = \frac{m \omega^2}{2} \left(x^2 + \frac{2\lambda x}{m \omega^2}+ \frac{\lambda^2}{m^2 \omega^4}\right) = V(x) + \frac{\lambda^2}{2 m \omega^2}.
    \end{equation*}
    Assim, o hamiltoniano é dado por
    \begin{equation*}
        H = \frac{\hbar \omega}{2} \left(\tilde{p}^2 + \tilde{x}^2\right) - \frac{\lambda^2}{2m \omega^2},
    \end{equation*}
    com as relações de comutação
    \begin{equation*}
        \commutator{\tilde{x}}{H} = \frac{\hbar \omega}{2} \commutator{\tilde{x}}{\tilde{p}^2} = i \hbar \omega \tilde{p}
    \end{equation*}
    e
    \begin{equation*}
        \commutator{\tilde{p}}{H} = -\frac{\hbar \omega}{2} \commutator{\tilde{x}^2}{\tilde{p}} = - i\hbar \omega \tilde{x},
    \end{equation*}
    logo
    \begin{equation*}
        \diff{\tilde{x}}{t} = \omega \tilde{p}\quad\text{e}\quad\diff{\tilde{p}}{t} = - \omega \tilde{x}
    \end{equation*}
    são as equações de movimento do sistema. Notemos que\footnote{Poderíamos aplicar o \cref{lem:pauli} como feito no \cref{ex:exercício02}, mas optamos por outra forma de solução.}
    \begin{equation*}
        \diff*{(\tilde{x} + i\tilde{p})}{t} = -i\omega(\tilde{x} + i \tilde{p})\quad\text{e}\quad
        \diff*{(\tilde{x} - i\tilde{p})}{t} = i\omega(\tilde{x} - i \tilde{p}),
    \end{equation*}
    logo
    \begin{equation*}
        \tilde{x}(t) + i \tilde{p}(t) = e^{-i \omega t} \left(\tilde{x}(0) + i \tilde{p}(0)\right)
        \quad\text{e}\quad
        \tilde{x}(t) - i \tilde{p}(t) = e^{i \omega t} \left(\tilde{x}(0) - i \tilde{p}(0)\right)
    \end{equation*}
    e obtemos
    \begin{equation*}
        \tilde{x}(t) = \cos(\omega t) \tilde{x}(0) + \sin(\omega t) \tilde{p}(0)
        \quad\text{e}\quad
        \tilde{p}(t) = - \sin(\omega t) \tilde{x}(0) + \cos(\omega t) \tilde{p}(0).
    \end{equation*}
    Retornando aos operadores originais, temos
    \begin{equation*}
        x(t) = \sin(\omega t) \frac{p(0)}{m \omega}+ \cos(\omega t) \left(x(0) + \frac{\lambda}{m \omega^2}\right) - \frac{\lambda}{m \omega^2}
        \quad\text{e}\quad
        p(t) = \cos(\omega t) p(0) - \sin(\omega t) \left( m \omega x(0) + \frac{\lambda}{\omega}\right)
    \end{equation*}
    como suas expressões no instante \(t\).
\end{proof}
