% vim: spl=pt
\begin{exercício}{}{exercício02}
    Os operadores \(S_j\), com \(j \in \set{1,2,3}\), obedecem às relações de comutação 
    \begin{equation*}
        \commutator{S_k}{S_\ell} = i \hbar \epsilon_{k\ell m} S_m,
    \end{equation*}
    e no instante \(t = 0\) são dados por \(S_j(0) = \frac12 \hbar \sigma_j\). O hamiltoniano de um sistema é dado por \(H = a S_3\), com \(a \in \mathbb{R}\). Utilizando a representação de Heisenberg,
    \begin{enumerate}[label=(\alph*)]
        \item obtenha os operadores \(S_j(t)\);
        \item obtenha o estado para \(t > 0\), se o estado inicial é \(\ket{\Psi} = \frac1{\sqrt{2}} \ket{+} + \frac1{\sqrt{2}} \ket{-},\) onde \(S_3(0) \ket{\pm} = \pm \frac12 \hbar \ket{\pm};\) e 
        \item determine \(\mean{S_k(t)}.\)
    \end{enumerate}
\end{exercício}
\begin{proof}[Resolução]
    
\end{proof}
