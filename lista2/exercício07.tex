% vim: spl=pt
\begin{lemma}{Regra de Leibniz para comutadores}{leibniz}
    Sejam \(X, Y\) operadores lineares que dependem suavemente de um parâmetro \(t \in \mathbb{R}\). Então
    \begin{equation*}
        \diff*{\commutator{X(t)}{Y(t)}}{t} = \commutator*{\diff{X}{t}}{Y(t)} + \commutator*{X(t)}{\diff{Y}{t}}
    \end{equation*}
    para todo \(t\).
\end{lemma}
\begin{proof}
    Temos
    \begin{align*}
        \diff*{\commutator{X(t)}{Y(t)}}{t} &= \diff*{\left(X(t)Y(t) - Y(t)X(t)\right)}{t}\\
                                           % &= \diff{X}{t} Y(t) + X(t)\diff{Y}{t} - \diff{Y}{t} X(t) - Y(t)\diff{X}{t}\\
                                           &= \left(\diff{X}{t} Y(t) - Y(t) \diff{X}{t}\right) + \left(X(t) \diff{Y}{t} - \diff{Y}{t} X(t)\right)\\
                                           &= \commutator*{\diff{X}{t}}{Y(t)} + \commutator*{X(t)}{\diff{Y}{t}}
    \end{align*}
    como desejado.
\end{proof}

\begin{lemma}{Derivada de um comutador iterado}{iterado}
    Seja \(X\) um operador linear e seja \(Y(t)\) um operador linear que depende suavemente de um parâmetro real. Denotemos o comutador iterado por
    \begin{equation*}
        \commutator{X}{Y(t)}^{[1]} = \commutator{X}{Y(t)}
        \quad\text{e}\quad
        \commutator{X}{Y(t)}^{[n]} = \commutator*{X}{\commutator{X}{Y(t)}^{[n-1]}},
    \end{equation*}
    onde \(n \in \mathbb{N} \setminus \set{1}\). Então
    \begin{equation*}
        \diff*{\commutator{X}{Y(t)}^{[n]}}{t} = \commutator*{X}{\diff{Y}{t}}^{[n]}
    \end{equation*}
    para todo \(n \in \mathbb{N}\).
\end{lemma}
\begin{proof}
    Pelo \cref{lem:leibniz}, segue que \(\diff*{\commutator{X}{Y(t)}}{t} = \commutator*{X}{\diff{Y}{t}}\), visto que \(X\) não depende de \(t\). Suponhamos que \(\diff*{\commutator{X}{Y(t)}^{[k]}}{t} = \commutator*{X}{\diff{Y}{t}}^{[k]}\) para algum \(k \in \mathbb{N}\), então
    \begin{equation*}
        \diff*{\commutator{X}{Y(t)}^{[k+1]}}{t} = \commutator*{X}{\diff*{\commutator{X}{Y(t)}^{[k]}}{t}} = \commutator*{X}{\commutator*{X}{\diff{Y}{t}}^{[k]}} = \commutator*{X}{\diff{Y}{t}}^{[k+1]}.
    \end{equation*}
    Concluímos a demonstração ao invocar o princípio da indução finita.
\end{proof}

\begin{lemma}{Operador comuta com a sua exponencial}{comuta}
    Seja \(X\) um operador linear, então
    \begin{equation*}
        [X, \exp(\lambda X)] = 0
    \end{equation*}
    para todo \(\lambda \in \mathbb{C}\).
\end{lemma}
\begin{proof}
    É claro que \(X\) comuta com suas potências pela associatividade do produto, logo
    \begin{equation*}
        X \exp(\lambda X) = X \left(\sum_{k=0}^\infty \frac{\lambda^k}{k!}X^k\right) = \sum_{k=0}^\infty \frac{\lambda^k}{k!}X^{k+1} = \left(\sum_{k=0}^\infty \frac{\lambda^k}{k!}X^k\right)X = \exp(\lambda X) X,
    \end{equation*}
    como desejado.
\end{proof}

\begin{lemma}{Lema de Campbell}{campbell}
    Seja \(f(t)\) um operador dado por
    \begin{equation*}
        f(t) = \exp(tA) B\exp(-tA),
    \end{equation*}
    onde \(A\) e \(B\) são operadores. Este operador satisfaz
    \begin{equation*}
        \diff[n]{f}{t} = \commutator{A}{f(t)}^{[n]}
    \end{equation*}
    para todo \(n \in \mathbb{N}\) e, então
    \begin{equation*}
        \exp(tA)B\exp(-tA) = B + \sum_{n = 1}^\infty \frac{t^n}{n!}\commutator{A}{B}^{[n]}.
    \end{equation*}
    para todo \(t.\)
\end{lemma}
\begin{proof}[Resolução]
    Pelo \cref{lem:comuta}, obtemos
    \begin{equation*}
        \diff{f}{t} = A \exp(tA) B\exp(-tA) - \exp(tA) B\exp(-tA) A = \commutator{A}{f(t)}
    \end{equation*}
    para todo \(t\). Suponhamos que \(\diff[k]{f}{t} = \commutator{A}{f(t)}^{[k]}\) para algum natural \(k > 1\), então
    \begin{equation*}
        \diff[k+1]{f}{t} = \diff*{\commutator{A}{f(t)}^{[k]}}{t} = \commutator*{A}{\diff{f}{t}}^{[k]} = \commutator*{A}{\commutator{A}{f(t)}}^{[k]} = \commutator{A}{f(t)}^{[k+1]},
    \end{equation*}
    onde utilizamos o \cref{lem:iterado}. Pelo princípio da indução finita, segue que
    \begin{equation*}
        \diff[n]{f}{t} = \commutator{A}{f(t)}^{[n]}
    \end{equation*}
    para todo \(n \in \mathbb{N}\). Desse modo, a série de Taylor em torno de \(t = 0\) para \(f(t)\) é
    \begin{equation*}
        \exp(tA)B\exp(-tA) = B + \sum_{n = 1}^{\infty} \frac{t^n}{n!}\diff[n]{f}{t}[t=0] = B + \sum_{n = 1}^{\infty} \frac{t^n}{n!}\commutator{A}{B}^{[n]},
    \end{equation*}
    como desejado.
\end{proof}

\begin{exercício}{Caso particular da série de Baker-Campbell-Hausdorff}{exercício07}
    Se \(A\) e \(B\) são dois operadores que não comutam entre si mas comutam com \(\commutator{A}{B}\), pode-se mostrar que 
    \begin{equation*}
        e^{A + B} = e^A e^B e^{-\frac12 \commutator{A}{B}}.
    \end{equation*}
    \begin{enumerate}[label=(\alph*)]
        \item Mostre que \(\commutator{B}{e^{xA}} = e^{xA} \commutator{B}{A}x;\)
        \item Definido \(G(x) = e^{xA}e^{xB}\), mostre que
            \begin{equation*}
                \diff{G}{x} = (A + B + \commutator{A}{B}x)G.
            \end{equation*}
        \item Integre a equação do item anterior para obter a relação desejada.
        \item Mostre ainda que para \(X\) e \(Y\) arbitrários
            \begin{equation*}
                e^{\alpha X} e^{\beta Y} = e^{\alpha X + \beta Y + \frac12 \alpha \beta \commutator{X}{Y} + Z},
            \end{equation*}
            onde \(Z\) depende de potências superiores de \(\alpha\) e \(\beta\).
    \end{enumerate}
\end{exercício}
\begin{proof}[Resolução]
    Pelo \cref{lem:campbell}, temos
    \begin{equation*}
        e^{-xA}Be^{xA} = B + \sum_{n = 1}^\infty \frac{(-t)^n}{n!} \commutator{A}{B}^{[n]},
    \end{equation*}
    então para \(\commutator{A}{B}\) central para \(A\) e \(B\), obtemos
    \begin{equation*}
        e^{-xA}Be^{xA} = B - x \commutator{A}{B} = B + x \commutator{B}{A} \implies \commutator{B}{e^{xA}} = B e^{xA} - e^{xA} B = e^{xA} \commutator{B}{A} x.
    \end{equation*}

    Para \(G(x) = e^{xA} e^{xB}\), temos
    \begin{align*}
        \diff{G}{x} &= A e^{xA} e^{xB} + e^{xA} B e^{xB}\\
                    &= A G(x) + \left(B e^{xA} + \commutator{e^{xA}}{B}\right) e^{xB}\\
                    &= A G(x) + B G(x) - e^{xA}\commutator{B}{A}x e^{xB}\\
                    &= (A + B) G(x) + x\commutator{A}{B} e^{xA}e^{xB}\\
                    &= \left(A + B + x \commutator{A}{B}\right) G(x),
    \end{align*}
    onde utilizamos que \(\commutator*{e^{xA}}{\commutator{A}{B}} = 0\) já que \(\commutator*{A}{\commutator{A}{B}} = 0.\) Notemos que \(M(x) = A + B + x\commutator{A}{B}\) comuta consigo em todo o seu domínio, pois temos
    \begin{align*}
        \commutator{M(s)}{M(t)} &= \commutator*{A + B + s\commutator{A}{B}}{A + B + t\commutator{A}{B}}\\
                                &= \commutator{A}{B} + t\commutator*{A}{\commutator{A}{B}} + \commutator{B}{A} + t \commutator*{B}{\commutator{A}{B}} + s\commutator*{\commutator{A}{B}}{A} + s\commutator*{\commutator{A}{B}}{B}\\
                                &= (t - s) \left(\commutator*{A}{\commutator{A}{B}} + \commutator*{B}{\commutator{A}{B}}\right)\\
                                &= 0
    \end{align*}
    já que \(\commutator{A}{B}\) é central para \(A\) e \(B\). Assim, a solução diferencial para \(G\), \(G'(x) = M(x) G(x)\), tem sua solução dada por
    \begin{align*}
        G(x) = \exp\left(\int_0^x \dli{\xi} M(\xi)\right) G(0) 
        &\implies e^{xA} e^{xB} = \exp\left(xA + xB + \frac{x^2}{2}\commutator{A}{B}\right)\\
        &\implies \exp[x(A+B)] = \exp(xA) \exp(xB) \exp\left(- \frac{x^2}{2} \commutator{A}{B}\right),
    \end{align*}
    portanto para \(x = 1\) obtemos \(e^{A+B} = e^A e^B e^{-\frac12 \commutator{A}{B}}.\)

    Pelo \cref{lem:campbell}, temos
    \begin{equation*}
        \commutator{e^{\alpha sX}}{\beta Y} = \left(\sum_{n = 1}^\infty \frac{s^n}{n!} \commutator{\alpha X}{\beta Y}^{[n]}\right)e^{s \alpha X} = \beta \left(\sum_{n = 1}^\infty \frac{(\alpha s)^n}{n!} \commutator{X}{Y}^{[n]}\right)e^{s \alpha X},
    \end{equation*}
    para quaisquer operadores \(X\) e \(Y\) e parâmetros \(\alpha, \beta \ll 1\). Seja \(F(s) = e^{sX}e^{sY},\) então
    \begin{equation*}
        \diff{F}{s} = (\alpha X + \beta Y)F(s) + \commutator{e^{s \alpha X}}{\beta Y} e^{s \beta Y} = \underbrace{\left[\alpha X + \beta Y + \beta \sum_{n = 1}^\infty \frac{(\alpha s)^n}{n!} \commutator{X}{Y}^{[n]}\right]}_{K(s)} F(s).
    \end{equation*}
    Sendo \(Z(s)\) o operador de \(K(s)\) que contém termos de até segunda ordem em \(\alpha\) e \(\beta\),
    \begin{equation*}
        K(s) = \alpha X + \beta Y + \alpha \beta s \commutator{X}{Y} + Z(s),
    \end{equation*}
    observamos que
    \begin{align*}
        \commutator{K(s)}{K(r)} &= \commutator*{\alpha X + \beta Y + \alpha \beta s \commutator{X}{Y} + Z(s)}{\alpha X + \beta Y + \alpha \beta r \commutator{X}{Y} + Z(r)}\\
                                &= \alpha \beta \left(\commutator{X}{Y} + \commutator{Y}{X}\right) + \alpha \beta (r - s) \commutator*{\alpha X+\beta Y}{\commutator{X}{Y}} + O(\alpha^3 \beta) + O(\alpha^2 \beta^2)\\
                                &= O(\alpha^2 \beta).
    \end{align*}
    Como na aproximação considerada \(K\) comuta em parâmetros distintos, aproximamos a solução da equação diferencial \(F'(s) = K(s) F(s)\) por
    \begin{equation*}
        F(s) = \exp\left[\int_0^s \dli\zeta K(\zeta)\right] F(0) = \exp\left[s \alpha X + s \beta Y + \frac12 s^2 \alpha \beta \commutator{X}{Y} + \tilde{Z}(s)\right],
    \end{equation*}
    com \(\tilde{Z}(s) = \int_0^s \dli{\zeta} Z(\zeta).\) Assim, obtemos a expressão desejada tomando \(s = 1.\)
\end{proof}
