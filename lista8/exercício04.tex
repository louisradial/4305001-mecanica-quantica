% vim: spl=pt
\begin{exercício}{Estrutura hiperfina e efeito Zeeman para o positrônio}{ex4}
   Consideraremos aqui a estrutura fina e hiperfina do positrônio (\ce{Ps}). Como esse átomo consiste de duas partículas de mesma massa, não há distinção de fato entre a estrutura fina e a hiperfina, mas é costume chamar a quebra do estado fundamental de efeito de estrutura hiperfina.
   \begin{enumerate}[label=(\alph*)]
      \item Em \ce{Ps} os spins entram de forma equivalente, em contraste com outros átomos onde o spin da partícula satélite e do núcleo têm papeis dinâmicos diferentes. Assim os estados \ce{Ps} são chamados \(n^1L_J\) e \(n^3 L_j\) para o singleto e tripleto de spin, onde \(n\) é o número quântico principal, \(J\) o momento angular total e \(L\) o momento angular orbital em torno do centro de massa (com \(L = 0, 1, 2, \dots\) designando por \(s,p,d,\) etc.) Liste os estados nessa notação para \(n \leq 3.\)
      \item A correção principal para o estado fundamental, chamada de correção hiperfina (embora seja muito maior que a do átomo de hidrogênio), é dada pelo hamiltoniano
         \begin{equation*}
            H_\mathrm{ehf} = 4\pi \mu_B^2 \left[\frac38 + \frac76 \vetor{\sigma}^+ \cdot \vetor{\sigma}^-\right]\delta(\vetor{r}),
         \end{equation*}
         onde \(\vetor{\sigma}^{\pm}\) são os spins de \(e^{\pm}.\) Mostre que o desvio do nível fundamental é \(\Delta_{\mathrm{ehf}} = \frac76 \alpha^2 \mathrm{Ry}.\)
      \item Considere o efeito Zeeman para o para o multipleto \(n = 1\) de \ce{Ps}. Use a simetria da interação Zeeman para mostrar que apenas dois dos quatro estados participam do efeito. Mostre que as suas energias na presença do campo magnético são
         \begin{equation*}
         E_\pm = \frac12 \Delta_\mathrm{ehf} \left[1 \pm \sqrt{1 + \left(\frac{4 \mu_B B}{\Delta_\mathrm{ehf}}\right)^2}\right].
         \end{equation*}
   \end{enumerate}
\end{exercício}
\begin{proof}[Resolução]
    
\end{proof}
