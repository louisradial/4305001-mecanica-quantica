% vim: spl=pt
\begin{exercício}{Operador vetorial em um subespaço de momento angular bem definido}{ex5}
   Usando o fato que os elementos de matriz de um operador vetorial \(\vetor{V}\) em um subespaço de momento angular total definido são proporcionais aos elementos de matriz de \(\vetor{J},\) mostre que
   \begin{equation*}
      j(j +1) \bra{jm}\vetor{V}\ket{j m'} = \bra{jm} (\vetor{V} \cdot\vetor{J})\vetor{J} \ket{jm'}.
   \end{equation*}
\end{exercício}
\begin{proof}[Resolução]
   Como \(\vetor{V}\) é um operador vetorial, temos \(\vetor{V} = \lambda \vetor{J},\) com \(\lambda\) escalar. Tomando o produto escalar com o \(\vetor{J}\) e multiplicando a direita por \(\vetor{J},\) obtemos
   \begin{equation*}
      \vetor{V} \cdot \vetor{J} = \lambda \vetor{J}^2 \implies (\vetor{V} \cdot \vetor{J}) \vetor{J} = \lambda \vetor{J} \vetor{J}^2 = \vetor{V} \vetor{J}^2,
   \end{equation*}
   logo
   \begin{equation*}
      \bra{jm}(\vetor{V}\cdot\vetor{J})\vetor{J}\ket{jm'} = \bra{jm} \vetor{V} \vetor{J}^2 \ket{jm'} = j(j+1) \bra{jm}\vetor{V} \ket{jm'}
   \end{equation*}
   é a relação que determina os elementos de matriz de um operador vetorial no subespaço de momento angular total \(j.\)
\end{proof}
