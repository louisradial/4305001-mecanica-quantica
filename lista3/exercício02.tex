% vim: spl=pt
\begin{lemma}{Integral gaussiana}{gaussiana}
    Vale
    \begin{equation*}
       \int_{\mathbb{R}} \dli{\xi} \exp\left(-a \xi^2 + b \xi\right) = \sqrt{\frac{\pi}{a}} \exp\left(\frac{b^2}{4a}\right)
    \end{equation*}
    para todos \(b,c \in \mathbb{C}\) e \(a \in \setc{z \in \mathbb{C}\setminus\set{0}}{\Re(z) \geq 0}\).
\end{lemma}
\begin{proof}
    Notemos que
    \begin{equation*}
       -a \xi^2 + b \xi = - a \left(\xi^2 - \frac{b}{a}\xi\right) = -a \left(\xi - \frac{b}{2a}\right)^2 + \frac{b^2}{4a},
    \end{equation*}
    portanto
    \begin{align*}
       \int_{\mathbb{R}} \dli{\xi} \exp\left(-a \xi^2 + b \xi\right) &= \exp\left(\frac{b^2}{4a}\right) \int_{\mathbb{R}} \dli\xi \exp\left[-a \left(\xi - \frac{b}{2a}\right)^2\right]\\
                                                                     &= \exp\left(\frac{b^2}{4a}\right) \int_{\mathbb{R}} \dli\zeta \exp\left(-a \zeta^2\right)\\
                                                                     &= \sqrt{\frac{\pi}{a}} \exp\left(\frac{b^2}{4a}\right)
    \end{align*}
    como proposto.
\end{proof}

\begin{exercício}{Evolução temporal de um pacote de onda gaussiano}{ex2}
   Calcule \(\varphi(x,t)\), a função de onda de partícula livre evoluída no tempo, dado que
   \begin{equation*}
      \varphi(x,0) = \left(\frac{\sigma^2}{\pi}\right)^{\frac14}\exp\left(ikx - \frac12 \sigma^2 x^2\right),
   \end{equation*}
   onde \(\sigma, k \in \mathbb{R}\) são constantes.
\end{exercício}
\begin{proof}[Resolução]
   Notemos que para \(t > 0\) temos
   \begin{equation*}
      \varphi(x', t) = \bra{x'} U(t) \ket{\varphi} = \int_{\mathbb{R}} \dli{x''} \bra{x'} U(t) \ket{x''} \braket{x''}{\varphi} = \int_{\mathbb{R}} \dli{x''} K(x't; x'') \varphi(x'', 0),
   \end{equation*}
   onde \(K(x't'; x'' t'')\) é o propagador para uma partícula livre em uma dimensão, dado por
   \begin{equation*}
      K(x't'; x'' t'') = \sqrt{\frac{m}{2\pi i \hbar (t' - t'')}} \exp\left[\frac{im (x' - x'')^2}{2 \hbar (t' - t'')}\right] \theta(t' - t''),
   \end{equation*}
   como mostrado no \cref{ex:ex4}. Assim sendo, temos
   \begin{align*}
      \varphi(x, t) &= \sqrt{\frac{m}{2\pi i \hbar t}} \left(\frac{\sigma^2}{\pi}\right)^{\frac14}\int_{\mathbb{R}} \dli{\xi} \exp\left[-\frac12 \sigma^2 \xi^2 + ik\xi + \frac{i m (x - \xi)^2}{2 \hbar t}\right]\\
                    &= \sqrt{\frac{m}{2\pi i \hbar t}} \left(\frac{\sigma^2}{\pi}\right)^{\frac14}\exp\left(\frac{im x^2}{ 2 \hbar t}\right) \int_{\mathbb{R}} \dli\xi \exp\left[- \frac12\left(\sigma^2 + \frac{im}{\hbar t}\right)\xi^2 + i\left(k - \frac{mx}{\hbar t}\right)\xi\right]\\
                    &= \sqrt{\frac{m}{i \hbar t\left(\sigma^2 + \frac{im}{\hbar t}\right)}} \left(\frac{\sigma^2}{\pi}\right)^{\frac14}\exp\left[\frac{im x^2}{ 2 \hbar t} - \frac{\left(k - \frac{mx}{\hbar t}\right)^2}{2\left(\sigma^2 + \frac{im}{\hbar t}\right)}\right]
   \end{align*}
   como a evolução temporal da função de onda.
\end{proof}
