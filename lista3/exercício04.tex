% vim: spl=pt
\begin{lemma}{Função de transformação entre posição e momento}{transformada}
   Sendo \(\ket{x'}\) um autovetor de posição com \(x\ket{x'} = x' \ket{x'}\) e \(\ket{p'}\) um autovetor de momento com \(p\ket{p'} = p' \ket{p'},\) então
   \begin{equation*}
      \braket{x'}{p'} = \frac{1}{\sqrt{2\pi \hbar}} \exp\left(\frac{i p' q'}{\hbar}\right).
   \end{equation*}
\end{lemma}
\begin{proof}
   Recordemos\footnote{Ver \href{https://github.com/louisradial/4305001-mecanica-quantica/releases/tag/lista2}{Lema 6}.} que o operador translação espacial é o operador unitário dado por \(T_a = \exp\left(-\frac{i a p}{\hbar}\right)\) e que satisfaz \(T_a\ket{x'} = \ket{x' + a}.\) Analogamente, o operador de translação no espaço de momento \(K_k = \exp\left(\frac{i k x}{\hbar}\right)\) satisfaz \(K_k\ket{p'} = \ket{p' + k}.\) Assim, sendo \(x\ket{0_x} = 0\) e \(p \ket{0_p} = 0\), temos
   \begin{equation*}
      \braket{x'}{p'} = \bra{0_x} \herm{T}_{x'} \ket{p'} = \exp\left(\frac{i p' x'}{\hbar}\right)\braket{0_x}{p'} = \exp\left(\frac{i p' x'}{\hbar}\right)\braket{0_x}K_{p'}{0_p} = \exp\left(\frac{i p' x'}{\hbar}\right) \braket{0_x}{0_p}.
   \end{equation*}
   Da relação de ortogonalidade \(\braket{x'}{x''} = \delta(x'' - x'),\) temos
   \begin{equation*}
      \delta(x'' - x') = \int \dli{p'} \braket{x'}{p'}\braket{p'}{x''} = \abs*{\braket{0_x}{0_p}}^2\int \dli{p'} \exp\left(\frac{i p' (x'' - x')}{\hbar}\right) = 2\pi \hbar \abs*{\braket{0_x}{0_p}}^2 \delta(x'' - x'),
   \end{equation*}
   portanto podemos tomar \(\braket{0_x}{0_p} = \frac{1}{\sqrt{2\pi \hbar}}\), a menos de uma fase.
\end{proof}

\begin{exercício}{Propagador para uma partícula livre em uma dimensão}{ex4}
   Considere uma partícula livre se movendo em uma dimensão. Determine os propagadores 
   \begin{enumerate}[label=(\alph*)]
       \item \(K(x't'; xt) = \braket{x't'}{xt},\)
       \item \(K(x't'; pt) = \braket{x't'}{pt},\) e
       \item \(K(p't'; pt) = \braket{p't'}{pt}.\)
   \end{enumerate}
\end{exercício}
\begin{proof}[Resolução]
    Consideramos \(t' > t\) e omitiremos os fatores \(\theta(t' - t)\) das expressões, exceto no resultado final de cada propagador. Com isso,
    \begin{align*}
       K(x't'; xt) &= \bra{x'}\exp\left[\frac{(t' - t) p^2}{2i m \hbar}\right]\ket{x}\\
                   &= \int_{\mathbb{R}} \dli{\tilde{p}} \bra{x'}\exp\left[\frac{(t' - t) p^2}{2i m \hbar}\right]\ket{\tilde{p}}\braket{\tilde{p}}{x}\\
                   &= \int_{\mathbb{R}} \dli{\tilde{p}} \exp\left[\frac{(t' - t) \tilde{p}^2}{2i m \hbar}\right]\braket{x'}{\tilde{p}}\braket{\tilde{p}}{x}\\
                   &= \frac{1}{2\pi \hbar}\int_{\mathbb{R}} \dli{\tilde{p}}\exp\left[-\frac{i(t' - t) \tilde{p}^2}{2m \hbar} -i\frac{\tilde{p}(x' - x)}{\hbar}\right]\\
                   &= \sqrt{\frac{m}{2i\pi \hbar (t'-t)}}\exp\left[\frac{i m (x' - x)^2}{2\hbar(t' - t)}\right]\theta(t' - t),
    \end{align*}
    \begin{align*}
       K(x' t'; \tilde{p} t) &= \bra{x'} \exp\left[\frac{(t' - t) p^2}{2i m \hbar}\right]\ket{\tilde{p}}\\
                             &= \exp\left[\frac{(t' - t)\tilde{p}^2}{2i m \hbar}\right] \braket{x'}{\tilde{p}}\\
                             &= \frac{1}{\sqrt{2 \pi \hbar}} \exp\left[\frac{(t' - t) \tilde{p}^2}{2i m \hbar} + \frac{i \tilde{p} x'}{\hbar}\right] \theta(t' - t),
    \end{align*}
    e
    \begin{align*}
       K(p' t'; \tilde{p} t) &= \bra{p'}\exp\left[\frac{(t' - t) p^2}{2i m \hbar}\right]\ket{\tilde{p}}\\
                             &= \exp\left[\frac{(t' - t) p'^2}{2i m \hbar}\right] \braket{p'}{\tilde{p}}\\
                             &= \exp\left[\frac{(t' - t) p'^2}{2i m \hbar}\right] \delta(\tilde{p} - p') \theta(t' - t)
    \end{align*}
    são as expressões para os propagadores para a partícula livre.
\end{proof}
