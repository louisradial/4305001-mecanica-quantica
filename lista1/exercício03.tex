\begin{exercício}{Derivada de um operador}{ex3}
    Seja \(A(x)\) um operador que depende de um parâmetro contínuo \(x\). Defina sua derivada por
    \begin{equation*}
       \diff{A}{x} \equiv A'(x) = \lim_{\epsilon \to 0}{\frac{A(x + \epsilon) - A(x)}{\epsilon}}.
    \end{equation*}
    Mostre que a derivada satisfaz a regra de Leibniz, 
    \begin{equation*}
       \diff{(AB)}{x} = \diff{A}{x} B + A \diff{B}{x},
    \end{equation*}
    e que se \(A\) possuir inverso então vale
    \begin{equation*}
       \diff{(A^{-1})}{x} = -A^{-1} \diff{A}{x}A^{-1}.
    \end{equation*}
\end{exercício}
\begin{proof}[Resolução]
    Sejam \(A, B\) operadores que dependem de um parâmetro contínuo e seja \(C = AB\), então para todo \(x\) no domínio e para \(\varepsilon\) suficientemente pequeno, temos
    \begin{align*}
       C(x + \varepsilon) - C(x) &= A(x + \varepsilon)B(x + \varepsilon) - A(x) B(x)\\
                                 &= A(x + \varepsilon) B(x + \varepsilon) - A(x + \varepsilon) B(x) + A(x + \varepsilon) B(x) - A(x) B(x)\\
                                 &= A(x + \varepsilon) \left[B(x + \varepsilon) - B(x)\right] + \left[A(x + \varepsilon) - A(x)\right]B(x),
    \end{align*}
    portanto
    \begin{align*}
       \diff{C}{x} &= \lim_{\epsilon \to 0}{\left[A(x + \epsilon) \frac{B(x + \epsilon) - B(x)}{\epsilon} + \frac{A(x + \epsilon) - A(x)}{\epsilon} B(x)\right]}\\
                   &= \lim_{\epsilon \to 0}{\left[A(x + \epsilon) \frac{B(x + \epsilon) - B(x)}{\epsilon}\right]} + \lim_{\epsilon \to 0}{\left[\frac{A(x + \epsilon) - A(x)}{\epsilon} B(x)\right]}\\
                   &= \lim_{\epsilon \to 0}{\left[A(x + \epsilon)\right]}\lim_{\epsilon \to 0}{\left[\frac{B(x + \epsilon) - B(x)}{\epsilon}\right]} + \lim_{\epsilon \to 0}{\left[\frac{A(x + \epsilon) - A(x)}{\epsilon}\right]}\lim_{\epsilon \to 0}{\left[B(x)\right]}\\
                   &= A(x) B'(x) + A'(x)B(x),
    \end{align*}
    isto é, a derivada definida satisfaz a regra de Leibniz.

    Se \(A\) é um operador invertível, então \(A^{-1} A = A A^{-1} = \unity\). Notemos que
    \begin{equation*}
       \diff{\unity}{x} = \lim_{\epsilon \to 0}{\frac{\unity(x + \epsilon) - \unity}{\epsilon}} = \lim_{\epsilon \to 0}{\frac{\unity - \unity}{\epsilon}} = 0,
    \end{equation*}
    portanto da regra de Leibniz segue que
    \begin{align*}
       0 = \diff{\unity}{x} = \diff{(AA^{-1})}{x} = 
       \diff{A}{x} A^{-1} + A \diff{(A^{-1})}{x} &\implies A \diff{(A^{-1})}{x} = - \diff{A}{x} A^{-1}\\
                                                 &\implies A^{-1} A \diff{(A^{-1})}{x} = - A^{-1} \diff{A}{x} A^{-1}\\
                                                 &\implies \diff{(A^{-1})}{x} = - A^{-1} \diff{A}{x} A^{-1},
    \end{align*}
    como desejado.
\end{proof}
