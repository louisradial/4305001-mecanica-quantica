\begin{exercício}{Derivada de um operador}{ex3}
    Seja \(A(x)\) um operador que depende de um parâmetro contínuo \(x\). Defina sua derivada por
    \begin{equation*}
       \diff{A}{x} \equiv A'(x) = \lim_{\epsilon \to 0}{\frac{A(x + \epsilon) - A(x)}{\epsilon}}.
    \end{equation*}
    Mostre que a derivada satisfaz a regra de Leibniz, 
    \begin{equation*}
       \diff{(AB)}{x} = \diff{A}{x} B + A \diff{B}{x},
    \end{equation*}
    e que se \(A\) possuir inverso então vale
    \begin{equation*}
       \diff{(A^{-1})}{x} = -A^{-1} \diff{A}{x}A^{-1}.
    \end{equation*}
\end{exercício}
\begin{proof}[Resolução]
    
\end{proof}
