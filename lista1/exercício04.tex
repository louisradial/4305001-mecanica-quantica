\begin{exercício}{}{ex4}
    Suponha que tenhamos um sistema com momento angular total 1. Escolha uma base correspondente aos três autovetores da componente \(z\) do momento angular, \(J_z\), com autovalores \(+1, 0, -1\), respectivamente. Seja um conjunto descrito pela matriz densidade
    \begin{equation*}
        \rho = \frac14 \begin{pmatrix}
           2 && 1 && 1\\
           1 && 1 && 0\\
           1 && 0 && 1
        \end{pmatrix}.
    \end{equation*}
    \begin{enumerate}[label=(\alph*)]
        \item \(\rho\) é uma matriz densidade admissível? Explique e determine se \(\rho\) descreve um estado puro ou de mistura.
        \item Dado o conjunto descrito por \(\rho\), qual o valor médio de \(J_z\)?
        \item Qual o desvio padrão para uma medida de \(J_z\)?
    \end{enumerate}
\end{exercício}
\begin{proof}[Resolução]
    
\end{proof}
