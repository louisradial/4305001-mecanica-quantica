\begin{exercício}{Operadores em \(\ell_2(\mathbb{C})\)}{ex2}
   Considere \(\ell_2(\mathbb{C})\), o espaço euclidiano  de dimensão infinita de quadrado somável dotado do produto escalar \(\braket{x}{y} = \sum_{n \in \mathbb{N}} \conj{x}_n y_n\). Para as aplicações definidas em \(\ell_2\) a seguir, determine se
   \begin{enumerate}[label=(\alph*)]
       \item é uma aplicação linear;
       \item é um operador limitado;
       \item é um operador hermitiano;
       \item a equação homogênea admite soluções não triviais e se a equação não homogênea sempre admite solução.
   \end{enumerate}
   As aplicações consideradas são
   \begin{align*}
      C : \ell_2(\mathbb{C}) &\to \ell_2(\mathbb{C})&
      G : \ell_2(\mathbb{C}) &\to \ell_2(\mathbb{C})\\
      (x_1, x_2, \dots) &\mapsto (0, x_1, x_2, \dots)&
      (x_1, x_2, \dots) &\mapsto \left(x_1, \frac{x_2}{2}, \frac{x_3}{3}, \dots\right).
   \end{align*}
\end{exercício}
\begin{proof}[Resolução]
    
\end{proof}
