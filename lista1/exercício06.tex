\begin{exercício}{Magnetização de um sistema na presença de um campo magnético externo}{ex6}
   Considere um sistema de \(N\) partículas de spin \(\frac12\) por unidade de volume em equilíbrio térmico, em um campo magnético externo \(\vetor{B}\). Lembre-se que a distribuição canônica é
   \begin{equation*}
      \rho = \frac{\exp(- \beta H)}{Z},
   \end{equation*}
   onde \(\beta = \frac{1}{k T}\) e a função de partição é dada por \(Z = \Tr(e^{-\beta H})\). Tal sistema de partículas tenderá a se orientar ao longo do campo magnético, resultando em uma magnetização \(\vetor{M}\).
   \begin{enumerate}[label=(\alph*)]
      \item Forneça uma expressão para \(\vetor{M}\).
      \item Qual a magnetização no limite ded altas temperaturas (em ordem mais baixa)?
   \end{enumerate}
\end{exercício}
\begin{proof}[Resolução]
    
\end{proof}
