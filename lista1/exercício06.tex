% vim: spl=pt
\begin{exercício}{Magnetização de um sistema na presença de um campo magnético externo}{ex6}
   Considere um sistema de \(N\) partículas de spin \(\frac12\) por unidade de volume em equilíbrio térmico, em um campo magnético externo \(\vetor{B}\). Lembre-se que a distribuição canônica é
   \begin{equation*}
      \rho = \frac{\exp(- \beta H)}{Z},
   \end{equation*}
   onde \(\beta = \frac{1}{k T}\) e a função de partição é dada por \(Z = \Tr(e^{-\beta H})\). Tal sistema de partículas tenderá a se orientar ao longo do campo magnético, resultando em uma magnetização \(\vetor{M}\).
   \begin{enumerate}[label=(\alph*)]
      \item Forneça uma expressão para \(\vetor{M}\).
      \item Qual a magnetização no limite de altas temperaturas (em ordem mais baixa)?
   \end{enumerate}
\end{exercício}
\begin{proof}[Resolução]
   Orientemos o sistema de coordenadas de forma que \(\vetor{B} = B \vetor{e}_z,\) então o Hamiltoniano é dado por \(H = - B \mu_z\), onde \(\mu_z = \frac12 \gamma \sum_{j = 1}^N \sigma_z^{(j)}\) em que 
   \begin{equation*}
      \sigma_z^{(j)} = \underbrace{\unity \otimes \unity \otimes \dots \otimes \unity}_{j-1} \otimes \sigma_z \otimes \unity \otimes \unity \dots \otimes \unity.
   \end{equation*}
   Na base \(\set{\ket{+}, \ket{-}}\) que diagonaliza \(\sigma_z\) com \(\sigma_z\ket{m} = m \ket{m}\), temos
   \begin{equation*}
      \mu_z \ket{m_1 m_2 \dots m_N} = \frac12 \gamma \sum_{j = 1}^N \sigma_z^{(j)} \ket{m_1 m_2 \dots m_N} = \frac12 \gamma \sum_{j = 1}^N m_j \ket{m_1 m_2 \dots m_N},
   \end{equation*}
   onde \(\ket{m_1 m_2 \dots m_N} = \bigotimes_{j = 1}^N \ket{m_j}\). Denotando \(\set{m_j}^N\) como a palavra \(\set{m_1, m_2, \dots m_N}\) em que cada \(m_j\) pode tomar os valores \(\pm 1\), a função de partição é dada por
   \begin{align*}
      Z = \Tr(e^{- \beta H}) &= \sum_{\set{m_j}^N} \bra{m_1m_2 \dots m_N} e^{\beta B \mu_z} \ket{m_1 m_2 \dots m_N}\\
                             &= \sum_{\set{m_j}^N} \bra{m_1 m_2 \dots m_N} \exp\left(\frac12 \beta \gamma B \sum_{k = 1}^{N}{m_k}\right)\ket{m_1 m_2 \dots m_N}\\
                             &= \sum_{\set{m_j}^N} \exp\left(\frac12 \beta \gamma B \sum_{k = 1}^N m_k\right)\\
                             &= \sum_{\set{m_j}^N} \prod_{k = 1}^N \exp\left(\frac12 \beta \gamma B m_k\right)\\
                             &= \sum_{m_1 \in \set{\pm1}}{\exp\left(\frac12 \beta \gamma B m_1\right)}\sum_{m_2 \in \set{\pm1}}{\exp\left(\frac12 \beta \gamma B m_2\right)}\dots \sum_{m_N \in \set{\pm1}}{\exp\left(\frac12 \beta \gamma B m_N\right)}\\
                             &= \prod_{k = 1}^N \sum_{m \in \set{\pm1}} \exp\left(\frac12 \beta \gamma B m\right)\\
                             &= \left[\exp\left(-\frac12 \beta \gamma B\right) + \exp\left(\frac12 \beta \gamma B\right)\right]^N\\
                             &= \left[2 \cosh\left(\frac12 \beta \gamma B\right)\right]^N
   \end{align*}
   Como \(\bra{m} \sigma_x \ket{m} = \braket{m}{-m} = 0\) e \(\bra{m} \sigma_y \ket{m} = im \braket{m}{-m} = 0\), notemos que
   \begin{equation*}
      \bra{m_1\dots m_N} \mu_x \rho \ket{m_1 \dots m_N} = \frac{\exp\left(\frac12 \beta \gamma B \sum_{j = 1}^N m_j\right)}{Z}\bra{m_1\dots m_N} \mu_x \ket{m_1\dots m_N} = 0
   \end{equation*}
   e analogamente para \(\mu_y,\) portanto \(\mean{\mu_x}_\rho = \mean{\mu_y}_{\rho} = 0.\) Dessa forma, \(\mean{\vetor{\mu}}_\rho = \mean{\mu_z}_{\rho} \vetor{e}_z\) e temos
   \begin{align*}
      \Tr(\rho \mu_z) &= \sum_{\set{m_j}^N} \bra{m_1 \dots m_N}\rho \mu_z \ket{m_1 \dots m_N}\\
                      &= \frac12 \gamma \sum_{\set{m_j}^N} \sum_{k = 1}^N m_k \bra{m_1 \dots m_N} \rho \ket{m_1 \dots m_N}\\
                      &= \frac{\gamma}{2Z} \sum_{\set{m_j}^N} \sum_{k = 1}^N m_k \exp\left(\frac12 \beta \gamma B \sum_{\ell = 1}^N m_\ell\right)\\
                      &= \frac{1}{\beta Z} \sum_{\set{m_j}^N} \left(\frac12 \beta \gamma \sum_{k = 1}^N m_k\right)\exp\left(\frac12 \beta \gamma B\sum_{\ell = 1}^N m_{\ell}\right) \\
                      &= \frac{1}{\beta Z} \sum_{\set{m_j}^N} \diffp*{\exp\left(\frac12 \beta \gamma B \sum_{\ell = 1}^N m_{\ell}\right)}{B}\\
                      &= \frac{1}{\beta Z} \diffp*{\underbrace{\left[\sum_{\set{m_j}^N} \exp\left(\frac12 \beta \gamma B \sum_{\ell = 1}^N m_{\ell}\right)\right]}_{Z}}{B}\\
                      &= \frac{1}{\beta} \diffp{\ln Z}{B}\\
                      &= \frac{N \gamma}{2} \tanh\left(\frac12 \beta \gamma B\right).
   \end{align*}
   Com este resultado, vemos que a magnetização é dada por \(\vetor{M} = \mean{\vetor{\mu}}_\rho = \frac12 N \gamma \tanh\left(\frac12 \beta \gamma B\right)\vetor{e}_z,\) ou então
   \begin{equation*}
      \vetor{M} = \frac{N\gamma}{2 \norm{\vetor{B}}} \tanh\left(\frac12 \beta \gamma \norm{\vetor{B}}\right) \vetor{B}.
   \end{equation*}
   Usando a expansão
   \begin{equation*}
      \tanh \xi = \tanh(0) + \frac{1}{\cosh^2(0)} \xi + O(\xi^2) = \xi + O(\xi^2),
   \end{equation*}
   temos
   \begin{equation*}
      \vetor{M} = \frac{N \gamma^2}{4k_B T}\vetor{B}
   \end{equation*}
   como a expressão assintótica para a magnetização no limite de altas temperaturas.
\end{proof}
