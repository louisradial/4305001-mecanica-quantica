\begin{exercício}{Vetor de polarização de uma partícula em um campo magnético externo}{ex5}
   Considere a aplicação do formalismo de matriz densidade para o problema de uma partícula de spin \(\frac12\) em um campo magnético estático \(\vetor{B}\). Em geral, a partícula com spin terá ambém um momento magnético, orientado ao longo da direção do spin. Para partículas de spin \(\frac12\), o operador de momento magnético tem a forma \(\vetor{\mu} = \frac12 \gamma \vetor{\sigma}\), onde \(\vetor{\sigma}\) são as matrizes de Pauli e \(\gamma\) é uma constante chamada de fator giromagnético. Lembre-se que o Hamiltoniano do sistema é \(H = -\vetor{\mu}\cdot \vetor{B}\).A partícula de spin \(\frac12\) pode ter uma orientação de spin, ou \emph{vetor de polarização}, dado por \(\vetor{P} = \mean{\vetor{\sigma}}.\) Qual é a variação temporal, \(\diff{\vetor{P}}{t}\), do vetor de polarização? Expresse sua resposta da forma mais simplies que puder, sem assumir a pureza do estado.
\end{exercício}
\begin{proof}[Resolução]
    
\end{proof}
