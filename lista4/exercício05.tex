% vim: spl=pt
\begin{exercício}{Fórmula de Rodrigues para representações de \(\mathrm{SO}(3).\)}{ex5}
   Considere uma rotação por um ângulo \(\theta\) em torno do eixo definido pelo vetor unitário \(\vetor{n}\) que pode ser descrita pela matriz
   \begin{equation*}
      R_{\vetor{n}}(\theta) = \exp\left(-i\theta \vetor{T}\cdot \vetor{n}\right).
   \end{equation*}
   \begin{enumerate}[label=(\alph*)]
      \item Mostre que \(\Tr R_{\vetor{n}}(\theta) = 1 + 2 \cos\theta.\)
      \item Use o fato que as componentes de um vetor \(\vetor{V}\) sobre essa rotação se transformam como \(\tilde{V}_i = [R_{\vetor{n}}(\theta)]_{ij} V_j + \theta \epsilon_{ijk} n_j V_k + O(\theta^2)\) para determinar uma representação para os operadores hermitianos \(T_i\).
      \item Mostre que \((\vetor{T}\cdot \vetor{n})^3 = \vetor{T}\cdot \vetor{n}.\)
      \item Mostre que \(\exp\left(-i\theta\vetor{T}\cdot\vetor{n}\right) = \unity - i \sin\theta (\vetor{T} \cdot \vetor{n}) - (1 - \cos\theta) (\vetor{T}\cdot \vetor{n})^2.\)
   \end{enumerate}
\end{exercício}
\begin{proof}[Resolução]
    
\end{proof}
