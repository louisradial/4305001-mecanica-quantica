% vim: spl=pt
\begin{exercício}{Função de green para um poço de potencial infinito unidimensional}{ex3}
   Considere uma partícula em uma dimensão confinada a \(x \in [a,b].\)
   \begin{enumerate}[label=(\alph*)]
       \item Construa a função de Green \(G(x,y)\).
       \item Determine o espectro de \(H\).
       \item Note que, usando
          \begin{equation*}
             G(x,y) = \sum_{k = 1}^\infty \frac{\phi_k(x) \conj{\phi}(y)}{\omega_k - z},
          \end{equation*}
          o autoestado \(\phi_k(x)\) normalizado pode ser obtido avaliando o resíduo de \(G\) no polo \(z = \omega_k\). Faça esse cálculo e verifique que seu resultado está corretamente normalizado.
       \item Mostre que
          \begin{equation*}
             G(x,y) = i \sqrt{\frac{m}{2z}} e^{i \rho \abs{x - y}}
          \end{equation*}
          no limite \(a \to -\infty\) e \(b \to \infty\).
   \end{enumerate}
\end{exercício}
\begin{proof}[Resolução]
    
\end{proof}

