% vim: spl=pt
\begin{exercício}{Propagador de partícula livre no formalismo de integral de trajetória}{ex1}
    Determine o propagador de partícula livre em uma dimensão \(K(x't', x''t'')\) usando integrais de trajetória.
    \begin{enumerate}[label=(\alph*)]
       \item Escreva a única trajetória clássica \(x_{\mathrm{cl}}(t)\) que liga \((x'',t'')\) a \((x',t')\) e calcule \(S_{\mathrm{cl}}(x't',x''t'').\)
       \item Mostre que se \(y(t) = x(t) - x_{\mathrm{cl}}(t)\), então
          \begin{equation*}
             S[x(t)] = S[x_{\mathrm{cl}(t)} + y(t)] = \frac12m\int_{t''}^{t'} \dli{t} \left[\dot{x}_{\mathrm{cl}}^2(t) + \dot{y}^2(t)\right].
          \end{equation*}
       \item Mostre que agora podemos escrever
          \begin{equation*}
             K(x't', x''t'') = F(t' - t'') \exp\left(\frac{i}{\hbar} S_{\mathrm{cl}}(x't', x''t'')\right),
          \end{equation*}
          onde
          \begin{equation*}
             F(\Delta t) = \Lim_{\substack{\kappa \to \infty\\ \tau \to 0\\ \kappa \tau = \Delta t}}{\left(\frac{m}{2\pi i \hbar \tau}\right)^{\frac\kappa2}\int_{\mathbb{R}} \dli{y_{\kappa - 1}} \dots \int_{\mathbb{R}} \dli{y_1} \exp\left[\frac{im}{2 \hbar \tau} \sum_{j = 0}^{\kappa - 1}(y_{k+1} - y_k)^2\right]}.
          \end{equation*}
       \item Mostre que a forma quadrática anterior pode ser diagonalizada levando ao resultado
          \begin{equation*}
             F(\Delta t) = \Lim_{\substack{\kappa \to \infty\\ \tau \to 0\\ \kappa \tau = \Delta t}} \sqrt{\frac{m}{2\pi i \hbar \tau \det \Lambda}},
          \end{equation*}
          onde \(\Lambda\) é a matriz diagonalizada.
       \item Mostre que \(\det \Lambda = 1 + \kappa.\)
       \item Encontre a expressão para o propagador \(K(x't', x''t'').\)
    \end{enumerate}
\end{exercício}
\begin{proof}[Resolução]
   Denotemos \(\Delta x = x' - x''\) e \(\Delta t = t' - t''\). Seja a ação \(S\) dada pelo funcional
   \begin{equation*}
      S[x(t)] = \frac12 m \int_{t''}^{t'} \dli{t} \dot{x}^2,
   \end{equation*}
   com domínio dado pelo conjunto de trajetórias \(x(t)\) que satisfazem \(x(t'') = x''\) e \(x(t') = x'\) e seja a ação \(\tilde{S}\) dada pelo mesmo funcional com domínio dado pelo conjunto de desvios \(\eta(t)\) que satisfazem \(\eta(t') = \eta(t'') = 0\). Para um desvio \(\eta\) e uma trajetória \(x,\) segue que \(x + \eta\) está no domínio de \(S,\) e temos
   \begin{align*}
      S[x(t) + \eta(t)] &= m \int_{t''}^{t'} \dli{t} \left(\frac12 \dot{x}^2 + \dot{x} \dot\eta + \dot{\eta}^2\right)\\
                        &= S[x(t)] + \tilde{S}[\eta(t)] - m\left[x(t) \eta(t)\right]_{t''}^{t'} - m \int_{t''}^{t'} \dli{t} \ddot{x} \eta\\
                        &= S[x(t)] + \tilde{S}[\eta(t)] - m\int_{t''}^{t'} \dli{t} \ddot{x} \eta.
   \end{align*}
   Tomando \(\eta\) infinitesimal, temos \(\tilde{S}[\eta(t)] \to 0\) e vemos que \(\diff.d.{S}{x} = -m\ddot{x}\). Assim, a trajetória clássica \(q(t)\) de \((t'', x'')\) até \((t', x')\) satisfaz
   \begin{align*}
      \diff.d.{S}{x}[x(t) = q(t)] = 0 &\implies \ddot{q}(t) = 0\\
                                      &\implies \dot{q}(t) = v\\
                                      &\implies q(t) = x'' + v(t - t''),
   \end{align*}
   onde \(v = \frac{\Delta x}{\Delta t}\). Assim, a ação para a trajetória clássica é dada por
   \begin{align*}
      S_{\mathrm{cl}} &= S[q(t)] = \frac12 m\int_{t''}^{t'} \dli{t} \dot{q}^2(t)\\
                      &= \frac12 m \int_{t''}^{t'} \dli{t} v^2\\
                      &= \frac12 m v^2 \Delta t\\
                      &= \frac12 m \frac{\Delta x^2}{\Delta t}.
   \end{align*}
   Para uma trajetória arbitrária \(x(t)\) com \(x(t') = x'\) e \(x(t'') = x'',\) podemos escrever \(x(t) = q(t) + y(t),\) onde \(y(t') = y(t'') = 0,\) portanto
   \begin{equation*}
      S[x(t)] = S[q(t)] + \tilde{S}[y(t)] - m \int_{t''}^{t'}\dli{t} \ddot{q} y = S_{\mathrm{cl}} + \tilde{S}[y(t)],
   \end{equation*}
   já que a derivada funcional de \(S\) se anula para \(q\). Assim, o propagador é dado pela integral de trajetória
   \begin{align*}
      K(x't'; x''t'') &= \int_{(x'', t'')}^{(x', t')} \dlp{x(t)} \exp\left(\frac{i}{\hbar} S[x(t)]\right)\\
                      &= \exp\left(\frac{i}{\hbar}S_{\mathrm{cl}}\right)\underbrace{\int_{(0,t'')}^{(0,t')} \dlp{y(t)} \exp\left(\frac{i}{\hbar} \tilde{S}[y(t)]\right)}_{F( t' - t'')},
   \end{align*}
   onde \(F(t' - t'')\) é o propagador de \((0,t'')\) até \((0, t'),\) que é igual ao propagador de \((0,0)\) até \((0, t' - t''),\) já que o Hamiltoniano não depende explicitamente do tempo, e é dado por
   \begin{equation*}
      F(\Delta t) = \Lim_{\substack{\kappa \to \infty\\ \tau \to 0\\ \kappa \tau = \Delta t}}{\left(\frac{m}{2\pi i \hbar \tau}\right)^{\frac\kappa2}\int_{\mathbb{R}} \dli{y_{\kappa - 1}} \dots \int_{\mathbb{R}} \dli{y_1} \exp\left[\frac{im}{2 \hbar \tau} \sum_{j = 0}^{\kappa - 1}(y_{j+1} - y_j)^2\right]}
   \end{equation*}
   pela definição de integral de trajetória.

   Consideramos a matriz \(\Lambda\) quadrada \((\kappa-1) \times (\kappa-1)\) cujos elementos são dados por
   \begin{equation*}
      \Lambda_{ij} = -\delta^{i + 1}_{j}  + 2\delta^{i}_{j} - \delta^{i - 1}_{j}
   \end{equation*}
   para \(i,j \in \set{1, \dots, \kappa-1}.\) Notemos que
   \begin{align*}
      \sum_{j = 0}^{\kappa - 1} (y_{j + 1} - y_{j})^2 
      &= \sum_{j = 0}^{\kappa - 1} y_{j + 1}^2 + \sum_{j = 0}^{\kappa - 1} y_j^2 - \sum_{j = 0}^{\kappa - 1} y_{j} y_{j + 1} - \sum_{j = 0}^{\kappa -1}\\
      &= \sum_{j = 1}^{\kappa} y_j^2 + \sum_{j = 1}^{\kappa - 1} y_j^2 - \sum_{j = 1}^{\kappa - 1} y_j y_{j+1} - \sum_{j = 1}^{\kappa} y_{j-1} y_j\\
      &= \sum_{j = 1}^{\kappa-1} y_j^2 + \sum_{j = 1}^{\kappa - 1} y_j^2 - \sum_{j = 1}^{\kappa - 1} y_j y_{j+1} - \sum_{j = 1}^{\kappa-1} y_{j-1} y_j\\
      &= \sum_{j = 1}^{\kappa - 1} (2 y_j^2 - y_j y_{j+1} - y_{j-1} y_j)\\
      &= \sum_{j = 1}^{\kappa - 1} y_j \sum_{i = 0}^{\kappa - 1} (2 \delta^i_j - \delta^{i-1}_{j} - y^{i + 1}_{j})y_i\\
      &= \sum_{j = 1}^{\kappa - 1} \sum_{i = 1}^{\kappa - 1} y_j \Lambda_{ij} y_i,
   \end{align*}
   onde usamos que \(y_0 =  y(t'') = 0 = y(t') = y_\kappa\). Notemos que \(\Lambda\) é simétrica e real, portanto é real-diagonalizável, isto é, existe uma matriz ortogonal \(M\) e uma matriz diagonal \(\lambda = \operatorname{diag}(\set{\lambda_j})\) tais que \(\Lambda = M\lambda M^{\top}\), onde \(\lambda_k\) são os autovalores de \(\Lambda.\) Na integral de trajetória, consideramos a mudança de variável \(\eta = M^\top y,\) de forma que 
   \begin{equation*}
      \sum_{j = 1}^{\kappa - 1} (y_{j+1} - y_j)^2 = \inner{y}{\Lambda y} =\inner{y}{M \lambda M^\top y} = \inner{M^\top y}{\lambda \eta} = \inner{\eta}{\lambda \eta} = \sum_{j = 1}^{\kappa - 1} \lambda_j \eta_j^2
   \end{equation*}
   e que o determinante do jacobiano da transformação é unitário, já que \(\diffp{\eta_j}{y_k} = M_{jk}\) e \(M\) é ortogonal. Assim, obtemos
   \begin{align*}
      F(\Delta t) &= \Lim_{\substack{\kappa \to \infty\\ \tau \to 0\\ \kappa \tau = \Delta t}}{\left(\frac{m}{2\pi i \hbar \tau}\right)^{\frac\kappa2}\int_{\mathbb{R}} \dli{\eta_{\kappa - 1}} \dots \int_{\mathbb{R}} \dli{\eta_1} \exp\left[\frac{im}{2 \hbar \tau} \sum_{j = 1}^{\kappa - 1} \lambda_j \eta_j^2\right]}\\
                  &= \Lim_{\substack{\kappa \to \infty\\ \tau \to 0\\ \kappa \tau = \Delta t}}{\left(\frac{m}{2\pi i \hbar \tau}\right)^{\frac\kappa2}\int_{\mathbb{R}} \dli{\eta_{\kappa - 1}} \exp\left[\frac{im \lambda_{\kappa - 1} \eta_{\kappa - 1}^2}{2 \hbar \tau}\right]\dots \int_{\mathbb{R}} \dli{\eta_1} \exp\left[\frac{im \lambda_1 \eta_{1}^2}{2 \hbar \tau}\right]}\\
                  &= \Lim_{\substack{\kappa \to \infty\\ \tau \to 0\\ \kappa \tau = \Delta t}}{\left(\frac{m}{2\pi i \hbar \tau}\right)^{\frac\kappa2} \prod_{j = 1}^{\kappa - 1} \int_{\mathbb{R}} \dli{\eta} \exp\left[-\frac{m \lambda_j \eta^2}{2i \hbar \tau}\right]}\\
                  &= \Lim_{\substack{\kappa \to \infty\\ \tau \to 0\\ \kappa \tau = \Delta t}}{\left(\frac{m}{2\pi i \hbar \tau}\right)^{\frac\kappa2} \prod_{j = 1}^{\kappa - 1} \sqrt{\frac{2\pi i\hbar \tau}{m \lambda_j}}}\\
                  &= \Lim_{\substack{\kappa \to \infty\\ \tau \to 0\\ \kappa \tau = \Delta t}}{\sqrt{\frac{m}{2\pi i \hbar \tau \det \Lambda}}},
   \end{align*}
   pois \(\det{\Lambda} = \det{M \lambda M^\top} = \det{\lambda} = \prod_{j = 1}^{\kappa - 1}{\lambda_{j}}.\)

   Mostraremos por indução em \(\kappa\) que \(\det{\Lambda} = \kappa\) para todo \(\kappa \geq 2\). Denotaremos \(\Lambda_{\kappa}\) como a matriz \(\Lambda\) para um dado \(\kappa\) para deixar claro qual a dimensão considerada. Notemos que para \(m > 2\) vale
   \begin{align*}
      \det \Lambda_{m + 1} &= \det \begin{pmatrix}
         2 && -1 && 0 && \dots\\
         -1 && 2 && -1 && \dots\\
         0 && -1 && 2 && \dots\\
         \vdots && \vdots && \vdots && \ddots
      \end{pmatrix}_{m \times m}\\
                           &= 2 \det \Lambda_m - (-1) \det \begin{pmatrix}
                              -1 && -1 && 0 && \dots\\
                              0 && 2 && -1 && \dots\\
                              0 && -1 && 2 && \dots\\
                              \vdots && \vdots && \vdots && \ddots
                           \end{pmatrix}_{(m-1) \times (m-1)}\\
                           &= 2\det \Lambda_m - \det \Lambda_{m -1}
      \end{align*}
      Para \(\kappa = 2,\) temos \(\Lambda_2 = [2]\) e para \(\kappa = 3,\) temos \(\Lambda_3 = \left(\begin{smallmatrix}
            2 && -1\\
            -1 && 2
      \end{smallmatrix}\right),\) portanto \(\det{\Lambda_2} = 2\) e \(\det{\Lambda_3} = 3,\) de acordo com a expressão afirmada. Assumindo válida para todo \(3 < n < m,\) temos pela relação de recorrência que
      \begin{equation*}
         \det \Lambda_{m} = 2 \det{\Lambda_{m - 1}} - \det{\Lambda_{m - 2}} = 2(m - 1) - (m - 2) = m,
      \end{equation*}
      portanto a expressão vale para \(m.\) Pelo princípio de indução completa, concluímos que \(\det \Lambda = \kappa\) para todo \(\kappa \geq 2.\) Com isso, concluímos que
      \begin{align*}
         K(x't'\xi''t'') &= \exp\left(\frac{i}{\hbar} S_{\mathrm{cl}}\right)F(t'' - t') \Lim_{\substack{\kappa \to \infty\\ \tau \to 0\\ \kappa \tau = t'' - t}}{\sqrt{\frac{m}{2\pi i \hbar \tau \det \Lambda}}}\\
                         &= \exp\left[\frac{i m (x' - x'')^2}{2 \hbar (t' - t'')}\right] \Lim_{\substack{\kappa \to \infty\\ \tau \to 0\\ \kappa \tau = t'' - t}}{\sqrt{\frac{m}{2\pi i \hbar \tau \kappa}}}\\
                         &= \sqrt{\frac{m}{2\pi i \hbar (t' - t'')}}\exp\left[\frac{i m (x' - x'')^2}{2 \hbar (t' - t'')}\right]
      \end{align*}
      é a expressão para o propagador para a partícula livre.
   \end{proof}
