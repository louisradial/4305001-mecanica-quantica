% vim: spl=pt
\begin{exercício}{Efeito Stark no multipleto \(n = 2\) do átomo de hidrogênio}{ex3}
   Vamos discutir aqui o efeito Stark no multipleto \(n = 2\) do átomo de hidrogênio. Como o desvio de Lamb é pequeno comparado ao desvio causado pela estrutura hiperfina, o espectro possui várias características interessantes.
   \begin{enumerate}[label=(\alph*)]
      \item Calcule os elementos de matriz de \(e E z\) na base que diagonaliza \(H_{\mathrm{ef}}\) (estrutura fina). Mostre que para \(\abs{m} = \frac32\) os estados não são afetados, enquanto que as energias perturbadas e autoestados do espaço \(\abs{m} = \frac12\) podem ser encontrados diagonalizando
         \begin{equation*}
            \mathcal{H} = \begin{pmatrix}
               0 && \sqrt{3} \epsilon && 0\\
               \sqrt{3} \epsilon && \Delta_1 && - \sqrt{6} \epsilon\\
               0 && - \sqrt{6} \epsilon && \Delta_2
            \end{pmatrix},
         \end{equation*}
         onde \(\epsilon = e E a_0,\) \(\Delta_1 = E(s_{\frac12}) - E(p_{\frac12}),\) e \(\Delta_2 = E(p_{\frac32}) - E(p_{\frac12}).\)
      \item \(\mathcal{H}\) pode ser diagonalizado exatamente, mas é mais instruturivo encontrar as energias em dois limites: quando as perturbações elétrica são pequenas ou grandes em comparação com a separação de estrutura fina. Considere o primeiro caso e mostre que os desvios dos níveis são quadráticos em \(\epsilon\) quando \(\epsilon \ll \Delta_1,\) mas lineares em \(\epsilon\) para \(\epsilon \gg \Delta_1,\) sempre com \(\epsilon \ll \Delta_2.\) Mostre que o desvio quadrático do estado \(p_{\frac32}\) também pode ser encontrado facilmente usando teoria de perturbação.
      \item Finalmente, considere o limite de campo forte \(\epsilon \gg \Delta_2.\) Aqui precisamos tomar cuidado para manter o zero de energia correto. Para garantir isso, note que \(\Tr\mathcal{H}\) é sempre o mesmo qualquer seja o valor de \(\epsilon,\) incluindo \(\epsilon = 0.\) Usando esse fato e escrevendo os autovalores para campos elevados como \(\delta E + \frac{\Delta_1 + \Delta_2}{3},\) mostre que \(\delta E = 0, \pm 3\epsilon.\)
   \end{enumerate}
\end{exercício}
\begin{proof}[Resolução]

\end{proof}
