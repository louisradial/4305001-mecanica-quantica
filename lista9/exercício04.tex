% vim: spl=pt
\begin{exercício}{Efeito Zeeman quadrático no estado fundamental do átomo de hidrogênio}{ex4}
   Calcule o efeito Zeeman quadrático para o estado fundamental do átomo de hidrogênio devido ao termo diamagnético, \(\frac{e^2\vetor{A}^2}{2m_e c^2}\), usualmente ignorado no Hamiltoniano de interação. Escreva o desvio da energia como \(\Delta = -\frac12 \chi \vetor{B}^2\) e obtenha uma expressão para a suscetibilidade magnética \(\chi.\)
\end{exercício}
\begin{proof}[Resolução]
    
\end{proof}
