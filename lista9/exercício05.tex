% vim: spl=pt
\begin{exercício}{Dipolo elétrico induzido de um átomo em seu estado fundamental}{ex5}
   Considere um átomo arbitrário com estado fundamental, \(\ket{0},\) não degenerado com \(J = 0\) colocado em um campo elétrico uniforme \(\vetor{E}.\) O momento de dipolo elétrico induzido \(D\) pode ser encontrado de duas formas: 
   \begin{enumerate}[label=(\alph*)]
       \item a partir da energia do estado fundamental como função de \(E\); e
       \item por um cálculo direto do momento de dipolo no estado no estado fundamental perturbado.
   \end{enumerate}
   Mostre que ambos os métodos leval a mesma expressão para \(D,\)
   \begin{equation*}
      D = 2e^2 E \sum_{n \neq 0} \frac{\abs{\bra{0}X\ket{n}}^2}{E_n - E_0},
   \end{equation*}
   onde \(X = \sum_i x_i\) e \(E_n\) são as energias dos estados estacionários não perturbados \(\ket{n}.\)
\end{exercício}
\begin{proof}[Resolução]
    
\end{proof}
