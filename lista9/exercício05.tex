% vim: spl=pt
\begin{exercício}{Dipolo elétrico induzido de um átomo em seu estado fundamental}{ex5}
   Considere um átomo arbitrário com estado fundamental, \(\ket{0},\) não degenerado com \(J = 0\) colocado em um campo elétrico uniforme \(\vetor{E}.\) O momento de dipolo elétrico induzido \(D\) pode ser encontrado de duas formas: 
   \begin{enumerate}[label=(\alph*)]
       \item a partir da energia do estado fundamental como função de \(E\); e
       \item por um cálculo direto do momento de dipolo no estado no estado fundamental perturbado.
   \end{enumerate}
   Mostre que ambos os métodos leval a mesma expressão para \(D,\)
   \begin{equation*}
      D = 2e^2 E \sum_{n \neq 0} \frac{\abs{\bra{0}X\ket{n}}^2}{E_n - E_0},
   \end{equation*}
   onde \(X = \sum_i x_i\) e \(E_n\) são as energias dos estados estacionários não perturbados \(\ket{n}.\)
\end{exercício}
\begin{proof}[Resolução]
     Escrevendo o termo de interação como \(W = q_e X \epsilon,\) onde \(-q_e X\) é o momento de dipolo elétrico na direção do campo, vemos que a correção em primeira ordem em \(\epsilon\) se anula no estado fundamental, já que os elementos de matriz de \(X\) se anulam entre estados de mesma paridade. Assim, a energia corrigida para o estado fundamental é
     \begin{equation*}
        \Delta E = \sum_{n \neq 0} \frac{\abs{\bra{n}q_e\epsilon X\ket{0}}^2}{E_0 - E_n} = q_e^2 \epsilon^2 \sum_{n \neq 0} \frac{\abs{\bra{0}X\ket{n}}^2}{E_0 - E_n}
     \end{equation*}
     logo o dipolo induzido é
     \begin{equation*}
        D = - \diffp{\Delta E}{\epsilon} = 2 q_e^2 \epsilon \sum_{n \neq 0} \frac{\abs{\bra{0}X\ket{n}}^2}{E_n - E_0}.
     \end{equation*}

     Podemos obter esse resultado determinando o estado fundamental perturbado em primeira ordem. Tomando a condição de normalização \(\braket{0}{0}_\epsilon = 1,\) temos
     \begin{equation*}
        \ket{0}_\epsilon = \ket{0} + \sum_{n \neq 0} \frac{1}{E_0 - E_n}\ketbra{n}{n}q_e \epsilon X\ket{0} + O(\epsilon^2),
     \end{equation*}
     com
     \begin{align*}
        _\epsilon\braket{0}{0}_\epsilon = {}_\epsilon\braket{0}{0} + \sum_{n \neq 0} \frac{1}{E_0 - E_n} {}_\epsilon\braket{0}{n}\bra{n}q_e \epsilon X\ket{0}
                                        = 1 + \sum_{n \neq 0} \frac{q_e^2 \epsilon^2\abs{\bra{n}X\ket{0}}^2}{(E_0 - E_n)^2}
                                        = 1 + O(\epsilon^2),
     \end{align*}
     isto é, normalizado em primeira ordem. Neste estado, temos
     \begin{align*}
        _\epsilon\bra{0}X\ket{0}_\epsilon &= \bra{0}X\ket{0}_\epsilon + \sum_{n\neq0} \frac{1}{E_0 - E_n} \bra{0} q_e \epsilon X \ket{n} \bra{n} X \ket{0}_\epsilon\\
                                          &= \bra{0} X \ket{0} + 2q_e \epsilon \sum_{n \neq 0} \frac{\abs{\bra{0}X\ket{n}}^2}{E_0 - E_n} + q_e^2 \epsilon^2\sum_{n \neq 0}\sum_{m \neq 0} \frac{\bra{0}X\ketbra{n}{n} X \ketbra{m}{m}X\ket{0}}{(E_0 - E_n)(E_0 - E_m)}\\
                                          &= -2q_e \epsilon \sum_{n \neq 0}\frac{\abs{\bra{0}X\ket{n}}^2}{E_n - E_0} + O(\epsilon^2)
     \end{align*}
     portanto
     \begin{equation*}
        D = -q_e {}_\epsilon\bra{0}X\ket{0}_{\epsilon} = 2 q_e^2 \epsilon \sum_{n \neq 0} \frac{\abs{\bra{0}X\ket{n}}^2}{E_n - E_0}
     \end{equation*}
     é o momento de dipolo induzido.
\end{proof}
