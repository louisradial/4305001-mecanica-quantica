% vim: spl=pt
\begin{exercício}{Átomo de hélio}{ex6}
   Considere o átomo de hélio com \(Z = 2\) e dois elétrons. O Hamiltoniano que descreve o sistema é
   \begin{equation*}
      H = \frac{\vetor{p}_1^2}{2m} - \frac{Ze^2}{r_1} + \frac{\vetor{p}_2^2}{2m} - \frac{Z e^2}{r_2} + \frac{e^2}{r_{12}},
   \end{equation*}
   onde \(r_1,\) \(r_2\) são as distâncias entre o núcleo e os dois elétrons e \(r_{12}\) é a distância entre os dois elétrons.
   \begin{enumerate}[label=(\alph*)]
       \item Trate a interação entre os elétrons como uma perturbação. Encontre uma estimativa para o estado fundamental do sistema.
       \item Admita que a função de onda do estado fundamental possa ser representada pela função de prova
          \begin{equation*}
             \psi(\vetor{r}_1, \vetor{r}_2) = \frac{\zeta^3}{\pi} e^{- \zeta(\norm{\vetor{r}_1} + \norm{\vetor{r}_2})}
          \end{equation*}
          e estime a energia do estado fundamental usando o método variacional, sendo \(\zeta\) o parâmetro variacional.
       \item Compare os valores obtidos com o valor experimental \SI{-78.6}{\eV}.
       \item Estime o valor dos estados fundamentais de \(\ce{Li^+}\) e \(\ce{Be^++}\) cujos valores experimentais são, respectivamente, \SI{-197.1}{\eV} e \(\SI{-370}{\eV}.\)
   \end{enumerate}
\end{exercício}
\begin{proof}[Resolução]
   Considerando \(W = \frac{e^2}{r_{12}}\) como perturbação, o espectro do Hamiltoniano não perturbado é a soma de espectros de problema de Kepler, então o estado fundamental não perturbado é dado por
   \begin{equation*}
      \ket{E_{11}^{(0)}} = \frac1{\sqrt{2}} \ket{1s}_1\ket{1s}_2 \left(\ket{+}_1 \ket{-}_2 - \ket{-}_1\ket{+}_2\right) \quad\text{com}\quad
      \braket{\vetor{r}}{1s} = \sqrt{\frac{Z^3}{\pi a_0^3}} \exp\left(-\frac{Z \norm{\vetor{r}}}{a_0}\right),
   \end{equation*}
   com \(Z = 2,\) e tem energia dada por \(E_{11}^{(0)} = - 2Z^2\mathrm{Ry} = \SI{-108.85}{\eV}.\) Vamos estimar a correção com teoria de perturbação determinando o valor esperado de \(W\) neste estado,
   \begin{equation*}
      \Delta = \bra{E_{11}^{(0)}}W\ket{E_{11}^{(0)}} = \bra{1s}_1\bra{1s}_2\frac{e^2}{r_{12}}\ket{1s}_1 \ket{1s}_2 = \frac{Z^6e^2}{\pi^2 a_0^6}\int_{\mathbb{R}^3} \dln3{r_1}\int_{\mathbb{R}^3}\dln3{r_2} \frac{\exp\left[-\frac{2Z}{a_0}\left(\norm{\vetor{r}_1} + \norm{\vetor{r}_2}\right)\right]}{\norm{\vetor{r}_1 - \vetor{r}_2}}.
   \end{equation*}
   Para estimar essa integral, vamos truncar a expansão multipolar \(\frac{1}{\norm{\vetor{r}_1 - \vetor{r}_2}} \simeq \frac{1}{r_>}\) e então as integrais angulares se resolvem em \((4\pi)^2\). Com essa aproximação, obtemos
   \begin{align*}
      \Delta &\simeq \frac{4^2 Z^6e^2}{a_0^6} \int_{0}^{\infty} \dli{r_1} r_1^2\exp\left(-\frac{2Z r_1}{a_0}\right) \left[\int_{r_1}^{\infty} \dli{r_2} r_2 \exp\left(-\frac{2Zr_2}{a_0}\right) + \int_{0}^{r_1} \dli{r_2} \frac{r^2_2}{r_1} \exp\left(-\frac{2Z r_2}{a_0}\right)\right]\\
             &= \frac{Ze^2}{2a_0} \left[\int_0^\infty \dli{\rho_1} \rho_1^2 e^{-\rho_1} \int_{\rho_1}^\infty \dli{\rho_2} \rho_2 e^{-\rho_2} + \int_0^\infty \dli{\rho_1} \rho_1 e^{-\rho_1} \int_0^{\rho_1} \dli{\rho_2} \rho_2^2 e^{-\rho_2}\right]\\
             &= \frac{Ze^2}{2 a_0}\left[\int_0^\infty \dli{\rho_1} \rho_1^2\left(\rho_1 + 1\right) e^{-2\rho_1} - \int_0^\infty\dli{\rho_1} \rho_1\left(\rho_1^2 + 2\rho_1 + 2\right)e^{-2\rho_1} + 2\int_0^{\infty}\dli{\rho_1} \rho_1 e^{-\rho_1}\right]\\
             &= \frac{Ze^2}{2a_0}\left\{2-\int_0^\infty \dli{\rho} \left[\frac14(2\rho)^2 + 2\rho\right]e^{-2\rho}\right\}\\
             &= \frac{5Ze^2}{8a_0}\\
             &= \frac{5Z}{8} \mathrm{Ry}
   \end{align*}
   isto é, a energia do estado fundamental do hélio pode ser estimado por \(E_{11} \simeq \SI{-6.75}{Ry} = \SI{-91.84}{\eV}.\)
\end{proof}
