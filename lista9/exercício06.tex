% vim: spl=pt
\begin{exercício}{Átomo de hélio}{ex6}
   Considere o átomo de hélio com \(Z = 2\) e dois elétrons. O Hamiltoniano que descreve o sistema é
   \begin{equation*}
      H = \frac{\vetor{p}_1^2}{2m} - \frac{Ze^2}{r_1} + \frac{\vetor{p}_2^2}{2m} - \frac{Z e^2}{r_2} + \frac{e^2}{r_{12}},
   \end{equation*}
   onde \(r_1,\) \(r_2\) são as distâncias entre o núcleo e os dois elétrons e \(r_{12}\) é a distância entre os dois elétrons.
   \begin{enumerate}[label=(\alph*)]
       \item Trate a interação entre os elétrons como uma perturbação. Encontre uma estimativa para o estado fundamental do sistema.
       \item Admita que a função de onda do estado fundamental possa ser representada pela função de prova
          \begin{equation*}
             \psi(\vetor{r}_1, \vetor{r}_2) = \frac{\zeta^3}{\pi} e^{- \zeta(\vetor{r}_1 + \vetor{r}_2)}
          \end{equation*}
          e estime a energia do estado fundamental usando o método variacional, sendo \(\zeta\) o parâmetro variacional.
       \item Compare os valores obtidos com o valor experimental \SI{-78.6}{\eV}.
       \item Estime o valor dos estados fundamentais de \(\ce{Li^+}\) e \(\ce{Be^++}\) cujos valores experimentais são, respectivamente, \SI{-197.1}{\eV} e \(\SI{-370}{\eV}.\)
   \end{enumerate}
\end{exercício}
\begin{proof}[Resolução]
    
\end{proof}
