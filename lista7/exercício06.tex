% vim: spl=pt
\begin{exercício}{Interação entre partículas de spin \(\frac12\) em campo magnético uniforme}{ex6}
   O Hamiltoniano da interação entre duas partículas de spin \(\frac12\) em um campo magnético uniforme \(\vetor{B} = B \vetor{e}_z\) é
   \begin{equation*}
      H = B (a_1 \sigma_z^{(1)} + a_2 \sigma_z^{(2)}) + K \vetor{\sigma}^{(1)} \cdot \vetor{\sigma}^{(2)},
   \end{equation*}
   onde \(a_1\) e \(a_2\) são o negativo do dos valores dos momentos magnéticos das partículas e \(K\) é a intensidade da interação entre os spins.
   \begin{enumerate}[label=(\alph*)]
       \item Calcule os autovalores da energia desse sistema em segunda ordem de teoria de perturbação assumindo que \(B\) é pequeno, ou seja, que o acoplamento com o campo magnético é a perturbação.
       \item Calcule os autovalores da energia desse sistema em segunda ordem de teoria de perturbação assumindo que \(K\) é pequeno, ou seja, que a interação entre os spins é a perturbação.
       \item Encontre os valores exatos das energias de \(H\) e compare esse resultado com os resultados que você obteve nos itens anteriores.
   \end{enumerate}
\end{exercício}
\begin{proof}[Resolução]
    
\end{proof}
