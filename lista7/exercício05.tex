% vim: spl=pt
\begin{exercício}{Teoria de perturbação para um sistema de três níveis}{ex5}
    Um sistema físico que tem três estados não perturbados pode ser representado pelo Hamiltoniano perturbado
    \begin{equation*}
        H \doteq \begin{pmatrix}
           E_1 & 0 & a\\
           0 & E_1 & b\\
           \conj{a} & \conj{b} & E_2
        \end{pmatrix},
    \end{equation*}
    onde \(E_2 > E_1\). As quantidades \(a\) e \(b\) devem ser vistas como perturubação do sistema que tem a mesma ordem e são muito menores que \(E_2 - E_1.\) Calcule os novos autovalores da energia usando teoria de perturbação para níveis não degenerados. Diagonalize a matriz \(H\) e encontre os autovalores exatos. Calcule novamente os autovalores de energia usando teoria de perturbação para níveis degenerados. Compare os três resultados.
\end{exercício}
\begin{proof}[Resolução]
    
\end{proof}
