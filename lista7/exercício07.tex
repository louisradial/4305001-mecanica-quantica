% vim: spl=pt
\begin{exercício}{}{ex7}
    Considere uma partícula de massa \(\mu\) sujeita a se movimentar no plano \(xy\) sobre um círculo centrado na origem e de raio fixo \(\rho\). A úinica variável do sistema é o ângulo \(\alpha\) entre o eixo \(x\) e a posição da partícula no círculo. O estado quântico da partícula é definido pela função de onda \(\psi(\alpha),\) com \(\psi(\alpha + 2\pi) = \psi(\alpha)\) e
    \begin{equation*}
       \int_0^{2\pi}{\dli\alpha \abs{\psi(\alpha)}^2} = 1.
    \end{equation*}
    \begin{enumerate}[label=(\alph*)]
       \item Considere o operador \(M = \frac{\hbar}{i} \diff{}{\alpha}\). Calcule os autovalores e as autofunções normalizadas desse operador. Qual o significado físico de \(M\)?
       \item A energia cinética da partícula é dada pelo Hamiltoniano \(H_0 = \frac{M^2}{2 \mu \rho^2}.\) Calcule os autovalores e as autofunções de \(H_0.\) Os níveis de energia encontrados são degenerados?
       \item No instante \(t = 0,\) a partícula tem como função de onda \(N \cos^2 \alpha,\) onde \(N\) é uma constante de normalização. Discuta a localização da partícula sobre o círculo em um instante posterior \(t.\)
       \item Suponha que a partícula tenha uma carga \(q\) e que ela interaja com um campo elétrico uniforme fraco \(\vetor{E} = E\vetor{e}_x.\) Nesse caso precisamos considerar também a contribuição perturbativa \(W = -q E \rho \cos\alpha.\) Calcule a nova função de onda do nível fundamental em primeira ordem em \(E.\) Determine o coeficiente de sucetibilidade linear \(\chi\) entre o dipolo elétrico paralelo a \(\vetor{e}_x\) adquirido pela partícula e o campo \(\vetor{E}.\)
       \item Considere a molécula de etano \ce{CH3-CH3}, estamos interessados na rotação de um agrupamento \ce{CH3} com relação ao outro, ao redor da reta que une os dois átomos de carbono. Em primeira aproximação, essa rotação pode se fazer livremente, e o Hamiltoniano \(H_0\) acima descreve a energia cinética de rotação de um dos agrupamentos \ce{CH3} com relação ao outro (é necessário nesse caso substituir \(2 \mu \rho^2\) por \(\lambda I,\) onde \(I\) é o momento de inércia do agrupamento \ce{CH3} com relação ao eixo de rotação e \(\lambda\) uma constante). Para levar em conta a energia de interação eletrostática entre os dois agrupamentos \ce{CH3}, somamos a \(H_0\) um termo sob a forma \(W = b \cos(3\alpha),\) onde \(b\) é uma constante real. Justifique fisicamente a dependência em \(\alpha\) em \(W\). Calcule a energia e a função de onda do nível fundamental, até a primeira ordem em \(b\) para a função de onda e até a segunda ordem para a energia. Interprete o resultado fisicamente.
    \end{enumerate}
\end{exercício}
\begin{proof}[Resolução]
    
\end{proof}
