% vim: spl=pt
\begin{exercício}{Autofunções de um potencial tridimensional com momento angular nulo}{ex2}
    Obtenha os autovalores e autofunções para a onda \(s\) no caso do potencial tridimensional ser dado por
    \begin{equation*}
        V(r) = -\frac{a^2}{8} e^{-\frac{r}{r_0}}.
    \end{equation*}
\end{exercício}
\begin{proof}[Resolução]
    Para o potencial dado, a equação radial com \(\ell = 0\) é dada por
    \begin{equation*}
        -\frac{\hbar^2}{2\mu} \diff[2]{u}{r} - \frac{a^2}{8} e^{-\frac{r}{r_0}}u(r) = E u(r),
    \end{equation*}
    onde \(\braket{\vetor{r}}{n00} = \frac{u(r)}{r} Y_{00}.\) Definindo os parâmetros adimensionais \(\alpha = \sqrt{\frac{2\mu r_0^2 a^2}{8 \hbar^2}}\) e \(\beta = \sqrt{-\frac{2\mu E r_0^2}{\hbar^2}},\) assim como a mudança de variáveis \(r = \rho r_0\) e \(w(\rho) = u(r),\) temos
    \begin{equation*}
        \diff[2]{w}{\rho} + \alpha^2 e^{-\rho} w(\rho) = \beta^2 w(\rho).
    \end{equation*}
    Introduzimos agora a mudança de variáveis \(x = 2 \alpha e^{-\frac{\rho}{2}}\) e \(h(x) = w(\rho),\) então de
    \begin{equation*}
        \diff[2]{w}{\rho} = \diff{\rho}{x}\diff*{\left(\diff{f}{x} \diff{\rho}{x}\right)}{x} = \frac{x}{2} \diff*{\left(\diff{f}{x} \frac{x}{2}\right)}{x} = \frac{x^2}{4}\diff[2]{f}{x} + \frac{x}{4} \diff{f}{x},
    \end{equation*}
    obtemos
    \begin{equation*}
        x^2 \diff{f}{x} + x \diff{f}{x} + x^2 f(x) = 4\beta^2 f(x),
    \end{equation*}
    que é uma equação de Bessel de ordem \(2 \beta,\) cuja solução geral é dada por
    \begin{equation*}
        f(x) = c_1 J_{2\beta}(x) + c_2 N_{2\beta}(x),
    \end{equation*}
    onde \(c_1, c_2 \in \mathbb{C}\) são constantes. Notemos que
    \begin{equation*}
        \lim_{r\to\infty}{\frac{w(\rho)}{\rho}} = -\lim_{x \to 0}{\frac{f(x)}{\ln\left(\frac{x}{2\alpha}\right)}} = 
        % -\lim_{x\to0}{\frac{f(x)}{\sqrt{x} \ln\left(\frac{x}{2\alpha}\right)}}
        -c_1 \lim_{x\to0}{\frac{J_{2 \beta}(x)}{\ln\left(\frac{x}{2\alpha}\right)}} - c_2 \lim_{x\to0}{\frac{N_{2 \beta}(x)}{\ln\left(\frac{x}{2\alpha}\right)}}
        = -c_2\lim_{x\to0}{\frac{N_{2 \beta}(x)}{\ln\left(\frac{x}{2\alpha}\right)}},
    \end{equation*}
    já que as funções de Bessel \(J_{2\beta}\) são polinomiais na origem, portanto para que a função de onda seja normalizável, devemos ter \(c_2 = 0\) para que \(\frac{u(r)}{r} \to 0\) conforme \(r \to \infty.\) Ainda, a função de onda deve ser regular na origem, portanto \(u(0) = 0,\) isto é,
    \begin{equation*}
        f(2\alpha) = 0 \implies J_{2\beta}(2 \alpha) = 0 \implies 2\alpha = \xi^m_{2\beta}, \quad m \in \mathbb{N}
    \end{equation*}
    onde \(\xi^m_{2 \beta}\) é o \(m\)-ésimo zero da função de Bessel \(J_{2\beta}\). Esta condição determina os níveis de energia para os estados ligados com \(\ell = 0,\) e enumerando-os de forma crescente, temos
    \begin{equation*}
        \braket{\vetor{r}}{n00} = C\frac{J_{2\beta}\left(\frac{r_0 a}{\hbar}\sqrt{e^{-\frac{r}{r_0}}\mu}\right)}{r}Y_{00},
    \end{equation*}
    onde a energia \(E_n = -\frac{\hbar^2 \beta_n^2}{2\mu r_0^2}\) é tal que \(\xi^m_{2 \beta_n} = 2\frac{r_0 a \sqrt{\mu}}{\hbar}\) para algum \(m \in \mathbb{N},\) e \(C\) é a constante de normalização.
\end{proof}
