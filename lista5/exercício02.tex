% vim: spl=pt
\begin{exercício}{Ângulos de Euler}{ex2}
    Considere a sequência de rotações de Euler representadas por
    \begin{equation*}
        D^{\frac12}(\alpha \beta \gamma) = e^{-iS_3 \alpha} e^{-i S_2 \beta} e^{-i S_3 \gamma},
    \end{equation*}
    onde \(S_{i} = \frac12 \sigma_i.\) Mostre que devido a propriedades do grupo de rotações essa sequência de operações é equivalente a uma rotação de ângulo \(\theta\) em torno de um único eixo. Determine \(\theta\).
\end{exercício}
\begin{proof}[Resolução]
    Denotamos a rotação por um ângulo \(\vartheta\) em torno de um eixo \(\vetor{n}\) por \(R_{\vetor{n}}(\vartheta)\) e sua representação no espaço de Hilbert por \(D(\vartheta, \vetor{n}),\) onde suprimimos a notação de \(j = \frac12.\) Nesta notação, temos \(D(\alpha \beta \gamma) = D(\alpha, \vetor{e}_z) D(\beta, \vetor{e}_y) D(\gamma, \vetor{e}_z),\) portanto pela lei de composição de representações, devemos ter \(D(\alpha \beta \gamma)\) como a representação da rotação \(R_{\vetor{e}_z}(\alpha) R_{\vetor{e}_y}(\beta) R_{\vetor{e}_z}(\gamma).\)
\end{proof}
