% vim: spl=pt
\begin{exercício}{Ângulos de Euler}{ex2}
    Considere a sequência de rotações de Euler representadas por
    \begin{equation*}
        D^{\frac12}(\alpha \beta \gamma) = e^{-iS_3 \alpha} e^{-i S_2 \beta} e^{-i S_3 \gamma},
    \end{equation*}
    onde \(S_{i} = \frac12 \sigma_i.\) Mostre que devido a propriedades do grupo de rotações essa sequência de operações é equivalente a uma rotação de ângulo \(\theta\) em torno de um único eixo. Determine \(\theta\).
\end{exercício}
\begin{proof}[Resolução]
    A representação de uma rotação em torno de um eixo \(\vetor{n}\) por um ângulo \(\theta\) é
    \begin{equation*}
        D(\theta, \vetor{n}) = \cos\frac\theta2 \unity - i\sin\frac\theta2 \vetor{n}\cdot \vetor{\sigma},
    \end{equation*}
    portanto \(\Tr\left[D(\theta, \vetor{n})\right] = 2 \cos\frac\theta2.\) Utilizando a rotação em termos dos ângulos de Euler, temos
    \begin{align*}
        \Tr\left[D(\theta, \vetor{n})\right] &= \Tr\left[D(\alpha, \vetor{e}_z) D(\beta, \vetor{e}_y) D(\gamma, \vetor{e}_z)\right]\\
                                             &= \Tr\left[D(\gamma, \vetor{e}_z) D(\alpha, \vetor{e}_z) D(\beta, \vetor{e}_y)\right]\\
                                             &= \Tr\left[D(\alpha + \gamma, \vetor{e}_z)D(\beta, \vetor{e}_y)\right],
    \end{align*}
    onde usamos que as rotações em relação a um mesmo eixo formam um subgrupo abeliano e que a representação é um homomorfismo. Juntando esses resultados, temos
    \begin{align*}
        2\cos\frac\theta2 &= \Tr\left\{\left[\cos\frac{\alpha + \gamma}{2}\unity - i \sin\frac{\alpha + \gamma}2 \sigma_z\right]\left[\cos\frac\beta2 \unity - i \sin\frac\beta2 \sigma_y\right]\right\}\\
                          &= \Tr\left[\cos\frac{\alpha + \gamma}{2} \cos\frac\beta2 \unity\right]\\
                          &= 2 \cos\frac{\alpha + \gamma}2 \cos\frac{\beta}{2},
    \end{align*}
    isto é, \(\theta = 2\arccos\left[\cos\frac{\alpha + \gamma}2 \cos\frac{\beta}{2}\right]\)
\end{proof}
