% vim: spl=pt
\begin{exercício}{Adição de momento angular}{ex6}
    Consideere os autoestados de momento angular total \(\vetor{J} = \vetor{J}_1 + \vetor{J}_2 + \vetor{J}_3,\) de três partículas de spin 1. Seja \(j(j+1)\) os autovalores de \(\vetor{J}^2.\)
    \begin{enumerate}[label=(\alph*)]
        \item Quais os valores possíveis de \(j\)? Quantos estados linearmente independentes existem para cada um desses valores?
        \item Construa explicitamente o estado \(j = 0.\) Se \(\vetor{a}\), \(\vetor{b}\) e \(\vetor{c}\) são vetores ordinários, o único escalar linear nos três vetores que podemos formar é \((\vetor{a} \times \vetor{b}) \cdot \vetor{c}.\) Encontre uma relação entre esse fato e o seu resultado para \(j = 0.\)
    \end{enumerate}
\end{exercício}
\begin{proof}[Resolução]
    
\end{proof}
