% vim: spl=pt
\begin{exercício}{Matriz densidade para spin 1}{ex4}
    Mostre que a matriz densidade \(\rho\) para um sistema físico de spin 1 é
    \begin{equation*}
        \rho = \frac13 \left(1 + P_iJ^i  + w_{ij} T^{ij}\right),
    \end{equation*}
    onde \(J^i\) são os operadores de momento angular e \(T^{ij} = \frac12\anticommutator{J^i}{J^j} - \frac23 \delta^{ij}\), com \(P_i\) e \(w_{ij}\) constantes reais. Encontre expressões explícitas para \(\vetor{P}\) e \(w_{ij}\). (análogas a \(\vetor{P} = \mean{\vetor{\sigma}}\) para \(j = \frac12\))
\end{exercício}
\begin{proof}[Resolução]
    Seja \(\mathscr{H}_1\) o espaço de Hilbert associado ao sistema de spin 1. O espaço de operadores sobre esse espaço pode ser isomorficamente identificado como \(\mathscr{H}_1 \otimes \mathscr{H}_1,\) pois um operador arbitrário neste espaço pode ser escrito como uma combinação linear de operadores dados por \(\ket{1m}\bra{1m'} \in \mathscr{H}_1 \otimes \mathscr{H}_1^\dag \cong \mathscr{H}_1 \otimes \mathscr{H}_1.\) Assim, pela decomposição \(\mathscr{H}_1 \otimes \mathscr{H}_1 = \mathscr{H}_0 \oplus \mathscr{H}_1 \oplus \mathscr{H}_2,\) concluímos que \(\set{\unity, J^i, T^{ij}}\) é uma base para \(\mathscr{H}_1 \otimes \mathscr{H}_1.\)

    % Notemos que
    % \begin{equation*}
    %     J^{i} J^{j} = \frac12 \left(\anticommutator{J^i}{J^j} + \commutator{J^i}{J^j}\right) = \frac12i\epsilon\indices{^{ij}_k} J^k + T^{ij} + \frac{2}{3} \delta^{ij},
    % \end{equation*}
    % e
    % \begin{align*}
    %     J^i J^j J^k &= \frac12i\epsilon\indices{^{ij}_\ell} J^\ell J^k  + T^{ij}J^k + \frac23 \delta^{ij} J^k\\
    %                 &= \frac12 \epsilon\indices{^{ij}_\ell}\left(\frac12 \epsilon\indices{^{\ell k}_n} J^n + T^{\ell k} + \frac23 \delta^{\ell k}\right) + T^{ij} J^k + \frac23 \delta^{ij} J^k\\
    %                 &= \frac14 \underbrace{\epsilon\indices{^{ij}_{\ell}} \epsilon\indices{^{\ell k}_n}}_{\delta^{ik} \delta\indices{^j_n} - \delta^{jk}\delta\indices{^i_n}} J^n + \frac12 \epsilon\indices{^{ij}_{\ell}} T^{\ell k} + \frac13 \epsilon\indices{^{ij}_{\ell}} \delta^{\ell k} + T^{ij} J^k + \frac23 \delta^{ij} J^k\\
    %                 &= \frac14 \delta^{ik} J^j - \frac14 \delta^{jk} J^i + \frac12 \epsilon\indices{^{ij}_\ell} T^{\ell k} + \frac13 \epsilon^{ijk} + T^{ij}J^k + \frac23 \delta^{ij} J^k.
    % \end{align*}

    Um operador estatístico \(\rho\) que descreve um estado de mistura neste sistema é dado pela combinação linear
    \begin{equation*}
        \rho = \alpha\unity + \beta_i J^i + \gamma_{ij} T^{ij}.
    \end{equation*}
    Como \(T^{ij}\) tem traço nulo por construção e \(J^i\) tem traço nulo, devemos tomar \(\alpha = \frac13\) para que \(\Tr\rho = 1.\) Assim, definindo \(\beta_i = \frac13 P_i\) e \(\gamma_{ij} = \frac13 w_{ij},\) obtemos
    \begin{equation*}
        \rho = \frac13 \left(\unity + P_i J^i + w_{ij} T^{ij}\right).
    \end{equation*}
    Como \(T^{ij}\) é simétrico e de traço nulo, podemos tomar \(w_{ij}\) como simétrico e de traço nulo. De fato, expandimos \(w_{ij}\) em sua parte de traço, \(\frac13 w\indices{^\ell_{\ell}}\delta_{ij},\) sua parte antissimétrica, \(w_{[ij]},\) e sua parte simétrica, \(w_{(ij)}\), então
    \begin{equation*}
        w_{ij} T^{ij} = \left(\frac13 w\indices{^\ell_{\ell}} \delta_{ij} + w_{[ij]}+ w_{(ij)}\right)T^{ij} = \frac13 w\indices{^\ell_\ell} T\indices{^k_k} + w_{[ij]} T^{(ij)} + w_{(ij)}T^{(ij)} = \frac13 w\indices{^\ell_\ell} T\indices{^k_k} + w_{(ij)} T^{ij},
    \end{equation*}
    isto é, a parte antissimétrica não contribui para o operador estatístico. Ainda, notemos que
    \begin{equation*}
        T\indices{^k_k} = \delta_{ij} T^{ij} = \frac12 \delta_{ij} \anticommutator{J^i}{J^j} - \frac23 \delta_{ij} \delta^{ij} = (J^1)^2 + (J^2)^2 + (J^3)^2 - 2 = \vetor{J}^2 - 2 = 0,
    \end{equation*}
    portanto a parte de traço \(w\indices{^\ell_\ell}\) não contribui para o operador estatístico.

    Notemos que
    \begin{align*}
        \mean{\vetor{J}}_\rho &= \Tr( \rho \vetor{J})\\
                              &= \frac13 \Tr(\vetor{J} + P_i J^i \vetor{J} + w_{ij} T^{ij} \vetor{J}) = \frac13 P_i \Tr(J^i J^k) \vetor{e}_k + \frac16 w_{ij} \Tr(\anticommutator{J^i}{J^j}J^k) \vetor{e}_k\\
                              &= \frac13 P_i \Tr\left(\frac12i\epsilon\indices{^{ik}_\ell}J^\ell + T^{ik} + \frac23 \delta^{ik}\right)\vetor{e}_k + \frac16 w_{ij} \Tr(\anticommutator{J^i}{J^j}J^k)\vetor{e}_k\\
                              &= \frac23 P_i \delta^{ik} \vetor{e}_k + \frac16 w_{ij} \left[\Tr(J^i J^j J^k) + \Tr(J^j J^i J^k)\right] \vetor{e}_k
    \end{align*}
    e que
    \begin{align*}
        \mean{T^{ij}}_{\rho} &= \Tr(\rho T^{ij})\\
                             &= \frac13 \Tr(T^{ij}) + \frac13 P_k \Tr(J^k T^{ij}) + \frac13 w_{\ell k} \Tr(T^{\ell k} T^{ij})\\
                             &= \frac13 P_k \left[\frac12 \Tr(J^k \anticommutator{J^i}{J^j}) - \frac23 \delta^{ij} \Tr(J^k)\right] + \frac13 w_{\ell k} \left[\frac12\Tr(T^{\ell k} \anticommutator{J^i}{J^j}) - \frac23 \delta^{ij} \Tr(T^{\ell k})\right]\\
                             &= \frac16 P_k \left[\Tr(J^k J^i J^j) + \Tr(J^k J^j J^i)\right] + \frac1{12} w_{\ell k} \Tr(\anticommutator{J^\ell}{J^k} \anticommutator{J^i}{J^j})
    \end{align*}
    Notemos que \(\Tr(J^i J^j J^k)\) e \(\Tr(J^i J^j J^k J^\ell)\) são lineares em relação a cada operador e que são invariantes por rotações,
    \begin{equation*}
        \Tr(J^i J^j J^k) \to \Tr(\herm{D} J^i J^j J^k D) = \Tr(J^i J^j J^k)\quad\text{e}\quad
        \Tr(J^i J^j J^k J^\ell) \to \Tr(\herm{D} J^i J^j J^k J^\ell D) = \Tr(J^i J^j J^k J^\ell)
    \end{equation*}
    portanto \(\Tr(J^i J^j J^k)\) deve ser proporcional ao tensor totalmente antissimétrico \(\epsilon^{ijk}\) e \(\Tr(J^i J^j J^k J^\ell)\) deve ser uma combinação linear dos produtos \(\delta^{ij} \delta^{k \ell},\) \(\delta^{i k} \delta^{\ell j}\) e \(\delta^{i \ell} \delta^{jk}\), isto é,
    \begin{equation*}
        \Tr(J^i J^j J^k) = \kappa \epsilon^{ijk}
        \quad\text{e}\quad
        \Tr(J^i J^j J^k J^\ell) = \lambda \delta^{ij} \delta^{k\ell} + \mu \delta^{i k} \delta^{j \ell} + \nu \delta^{i \ell} \delta^{jk},
    \end{equation*}
    com \(\kappa, \lambda, \mu \nu \in \mathbb{C}\). Com um cálculo explícito, vemos que 
    \begin{equation*}
        \Tr(J^1 J^2 J^3) = i,\quad\Tr(J^1 J^2 J^1 J^2) = 0,\quad \Tr(J^1 J^1 J^2 J^2) = 1\quad\text{e}\quad\Tr(J^1 J^2 J^2 J^1) = 1, 
    \end{equation*}
    portanto 
    \begin{equation*}
        \Tr(J^i J^j J^k) = i \epsilon^{ijk}
        \quad\text{e}\quad
        \Tr(J^i J^j J^k J^\ell) = \delta^{ij} \delta^{k \ell} + \delta^{ik} \delta^{j \ell}.
    \end{equation*}
    Com esses resultados, obtemos
    \begin{equation*}
        \mean{\vetor{J}}_{\rho} = \frac23 \vetor{P} \implies P_i = \frac32 \mean{J^i}_\rho,
    \end{equation*}
    pois \(w_{ij} \epsilon^{ijk} \vetor{e}_k = - w_{ij} \epsilon^{jik} = 0\) e
    \begin{align*}
        \mean{T^{ij}}_{\rho} &= \frac{i}{3} P_k \left[\epsilon^{kij} + \epsilon^{kji}\right] + \frac1{12} w_{\ell k} \left[\Tr(J^\ell J^k J^i J^j) + \Tr(J^\ell J^k J^j J^i) + \Tr(J^k J^\ell J^i J^j) + \Tr(J^k J^\ell J^j J^i)\right]\\
                             &= \frac{1}{12} w_{\ell k} \left[4\delta^{\ell k} \delta^{ij} + 2\delta^{\ell i} \delta^{kj} + 2\delta^{\ell j} \delta^{ki}\right]\\
                             &= \frac13 w\indices{^\ell_\ell}\delta^{ij} + \frac16 w^{ij} + \frac16 w^{ji}\\
                             &= \frac13 w^{ij},
    \end{align*}
    isto é, \(w^{ij} = 3\mean{T^{ij}}_{\rho},\) já que \(w\indices{^\ell_\ell} = 0,\) como discutido.
\end{proof}
