% vim: spl=pt
\begin{exercício}{Probabilidade de medir componentes de momento angular após rotação}{ex7}
    Considere o estado de momento angular orbital \(\ket{2 0}\). Imagine que esse estado é rodado de um ângulo \(\theta\) em torno do eixo \(y\). Encontre a probabilidade do novo estado ser encontrado com um dado valor de \(m\).
\end{exercício}
\begin{proof}[Resolução]
    Usando os elementos de matriz \(\bra{2 m'}J_y\ket{2m},\) análogo ao que foi feito no \cref{ex:ex3}, obtemos
    % \begin{equation*}
    %     \bra{m'}J_y\ket{m} = \frac{\sqrt{6 - m(m + 1)} \delta_{m', m+1} - \sqrt{6 - m(m-1)}\delta_{m', m-1}}{2i},
    % \end{equation*}
    % e em particular temos
    \begin{equation*}
        J_y \ket{22} = i \ket{21},\quad
        J_y \ket{20} = -i\frac{\sqrt{6}}{2} \ket{21} + i \frac{\sqrt{6}}{2} \ket{2{-1}},\quad\text{e}\quad
        J_y \ket{2{-2}} = -i \ket{2{-1}},
    \end{equation*}
    logo
    \begin{equation*}
        J_y \left(\frac{\sqrt{6}}{4}\ket{22} + \frac12 \ket{20} + \frac{\sqrt{6}}{4} \ket{2{-2}}\right) 
        = i\frac{\sqrt{6}}{4} \ket{21} + \frac12 \left(-i \frac{\sqrt{6}}{2} \ket{21} + i \frac{\sqrt{6}}{2} \ket{2 {-1}}\right) - i\frac{\sqrt{6}}{4}\ket{2{-1}}
        = 0,
    \end{equation*}
    isto é, o vetor unitário \(\ket{20_y} = \frac{\sqrt{6}}{4}\ket{22_z} + \frac12 \ket{20_z} + \frac{\sqrt{6}}{4} \ket{2{-2}_z},\) satisfaz \(J_y \ket{20_y} = 0.\) Para os autovetores de \(J_y,\) os operadores de levantamento e abaixamento são 
    \begin{equation*}
        J^{(y)}_{\pm} = J_z \pm i J_x = J_z \pm \frac{i}{2} \left(J_+ + J_-\right),
    \end{equation*}
    portanto os demais autoestados de \(J_y\) são
    \begin{align*}
        \ket{21_y} &= \frac{1}{a_+(20)}\left[J_z + \frac{i}{2} \left(J_+ + J_-\right)\right]\ket{20_y}\\
                   &= \frac{1}{\sqrt{6}}\left[J_z + \frac{i}{2} \left(J_+ + J_-\right)\right]\left[\frac{\sqrt{6}}{4}\ket{22_z} + \frac12 \ket{20_z} + \frac{\sqrt{6}}{4} \ket{2{-2}_z}\right]\\
                   &= \frac{1}{2}\ket{22_z} + \frac{i}{2}\left(\frac{1}{4} a_-(22) + \frac12 \right)\ket{21_z} + \frac{i}{2}\left(\frac12 + \frac{1}{4} a_+(22)\right)\ket{2{-1}_z} - \frac{1}{2}\ket{2{-2}_z}\\
                   &= \frac12 \ket{22_z} + \frac{i}2\ket{21_z} + \frac{i}{2} \ket{2 {-1}_z} - \frac12 \ket{2 {-2}_z}\\
        \ket{22_y} &= \frac{1}{a_+(21)}\left[J_z + \frac{i}{2} \left(J_+ + J_-\right)\right]\ket{21_y}\\
                   &= \frac{1}{2}\left[J_z + \frac{i}{2} \left(J_+ + J_-\right)\right]\left(\frac12 \ket{22_z} + \frac{i}2\ket{21_z} + \frac{i}{2} \ket{2 {-1}_z} - \frac12 \ket{2 {-2}_z}\right)\\
                   &= \frac14 \ket{22_z} + \frac{i}2 \ket{21_z} - \frac{\sqrt{6}}4\ket{20_z} -\frac{i}{2}\ket{2{-1}_z} + \frac14\ket{2{-2}_z}\\
        \ket{2{-1}_y}&= \frac{1}{a_-(20)}\left[J_z - \frac{i}{2} \left(J_+ + J_-\right)\right]\ket{20_y}\\
                   &= \frac{1}{\sqrt{6}}\left[J_z - \frac{i}{2} \left(J_+ + J_-\right)\right]\left[\frac{\sqrt{6}}{4}\ket{22_z} + \frac12 \ket{20_z} + \frac{\sqrt{6}}{4} \ket{2{-2}_z}\right]\\
                   % &= \frac{1}{2}\ket{22_z} - \frac{i}{2}\left(\frac{1}{4} a_-(22) + \frac12 \right)\ket{21_z} - \frac{i}{2}\left(\frac12 + \frac{1}{4} a_+(22)\right)\ket{2{-1}_z} - \frac{1}{2}\ket{2{-2}_z}\\
                   &= \frac12\ket{22_z} - \frac{i}{2} \ket{21_z} - \frac{i}{2}\ket{2{-1}_z} - \frac12 \ket{2{-2}_z}\\
                   \ket{2{-2}_y} &= \frac{1}{a_-(2{-1})}\left[J_z - \frac{i}{2} \left(J_+ + J_-\right)\right]\ket{21_y}\\
                   &= \frac{1}{2}\left[J_z - \frac{i}{2} \left(J_+ + J_-\right)\right]\left(\frac12\ket{22_z} - \frac{i}{2} \ket{21_z} - \frac{i}{2}\ket{2{-1}_z} - \frac12 \ket{2{-2}_z}\right)\\
                   &= \frac14 \ket{22_z} - \frac{i}2 \ket{21_z} - \frac{\sqrt{6}}4\ket{20_z} +\frac{i}{2}\ket{2{-1}_z} + \frac14\ket{2{-2}_z}.
    \end{align*}
    Com isso, podemos obter os elementos de matriz \(\bra{2m_z}D(\theta,\vetor{e}_y)\ket{20_z}\) com
    \begin{align*}
        \bra{2m_z}D(\theta,\vetor{e}_y) \ket{20_z} &= \sum_{m_y = -2}^{2} \bra{2m_z} e^{-i\theta J_y}\ket{2m_y} \braket{2m_y}{20_z}\\
                                                   &= \sum_{m_y = -2}^{2}  \braket{2m_z}{2m_y}e^{-i \theta m_y}\braket{2m_y}{20_z}\\
                                                   &= \sum_{m_y = -2}^{2} \braket{2m_z}{2m_y} e^{-i\theta m_y} \left(-\frac{\sqrt{6}}{4} \delta_{m_y, -2} + \frac12 \delta_{m_y,0}- \frac{\sqrt{6}}{4} \delta_{m_y, 2}\right)\\
                                                   &=- \frac{\sqrt{6}}{4}\braket{2m_z}{2{-2}_y} e^{2i\theta} + \frac12 \braket{2m_z}{20_y} - \frac{\sqrt{6}}{4} \braket{2m_z}{22_y} e^{-2i\theta},
    \end{align*}
    isto é,
    \begin{align*}
        \bra{22_z}D(\theta,\vetor{e}_y) \ket{20_z} &= -\frac{\sqrt{6}}{16} e^{2i \theta} + \frac{\sqrt{6}}{8} - \frac{\sqrt{6}}{16} e^{-2i\theta} = \frac{\sqrt{6}}{8} (1 - \cos2\theta)\\
        \bra{21_z}D(\theta,\vetor{e}_y) \ket{20_z} &= -\frac{\sqrt{6}}{8i} e^{2i\theta} + \frac{\sqrt{6}}{8i} e^{-2i\theta} = -\frac{\sqrt{6}}{4}\sin2\theta\\
        \bra{20_z}D(\theta,\vetor{e}_y) \ket{20_z} &= \frac{6}{16}e^{2i\theta} + \frac14 + \frac{6}{16} e^{-2i\theta} = \frac14 + \frac{3}{4}\cos2\theta\\
        \bra{2{-1}_z}D(\theta,\vetor{e}_y) \ket{20_z} &= \frac{\sqrt{6}}{8i}e^{2i\theta} - \frac{\sqrt{6}}{8i} e^{-2i\theta} = \frac{\sqrt{6}}{4} \sin2\theta\\
        \bra{2{-2}_z}D(\theta,\vetor{e}_y) \ket{20_z} &= - \frac{\sqrt{6}}{16} e^{2i\theta} + \frac{\sqrt{6}}{8} - \frac{\sqrt{6}}{16} e^{-2i\theta} = \frac{\sqrt{6}}{8} (1 - \cos2\theta).
    \end{align*}
    Assim, 
    \begin{equation*}
        P_2 = P_{-2} = \frac{3}{32} (1 - \cos 2\theta)^2,\quad
        P_1 = P_{-1} = \frac{3}{8} \sin^22\theta,\quad\text{e}\quad
        P_0 = \left(\frac14 + \frac34 \cos2\theta\right)^2
    \end{equation*}
    são as probabilidades \(P_m = \abs{\bra{2m_z} D(\theta, \vetor{e}_y) \ket{20_z}}^2\) do novo estado ser encontrado com um dado valor de \(m_z.\)
\end{proof}
