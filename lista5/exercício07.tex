% vim: spl=pt
\begin{exercício}{}{ex7}
    Considere o estado de momento angular orbital \(\ket{2 0}\). Imagine que esse estado é rodado de um ângulo \(\theta\) em torno do eixo \(y\). Encontre a probabilidade do novo estado ser encontrado com um dado valor de \(m\).
\end{exercício}
\begin{proof}[Resolução]
    Escrevamos \(D^2(\theta, \vetor{e}_y) \ket{20} = \exp(-i \theta J_y) \ket{20}.\) Como no \cref{ex:ex3}, temos
    \begin{equation*}
        \bra{m'}J_y\ket{m} = \frac{\sqrt{6 - m(m + 1)} \delta_{m', m+1} - \sqrt{6 - m(m-1)}\delta_{m', m-1}}{2i},
    \end{equation*}
    portanto
    \begin{equation*}
        J_y \doteq \frac{1}{2i}\begin{pmatrix}
             && 2 &&  &&  && \\
            -2 &&  && \sqrt{6} && &&\\
               && -\sqrt{6} &&  && \sqrt{6} &&\\
               &&  && -\sqrt{6} && && 2\\
               && && && -2 &&
        \end{pmatrix}
    \end{equation*}
\end{proof}
