% vim: spl=pt
\begin{exercício}{Rotações em três dimensões}{ex5}
    Mostre que
    \begin{equation*}
        e^{i\frac{\theta}{2} \vetor{\sigma} \cdot \vetor{n}} \vetor{\sigma} e^{-i\frac{\theta}{2} \vetor{\sigma} \cdot \vetor{n}} = \cos\theta \vetor{\sigma} + (1 - \cos\theta)(\vetor{\sigma} \cdot \vetor{n})\vetor{n} + \sin\theta (\vetor{n} \times \vetor{\sigma}),
    \end{equation*}
    onde \(\vetor{n} \in \mathbb{R}^3\) é um vetor unitário. Encontre um argumento para mostrar que essa relação é válida para qualquer operador vetorial \(\vetor{V}\) desde que o momento angular apropriado seja usado para gerar a rotação, isto é,
    \begin{equation*}
        e^{i\theta \vetor{J} \cdot \vetor{n}} \vetor{V} e^{-i\theta \vetor{J} \cdot \vetor{n}} = \cos\theta \vetor{V} + (1 - \cos\theta)(\vetor{V} \cdot \vetor{n})\vetor{n} + \sin\theta (\vetor{n} \times \vetor{V}).
    \end{equation*}
\end{exercício}
\begin{proof}[Resolução]
    
\end{proof}
