% vim: spl=pt
\begin{lemma}{Derivada da exponencial de um operador dependente de um parâmetro}{derivada}
    Seja \(A(\lambda)\) um operador dependente de um parâmetro contínuo, então valem as expressões
    \begin{equation*}
       \diff*{e^{\mu A(\lambda)}}{\lambda} = e^{\mu A(\lambda)}\left[\int_0^\mu \dli\tau e^{-\tau A(\lambda)} \diff{A(\lambda)}{\lambda} e^{\tau A(\lambda)}\right]
    \end{equation*}
    e
    \begin{equation*}
       \diff*{e^{\mu A(\lambda)}}{\lambda} = \left[\int_0^\mu \dli\tau e^{\tau A(\lambda)} \diff{A(\lambda)}{\lambda} e^{-\tau A(\lambda)}\right] e^{\mu A(\lambda)}
    \end{equation*}
    para todo \(\mu.\)
\end{lemma}
\begin{proof}
   Consideramos as aplicações
   \begin{equation*}
      F(\mu, \lambda) = \int_0^\mu \dli\tau e^{-\tau A(\lambda)} \diff{A(\lambda)}{\lambda} e^{\tau A(\lambda)}
      \quad\text{e}\quad
      G(\mu, \lambda) =  e^{-\mu A(\lambda)}\diff*{e^{\mu A(\lambda)}}{\lambda}
   \end{equation*}
   com \(F(0, \lambda) = 0 = G(0, \lambda)\) trivialmente. Diferenciando as funções em relação a \(\mu,\) obtemos
   \begin{equation*}
      \diffp{F(\mu,\lambda)}{\mu} = e^{-\mu A(\lambda)} \diff{A(\lambda)}{\lambda} e^{\mu A(\lambda)}
   \end{equation*}
   pelo teorema fundamental do cálculo e
   \begin{align*}
      \diffp{G(\mu, \lambda)}{\mu} &= e^{-\mu A(\lambda)} \diff*{\left[A(\lambda) e^{\mu A(\lambda)}\right]}{\lambda}-e^{-\mu A(\lambda)} A(\lambda) \diff*{e^{\mu A(\lambda)}}{\lambda}\\
                                   &= e^{-\mu A(\lambda)} \diff{A(\lambda)}{\lambda} e^{\mu A(\lambda)} + e^{-\mu A(\lambda)} A(\lambda) \diff*{e^{\mu A(\lambda)}}{\lambda} - e^{-\mu A(\lambda)} A(\lambda) \diff*{e^{\mu A(\lambda)}}{\lambda}\\
                                   &= e^{-\mu A(\lambda)} \diff{A(\lambda)}{\lambda} e^{\mu A(\lambda)},
   \end{align*}
   onde usamos que \(e^{\mu A(\lambda)} A(\lambda) = \diffp{e^{\mu A(\lambda)}}{\mu} = A(\lambda) e^{\mu A(\lambda)}.\) Para todo \(\lambda,\) mostramos que \(F(0,\lambda) = G(0, \lambda)\) e que \(\diffp{F(\mu, \lambda)}{\mu} = \diffp{G(\mu, \lambda)}{\mu},\) portanto concluímos que \(G(\mu, \lambda) = F(\mu, \lambda)\) para todos \(\mu, \lambda\). A partir disso, sabemos que
   \begin{equation*}
      \diff*{e^{\mu A(\lambda)}}{\lambda} = e^{\mu A(\lambda)} \int_0^\mu \dli{\tau} e^{-\tau A(\lambda)} \diff{A(\lambda)}{\lambda} e^{\tau A(\lambda)}.
   \end{equation*}
   Conjugando a igualdade \(G(\mu \lambda) = F(\mu, \lambda)\) por \(e^{\mu A(\lambda)}\) e multiplicando \(e^{\mu A(\lambda)}\) pela direita, obtemos
   \begin{align*}
      \diff*{e^{\mu A(\lambda)}}{\lambda} &= \left[e^{\mu A(\lambda)} F(\mu, \lambda) e^{-\mu A(\lambda)}\right]e^{\mu A(\lambda)}\\
                                          &= \left[\int_0^\mu \dli{\tau} e^{(\mu - \tau) A(\lambda)} \diff{A(\lambda)}{\lambda} e^{-(\mu - \tau)A(\lambda)}\right] e^{\mu A(\lambda)}.
   \end{align*}
   Com a substituição \(\tau \to \mu - \tau,\) temos
   \begin{equation*}
      \diff*{e^{\mu A(\lambda)}}{\lambda} = \left[\int_0^\mu \dli{\tau} e^{\tau A(\lambda)} \diff{A(\lambda)}{\lambda} e^{-\tau A(\lambda)}\right] e^{\mu A(\lambda)}.
   \end{equation*}
   como desejado.
\end{proof}

\begin{exercício}{Operador evolução temporal para o oscilador harmônico forçado}{ex1}
   A transformação unitária que resolve o problema do oscilador forçado não precisa apenas produzir solução para a equação de movimento de Heisenberg,
   \begin{equation*}
      \herm{U}(t) a U(t) = a(t) = a(0) e^{-it} + s(t),
   \end{equation*}
   mas também para a equação de Schrödinger. A transformação \(U(t) = D(s(t)) e^{-i H_0 t}\) não satisfaz essa última condição. Mostre que o fator de fase que está faltando é \(e^{i \alpha(t)},\) onde
   \begin{equation*}
      \alpha(t) = - \frac{1}{\sqrt{2}} \int_0^t \dli{\tau} f(\tau) \Re\left[s(\tau)\right].
   \end{equation*}
\end{exercício}
\begin{proof}[Resolução]
   Antes de tomar a derivada temporal do operador \(U(t)\) que realiza a equação de movimento de Heisenberg para o operador de aniquilação, preparamos alguns resultados preliminares. Notemos que
   \begin{align*}
      \commutator{D(z)}{H_0} &= \commutator{D(z)}{\herm{a}a}\\
                             &= \commutator{D(z)}{\herm{a}}a + \herm{a}\commutator{D(z)}{a}\\
                             &= \diffp{D(z)}{a}a - \herm{a} \diffp{D(z)}{\herm{a}}\\
                             &= -z^*D(z) a - z\herm{a} D(z)\\
                             &= -(z^* a + z \herm{a}) D(z) - z^* \commutator{D(z)}{a}\\
                             &= -(z^*a + z \herm{a} - \abs{z}^2) D(z).
   \end{align*}
   Escrevamos \(A(t) = z \herm{a} - z^* a\), \(\dot{z} = \diff{z}{t}\) e \(\dot{z}^* = \diff{z^*}{t}\). Pelo \cref{lem:derivada} e pelo \cref{ex:ex2}, temos
   \begin{align*}
      \diffp{D(z)}{t}%&= \diffp{e^{A(t)}}{t}\\
      &= \left[\int_0^1 \dli{\tau} e^{\tau A(t)} \diff{A(t)}{t} e^{-\tau A(t)}\right]D(z)\\
      &= \left[\int_0^1 \dli{\tau} D(\tau z) \left(\dot{z}\herm{a} - \dot{z}^*a\right) \herm{D}(\tau z)\right]D(z)\\
      &= \left[\int_0^1 \dli{\tau} \left(\dot{z} \herm{a} - \tau \dot{z}z^* - \dot{z}^* a + \tau \dot{z}^* z\right)\right]D(z)\\
      &= \left[\dot{z} \herm{a} - \dot{z}^* a - \frac12 \left(\dot{z} z^* - \dot{z}^* z\right)\right]D(z),
   \end{align*}
   onde usamos que \(D(z) a \herm{D}(z) = a - z\) e \(D(z) \herm{a} \herm{D}(z) = \herm{a} - z^*\). Para \(s(t) = - \frac{i}{\sqrt{2}} e^{-it} \int_0^t \dli{\tau} f(\tau) e^{i \tau},\) temos
   \begin{equation*}
      i\diff{s(t)}{t} = s(t) + \frac{f(t)}{\sqrt{2}}\quad\text{e}\quad -i \diff{s^*(t)}{t} = s^*(t) + \frac{f(t)}{\sqrt{2}},
   \end{equation*}
   para \(f(t)\) real. Assim, como \(i \dot{s} s^* - i\dot{s}^* s = 2\abs{s} + \frac{f}{\sqrt{2}} (s^* + s),\) 
   \begin{equation*}
      i\diffp{D(s(t))}{t} = \left[s(t) \herm{a} + s^*(t)a + H_\mathrm{int} - \left(\abs{s(t)}^2 + \frac{f(t)}{\sqrt{2}} \Re[s(t)]\right)\right]D(s(t))
   \end{equation*}
   é a derivada temporal de \(D(s(t)).\) Agora computamos a derivada temporal de \(U(t)\) e verificamos que não equivale a \(H U(t),\)
   \begin{align*}
      i\diffp{U(t)}{t} &= i\diffp{D(s(t))}{t} e^{-i H_0 t} + D(s(t)) H_0e^{-i H_0t}\\
                       &= i \diffp{D(s(t))}{t} e^{-iH_0t} + H_0 D(s(t))e^{-iH_0 t} + \commutator{D(s(t))}{H_0} e^{-iH_0 t}\\
                       &= \left\{\left[s(t) \herm{a} + s^*(t)a + H_\mathrm{int} - \left(\abs{s(t)}^2 + \frac{f(t)}{\sqrt{2}} \Re[s(t)]\right)\right] + H_0 - \left[s(t) \herm{a} + s^*(t) a - \abs{s(t)}^2\right]\right\}U(t)\\
                       &= \left[H_\mathrm{int} + H_0 - \frac{f(t)}{\sqrt{2}} \Re\left[s(t)\right]\right] U(t)\\
                       &= H U(t) - \frac{f(t)}{\sqrt{2}} \Re\left[s(t)\right]U(t),
   \end{align*}
   isto é, \(U(t)\) não pode ser o operador evolução temporal, a menos que \(s(t)\) seja puramente imaginária. Consideramos uma fase \(e^{i\alpha(t)}\) e o ansatz \(\tilde{U}(t) = e^{i \alpha(t)} U(t),\) que satisfaz
   \begin{align*}
      i\diffp{\tilde{U}(t)}{t} &= - \diff{\alpha(t)}{t} e^{i \alpha(t)} U(t) + e^{i \alpha(t)} \left(H - \frac{f(t)}{\sqrt{2}}\Re\left[s(t)\right]\right)U(t)\\
                               &= H \tilde{U}(t) - \left(\diff{\alpha(t)}{t} + \frac{f(t)}{\sqrt{2}}\Re\left[s(t)\right]\right) \tilde{U}(t)
   \end{align*}
   portanto \(\tilde{U}(t)\) é o operador evolução temporal se \(\diff{\alpha(t)}{t} = - \frac{f(t)}{\sqrt{2}}\Re\left[s(t)\right].\) Notemos que
   \begin{equation*}
      \diff*{\left(\frac1{\sqrt{2}} \int_0^t \dli{\tau} f(\tau) \Re\left[s(\tau)\right]\right)}{t} 
      = \frac{1}{\sqrt{2}} f(t) \Re\left[s(t)\right],
   \end{equation*}
   isto é, podemos tomar 
   \begin{equation*}
      \alpha(t) = - \frac{1}{\sqrt{2}} \int_0^t \dli{\tau} f(\tau) \Re\left[s(\tau)\right]
   \end{equation*}
   de tal sorte que \(i \diffp{\tilde{U}(t)}{t} = H \tilde{U}(t).\)
\end{proof}
