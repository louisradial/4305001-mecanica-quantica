% vim: spl=pt
\begin{exercício}{Operador evolução temporal para o oscilador harmônico forçado}{ex1}
   A transformação unitária que resolve o problema do oscilador forçado não precisa apenas produzir solução para a equação de movimento de Heisenberg,
   \begin{equation*}
      \herm{U}(t) a U(t) = a(t) = a(0) e^{-it} + s(t),
   \end{equation*}
   mas também para a equação de Schrödinger. A transformação \(U(t) = D(s(t)) e^{-i H_0 t}\) não satisfaz essa última condição. Mostre que o fator de fase que está faltando é \(e^{i \alpha(t)},\) onde
   \begin{equation*}
      \alpha(t) = - \frac{1}{\sqrt{2}} \int_0^t \dli{\tau} f(\tau) \Re\left[s(\tau)\right].
   \end{equation*}
\end{exercício}
\begin{proof}[Resolução]
   Notemos que
   \begin{align*}
      \commutator{D(z)}{H_0} &= \commutator{D(z)}{\herm{a}a}\\
                             &= \commutator{D(z)}{\herm{a}}a + \herm{a}\commutator{D(z)}{a}\\
                             &= \diffp{D(z)}{a}a - \herm{a} \diffp{D(z)}{\herm{a}}\\
                             &= -z^*D(z) a - z\herm{a} D(z)\\
                             &= -(z^* a + z \herm{a}) D(z) - z^* \commutator{D(z)}{a}\\
                             &= -(z^*a + z \herm{a} + \abs{z}^2) D(z).
   \end{align*}
   Para \(s(t) = - \frac{i}{\sqrt{2}} e^{-it} \int_0^t \dli{\tau} f(\tau) e^{i \tau},\) temos
   \begin{equation*}
      i\diff{s(t)}{t} = s(t) + \frac{f(t)}{\sqrt{2}}\quad\text{e}\quad -i \diff{s^*(t)}{t} = s^*(t) + \frac{f(t)}{\sqrt{2}},
   \end{equation*}
   para \(f(t)\) real. Assim,
   \begin{align*}
      i\diffp{U(t)}{t} &= i\diffp{D(s(t))}{t} e^{-i H_0 t} + D(s(t)) H_0e^{-i H_0t}\\
                    &= i \diffp*{\left(s(t) \herm{a} - s^*(t) a\right)}{t} D(s(t)) e^{-iH_0t} + H_0 D(s(t))e^{-iH_0 t} + \commutator{D(s(t))}{H_0} e^{-iH_0 t}\\
                    &= \left[s(t) \herm{a} + s^*(t) a + \frac{f(t)}{\sqrt{2}}(\herm{a} + a)\right]U(t) + H_0 U(t) + \commutator{D(s(t))}{H_0} e^{-iH_0t}\\
                    &= \left[H_\mathrm{int} + H_0 - \abs{s(t)}^2\right] U(t)\\
                    &= H U(t) - \abs{s(t)}^2 U(t),
   \end{align*}
   isto é, \(U(t)\) não pode ser o operador evolução temporal. Consideramos uma fase \(e^{i\alpha(t)}\) e o ansatz \(\tilde{U}(t) = e^{i \alpha(t)} U(t),\) que satisfaz
   \begin{equation*}
      i\diffp{\tilde{U}(t)}{t} = - \left[\diff{\alpha(t)}{t} + \abs{s(t)}^2\right] \tilde{U}(t) + H \tilde{U}(t),
   \end{equation*}
   portanto \(\tilde{U}(t)\) é o operador evolução temporal se \(\diff{\alpha(t)}{t} = - \abs{s(t)}^2.\) Notemos que
   \begin{align*}
      \diff*{\left(\frac1{\sqrt{2}} \int_0^t \dli{\tau} f(\tau) \Re\left[s(\tau)\right]\right)}{t} 
      &= \frac{1}{\sqrt{2}} f(t) \Re\left[s(t)\right]\\
      &= \frac{1}{2\sqrt{2}} f(t) \left[s(t) + s^*(t)\right]\\
      &= \frac{1}{4i} f(t) \left[\int_0^t \dli{\tau} f(\tau) \left(e^{i(\tau-t)} - e^{-i(\tau-t)}\right)\right]\\
      &= \frac12 \int_0^t \dli{\tau}f(\tau) \sin(\tau - t) f(t)
   \end{align*}
   e que
   \begin{equation*}
      \abs{s(t)}^2 = \frac12 \int_0^t \dli{\tau} \int_0^t \dli{\eta} f(\tau) f(\eta) e^{i(\tau - \eta)}
   \end{equation*}
\end{proof}
