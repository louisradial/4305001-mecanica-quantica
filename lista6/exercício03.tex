% vim: spl=pt
\begin{exercício}{Estados coerentes}{ex3}
   Queremos encontrar estados \(\ket{\varphi; t}\) tais que os valores esperados de \(H\) e de \(a\) tenham propriedades idênticas as suas propriedades clássicas, a menos da energia de ponto zero. 
   \begin{enumerate}[label=(\alph*)]
      \item Para isso, se \(\mean{a}(t) = \bra{\varphi; t}a\ket{\varphi; t},\) mostre que \(i \diff{\mean{a}(t)}{t} = \omega \mean{a}(t),\) isto é, que \(\mean{a}(t)\) obedece a mesma equação diferencial que \(z(t).\) Definimos o número complexo \(z_0 = \mean{a}(0)\) de tal sorte que \(\mean{a}(t) = z_0 e^{-i \omega t}.\)

      \item Vamos exigir que o valor esperado de \(H\) seja \(\bra{\varphi; 0} H \ket{\varphi; 0} = \hbar \omega \left(\abs{z_0}^2 + \frac12\right)\). Considere o operador \(b(z_0) = a - z_0.\) Mostre que \(\bra{\varphi; 0} \herm{b}(z_0) b(z_0) \ket{\varphi; 0} = 0,\) deduzindo que \(a \ket{\varphi; 0} = z_0 \ket{\varphi; 0},\) logo o estado \(\ket{\varphi; 0}\) é o estado coerente \(\ket{z_0}\).
   \end{enumerate}
\end{exercício}
\begin{proof}[Resolução]
   Denotaremos \(a(t) = \herm{U}(t) a U(t)\) como o operador na representação de Heisenberg, com \(a = a(0)\) sendo o operador na representação de Schrödinger. Observemos que
   \begin{equation*}
      \mean{a}(t) = \bra{\varphi; t}a\ket{\varphi; t} = \bra{\varphi} \herm{U}(t) a U(t) \ket{\varphi} = \bra{\varphi} a(t) \ket{\varphi},
   \end{equation*}
   então
   \begin{equation*}
      i \diff{\mean{a}(t)}{t} = \bra{\varphi} i\diff{a(t)}{t} \ket{\varphi} = \bra{\varphi} \omega a(t) \ket{\varphi} = \omega \mean{a}(t).
   \end{equation*}
   Escrevendo \(\mean{a}(0) = z_0,\) temos \(\mean{a}(t) = z_0 e^{-i \omega t},\) análogo a \(z(t).\) 
   Tomando o conjugado da condição inicial para \(\mean{a}\) e repetindo os argumentos para \(\herm{a}\), obtemos \(\mean{\herm{a}}(t) = z_0^* e^{-i \omega t}\).

   Vamos supor que \(\bra{\varphi}H\ket{\varphi} = \hbar \omega \left(\abs{z_0}^2 + \frac12\right)\) e consideramos o operador \(b(z_0) = a - z_0,\) com
   \begin{equation*}
      \herm{b}(z_0) b(z_0) = (\herm{a} - z_0^*)(a - z_0) = \herm{a}a - z_0 \herm{a} - z_0^* a + \abs{z_0}^2 = \frac1{\hbar \omega} H + \abs{z_0}^2 - \frac12 - z_0 \herm{a} - z_0^* a.
   \end{equation*}
   Assim, tomando o valor esperado em \(\ket{\varphi},\) obtemos
   \begin{align*}
      \bra{\varphi} \herm{b}(z_0)b(z_0) \ket{\varphi} &= \frac{1}{\hbar \omega} \bra{\varphi}H\ket{\varphi} + \abs{z_0}^2 - \frac12 - z_0 \mean{\herm{a}}(0) - z_0^* \mean{a}(0)\\
                                                      &= 2\abs{z_0}^2 - \abs{z_0}^2 - \abs{z_0}^2\\
                                                      &= 0.
   \end{align*}
   Isto é, a norma do vetor \(b(z_0) \ket{\varphi}\) é zero, portanto \(a \ket{\varphi} = z_0 \ket{\varphi}\) e vemos que \(\ket{\varphi}\) é o estado coerente \(\ket{z_0}.\)
\end{proof}
