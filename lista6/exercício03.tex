% vim: spl=pt
\begin{exercício}{}{ex3}
   Queremos encontrar estados \(\ket{\varphi; t}\) tais que os valores esperados de \(H\) e de \(a\) tenham propriedades idênticas as suas propriedades clássicas, a menos da energia de ponto zero. 
   \begin{enumerate}[label=(\alph*)]
      \item Para isso, se \(\mean{a}(t) = \bra{\varphi; t}a\ket{\varphi; t},\) mostre que \(i \diff{\mean{a}(t)}{t} = \omega \mean{a}(t),\) isto é, que \(\mean{a}(t)\) obedece a mesma equação diferencial que \(z(t).\) Definimos o número complexo \(z_0 = \mean{a}(0)\) de tal sorte que \(\mean{a}(t) = z_0 e^{-i \omega t}.\)

      \item Vamos exigir que o valor esperado de \(H\) seja \(\bra{\varphi; 0} H \ket{\varphi; 0} = \hbar \omega \left(\abs{z_0}^2 + \frac12\right)\). Considere o operador \(b(z_0) = a - z_0.\) Mostre que \(\bra{\varphi; 0} \herm{b}(z_0) b(z_0) \ket{\varphi; 0} = 0,\) deduzindo que \(a \ket{\varphi; 0} = z_0 \ket{\varphi; 0},\) logo o estado \(\ket{\varphi; 0}\) é o estado coerente \(\ket{z_0}\).
   \end{enumerate}
\end{exercício}
\begin{proof}[Resolução]
    
\end{proof}
