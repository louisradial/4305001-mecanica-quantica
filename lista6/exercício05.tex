% vim: spl=pt
\begin{exercício}{Dependência de gauge no moemnto angular}{ex5}
   No movimento de uma partícula carregada em um campo magnético, os operadores \(v_x \pm i v_y\) levantam e abaixam os autovalores de \(L_z,\) mesmo \(v_x\) e \(v_y\) sendo combinações lineares de variáveis canônicas ao longo de direções diferentes. Isso está relacionado ao fato que a definição de momento angular é dependente de calibre.
   \begin{enumerate}[label=(\alph*)]
      \item Mostre que se \(\vetor{A} \to \vetor{A} + \nabla\chi,\) então \(\vetor{L} \to \vetor{L} + \frac{e}{c} \vetor{r} \times \nabla\chi.\)
      \item Mostre que para um calibre arbitrário, vale
         \begin{equation*}
            \commutator{v_x}{L_z} = \frac{\hbar}{im} p_y + \frac{e \hbar}{mc} \mathcal{L}_z A_x
            \quad\text{e}\quad
            \commutator{v_y}{L_z} = -\frac{\hbar}{im} p_y + \frac{e \hbar}{mc} \mathcal{L}_z A_y,
         \end{equation*}
         onde \(\mathcal{L}_z = \frac1i \left(x \partial_y - y \partial_x\right).\)
      \item Como o item anterior resolve o problema?
   \end{enumerate}
\end{exercício}
\begin{proof}[Resolução]
   Pelo \cref{lem:calibre}, temos \(\vetor{p} \to \vetor{p}' = \vetor{p} + \frac{e}{c} \nabla \chi,\) portanto
   \begin{equation*}
      \vetor{L} \to \vetor{L}
   \end{equation*}
\end{proof}
