% vim: spl=pt
\begin{exercício}{Calibre de Landau}{ex7}
   O movimento em um campo magnético pode também ser tratado usando o calibre \(A_x = - By\) e \(A_y = A_z = 0,\) chamado de calibre de Landau. 
   \begin{enumerate}[label=(\alph*)]
      \item Mostre que no calibre de Landau o par conveniente de constantes de movimento são \(p_x\) e o hamiltoniano para o movimento no plano \(xy\),
         \begin{equation*}
            H_L = \frac{p_y^2}{2m} + \frac12 m \omega_c^2 (y - y_0)^2,
         \end{equation*}
         onde \(y_0 = - \frac{cp_x}{eB}.\)
      \item Mostre que o espectro desse hamiltoniano é
         \begin{equation*}
            E_n = \hbar \omega_c \left(n + \frac12\right),\quad\text{com}\quad n \in \mathbb{N}_0,
         \end{equation*}
         onde \(\hbar k\) é o autovalor de \(p_x,\) que varia continuamente em \(\mathbb{R},\) mas que não afeta a energia. Mostre que as autofunções são
         \begin{equation*}
            \Phi_{n, \kappa}(x,y) = \frac{1}{\sqrt{2\pi}} e^{i \kappa \xi} \phi_n(\eta + \kappa),
         \end{equation*}
         onde \(\kappa = k \ell_B\) e \(\phi_n\) é a \(n\)-ésima autofunção do oscilador harmônico unidimensional.
      \item Seja \(\tilde{\Psi}_{n,\mu}\) a autofunção de energia e momento angular bem definidos após a transformação para o calibre de Landau. Mostre que os coeficientes da expansão
         \begin{equation*}
            \Phi_{0, \kappa} = \sum_{\mu} c_\mu(\kappa) \tilde{\Psi}_{0,\mu}
         \end{equation*}
         são dados por
         \begin{equation*}
            \exp\left(\frac14 z^2 + i \kappa z\right) = \pi^{\frac14} \exp\left(\frac{\kappa^2}{4}\right) \sum_{\mu} c_{\mu}(\kappa) \frac{z^\mu}{\sqrt{\mu!}}.
         \end{equation*}
   \end{enumerate}
\end{exercício}
\begin{proof}[Resolução]
   No gauge de Landau, temos
   \begin{equation*}
      \vetor{v} = \frac1m \left(\vetor{p} - \frac{e}{c} \vetor{A}\right) = \frac1m\vetor{p} + \omega_c y \vetor{e}_x,
   \end{equation*}
   então
   \begin{align*}
      H = \frac12 m \vetor{v}^2 &= \frac{p_y^2}{2m} + \frac{p_z^2}{2m} + \frac12 m\left(\frac{p_x}{m} + \omega_c y\right)^2\\
                                &= \frac{p_y^2}{2m} + \frac{p_z^2}{2m} + \frac12 m \omega_c^2 \left(y + \frac{p_x}{m\omega_c}\right)^2\\
                                &= \frac{p_z^2}{2m} + \frac{p_y^2}{2m} + \frac12m \omega_c^2 \left(y - y_0\right)^2,
   \end{align*}
   onde definimos \(y_0 = - \frac{p_x}{m \omega_c} = -\frac{c}{e B} p_x.\) Como não há \(x\) nem \(z\) na expressão do hamiltoniano, é claro que \(\commutator{p_x}{H} = \commutator{p_z}{H} = 0,\) portanto esses operadores são constantes de movimento. Assim, \(\commutator{H - \frac{p_z^2}{2m}}{H} = 0,\) isto é, o hamiltoniano da dinâmica no plano \(xy\),
   \begin{equation*}
      H_L = \frac{p_y^2}{2m} + \frac12 m \omega_c^2 (y - y_0)^2,
   \end{equation*}
   é uma constante de movimento. Como \(\commutator{p_x}{H_L} = 0,\) podemos diagonalizar \(p_x\) e \(H_L\) simultaneamente.

   Notemos que para \(H_L,\) \(p_x\) atua como um deslocamento constante, portanto adaptamos\footnote{Na notação daquele exercício, temos \(\lambda = - m \omega_c^2 y_0.\)} o desenvolvimento do \href{https://github.com/louisradial/4305001-mecanica-quantica/releases/tag/lista2}{Exercício 5 da Lista 2} e definimos os operadores de aniquilação e criação para \(H_L - \frac12 m \omega_c^2 y_0^2\)
   \begin{equation*}
      b = \frac{1}{\sqrt{2}} \left[\sqrt{\frac{m \omega_c}{\hbar}}\left(y - y_0\right) + \frac{i}{\sqrt{\hbar \omega_c m}} p_y\right]
      \quad\text{e}\quad
      \herm{b} = \frac{1}{\sqrt{2}} \left[\sqrt{\frac{m \omega_c}{\hbar}}\left(y - y_0\right) - \frac{i}{\sqrt{\hbar \omega_c m}} p_y\right]
   \end{equation*}
   de modo que
   \begin{equation*}
      H_L - \frac12 m \omega_c^2 y_0^2 = \hbar\omega_c \herm{b}b + \frac12\left(\hbar\omega - m\omega_c^2 y_0^2\right) = \hbar\omega_c \left(\herm{b}b + \frac12\right) - \frac12 m \omega_c^2 y_0^2,
   \end{equation*}
   isto é,
   \begin{equation*}
      H_L = \hbar \omega_c \left(\herm{b} b + \frac12\right).
   \end{equation*}
   Assim, o espectro de \(H_L\) é simplesmente o de um oscilador harmônico unidimensional de frequência \(\omega_c,\) sem a influência do autovalor do operador \(p_x.\) 

   Consideramos a base \(\ket{n, \kappa}\) em que \(H_L \ket{n, \kappa} = \hbar \omega_c (n + \frac12) \ket{n,\kappa}\) e \(p_x \ket{n,\kappa} = \hbar \ell_B^{-1} \kappa \ket{n, \kappa}\) e em particular as autofunções \(\Phi_{n, \kappa}(\xi', \eta') = \braket{\xi'\eta'}{n,\kappa}.\) Como no \cref{ex:ex6}, consideramos os operadores adimensionais, então
   \begin{equation*}
      b = \frac{1}{\sqrt{2}} \left[\eta + p_\xi + ip_\eta\right]
      \quad\text{e}\quad
      \herm{b} = \frac{1}{\sqrt{2}} \left[\eta + p_\xi - ip_\eta\right]
   \end{equation*}
   Vamos obter a autofunção do estado \(\ket{0,\kappa}\) e utilizar o operador de criação para obter a autofunção do estado \(\ket{n, \kappa}.\) Este estado satisfaz \(p_\xi \ket{0,\kappa} = \kappa\ket{0,\kappa}\) e \(b\ket{0,\kappa} = 0,\) portanto 
   \begin{equation*}
      \kappa \Phi_{0, \kappa}(\xi', \eta') = \bra{\xi'\eta'} p_\xi \ket{0,\kappa} = -i \diffp{\Phi_{0, \kappa}(\xi', \eta')}{\xi'} \implies \diffp{\Phi_{0, \kappa}(\xi', \eta')}{\xi'} = i \kappa \Phi_{0, \kappa}(\xi', \eta')
   \end{equation*}
   e
   \begin{equation*}
      0 = \bra{\xi'\eta'} b \ket{0,\kappa}  = \frac{1}{\sqrt{2}}\left(\eta' - i \diffp*{}{\xi'} + \diffp*{}{\eta'}\right)\braket{\xi'\eta'}{0,\kappa} \implies \left(\eta' - i \diffp*{}{\xi'} + \diffp*{}{\eta'}\right)\Phi_{0, \kappa}(\xi', \eta') = 0.
   \end{equation*}
   Substituindo a primeira equação na segunda, obtemos
   \begin{equation*}
      \left(\diffp{}{\eta'} + \eta' +  \kappa\right)\Phi_{0, \kappa}(\xi', \eta') = 0,
   \end{equation*}
   e vemos que podemos separar as variáveis de acordo com \(\Phi_{0, \kappa}(\xi', \eta') = \alpha e^{i\kappa \xi} \tilde{\phi}_0(\eta'),\) onde \(\alpha \in \mathbb{C}\) é uma constante. Escrevemos \(\zeta = \eta' + \kappa\) e \(\phi_0(\zeta) = \tilde{\phi}_0(\eta'),\) então a equação diferencial pode ser escrita como a equação diferencial satisfeita pela função de onda do estado fundamental de um oscilador harmônico unidimensional,
   \begin{equation*}
      \diff{\phi_0(\zeta)}{\zeta} + \zeta \phi_0(\zeta) = 0,
   \end{equation*}
   portanto \(\phi_0(\zeta) = \pi^{-\frac14} e^{-\frac12 \zeta^2}\) e temos
   \begin{equation*}
      \Phi_{0, \kappa}(\xi', \eta') = \frac{1}{\sqrt{2\pi}}e^{i\kappa \xi'} \phi_0(\eta' + \kappa)
   \end{equation*}
   como a função de onda do estado \(\ket{0,\kappa},\) onde escolhemos \(\alpha = \frac{1}{\sqrt{2\pi}},\) já que esta função de onda não é normalizável.\footnote{Neste gauge, deveríamos considerar estados dados por pacotes de onda formado por estados como \(\ket{n, \kappa}\).} 

   Consideramos o conjunto
   \begin{equation*}
      S = \setc*{n \in \mathbb{N}_0}{\Phi_{n, \kappa}(\xi', \eta') = \frac1{\sqrt{2\pi}} e^{i \kappa \xi'} \phi_n(\eta' + \kappa)},
   \end{equation*}
   onde \(\phi_n\) é a \(n\)-ésima autofunção do oscilador harmônico unidimensional. Já mostramos que \(0 \in S,\) portanto \(S\) é não vazio. Assim, consideramos \(n \in S\) e obtemos a função de onda \(\Phi_{n+1,\kappa}(\xi', \eta')\) notando que
   \begin{equation*}
      \Phi_{n+1,\kappa}(\xi', \eta') = \braket{\xi'\eta'}{n+1,\kappa} = \frac{1}{\sqrt{n+1}} \bra{\xi'\eta'} \herm{b}\ket{n,\kappa} = \frac{1}{\sqrt{2(n+1)}} \left(\eta' - i \diffp{}{\xi'} - \diffp{}{\eta'}\right)\Phi_{n,\kappa}(\xi'\eta'),
   \end{equation*}
   isto é,
   \begin{equation*}
      \Phi_{n+1,\kappa}(\xi', \eta') = \frac{e^{i\kappa \xi'}}{\sqrt{4\pi(n+1)}} \left(\eta' + \kappa - \diffp{}{\eta'}\right)\phi_n(\eta' + \kappa).
   \end{equation*}
   Repetindo a mesma mudança de variáveis, \(\zeta = \eta' + \kappa,\) e utilizando a propriedade das funções de onda para o oscilador harmônico unidimensional, temos
   \begin{equation*}
      \frac{1}{\sqrt{2(n+1)}} \left(\zeta - \diff*{}{\zeta}\right)\phi_n(\zeta) = \phi_{n+1}(\zeta),
   \end{equation*}
   logo,
   \begin{equation*}
      \Phi_{n+1,\kappa}(\xi', \eta') = \frac{1}{\sqrt{2\pi}} e^{i \kappa \xi'} \phi_{n+1}(\eta' + \kappa),
   \end{equation*}
   isto é, \(n + 1 \in S.\) Pelo princípio de indução finita, concluímos que \(S = \mathbb{N}_0,\) e encontramos assim as autofunções dos estados \(\ket{n,\kappa}.\)

   Notemos que para o obter o gauge de Landau podemos considerar o gauge simétrico e adicionar o gradiente de \(\chi(x,y,z) = -\frac12 B xy\) ao potencial vetor daquele calibre. Isto é, no gauge de Landau, as autofunções de energia \(n = 0\) e momento angular bem definidos são
   \begin{equation*}
      \tilde{\Psi}_{0, \mu}(\xi',\eta') = \exp\left(-i \frac12\xi'\eta'\right) \Psi_{0,\mu}(\xi', \eta') = \frac{1}{\sqrt{2\pi \mu!}} \left(\frac{\xi' + i \eta'}{\sqrt{2}}\right)^\mu \exp\left(-\frac{{\xi'}^2 + 2 i \eta' \xi' +  {\eta'}^2}{4}\right).
   \end{equation*}
   Consideramos a expansão
   \begin{equation*}
      \Phi_{0,\kappa}(\xi', \eta') = \sum_{\mu = 0}^\infty c_{\mu, \kappa}\tilde{\Psi}_{0,\mu}(\xi', \eta')
   \end{equation*}
   e desejamos simplificar a relação para determinar os coeficientes \(c_{\mu,\kappa}.\) Começamos dividindo ambos os lados pelo termo de \(\tilde{\Psi}_{0,\mu}\) que não depende de \(\mu,\) e obtemos
   %%
   \begin{align*}
      \pi^{\frac14} \sum_{\mu = 0}^{\infty} \frac{c_{\mu,\kappa}}{\sqrt{\mu!}}\left(\frac{\xi' + i \eta'}{\sqrt{2}}\right)^\mu 
      &= \exp\left[i \kappa \xi' - \frac12 (\eta' + \kappa)^2 + \frac14({\xi'}^2 + 2 i \eta' \xi' + {\eta'}^2)\right]\\
      &= \exp\left[i \kappa (\xi' + i \eta') - \frac12 \kappa^2 + \frac14 ({\xi'}^2 + 2i \eta' \xi' - {\eta'}^2)\right]\\
      &= \exp\left(-\frac12 \kappa^2\right) \exp\left[i \kappa (\xi' + i \eta') + \frac14 (\xi' + i \eta')^2\right].
   \end{align*}
   Definindo \(w = \xi' + i \eta',\) temos
   \begin{equation*}
      \exp\left[\frac14 w^2 + i \kappa w\right] = \pi^{\frac14} \exp\left(\frac12 \kappa^2\right) \sum_{\mu = 0}^\infty \frac{c_{\mu, \kappa}}{\sqrt{\mu!}} \left(\frac{w}{\sqrt{2}}\right)^\mu
   \end{equation*}
   como desejado.
\end{proof}
