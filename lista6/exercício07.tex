% vim: spl=pt
\begin{exercício}{Calibre de Landau}{ex7}
    O movimento em um campo magnético pode também ser tratado usando o calibre \(A_x = - By\) e \(A_y = A_z = 0,\) chamado de calibre de Landau. 
    \begin{enumerate}[label=(\alph*)]
        \item Mostre que no calibre de Landau o par conveniente de constantes de movimento são \(p_x\) e o hamiltoniano para o movimento no plano \(xy\),
           \begin{equation*}
              H_L = \frac{p_y^2}{2m} + \frac12 m \omega_c^2 (y - y_0)^2,
           \end{equation*}
           onde \(y_0 = - \frac{cp_x}{eB}.\)
        \item Mostre que o espectro desse hamiltoniano é
           \begin{equation*}
              E_n = \hbar \omega_c \left(n + \frac12\right),\quad\text{com}\quad n \in \mathbb{N}_0,
           \end{equation*}
           onde \(\hbar k\) é o autovalor de \(p_x,\) que varia continuamente em \(\mathbb{R},\) mas que não afeta a energia. Mostre que as autofunções são
           \begin{equation*}
              \Phi_{n \kappa}(x,y) = \sqrt{2\pi} e^{i \kappa \xi} \phi_n(\eta + \kappa),
           \end{equation*}
           onde \(\kappa = k \ell_B\) e \(\phi_n\) é a \(n\)-ésima autofunção do oscilador harmônico unidimensional.
        \item Seja \(\tilde{\Psi}_{n \mu}\) a autofunção de energia e momento angular bem definidos após a transformação para o calibre de Landau. Mostre que os coeficientes da expansão
           \begin{equation*}
              \Phi_{0 \kappa} = \sum_{\mu} c_\mu(\kappa) \tilde{\Psi}_{0 \mu}
           \end{equation*}
           são dados por
           \begin{equation*}
              \exp\left(\frac14 z^2 + i \kappa z\right) = \pi^{\frac14} \exp\left(\frac{\kappa^2}{4}\right) \sum_{\mu} c_{\mu}(\kappa) \frac{z^\mu}{\sqrt{\mu!}}.
           \end{equation*}
    \end{enumerate}
\end{exercício}
\begin{proof}[Resolução]
    
\end{proof}
