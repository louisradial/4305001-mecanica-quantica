% vim: spl=pt
\begin{exercício}{Composição de operadores de deslocamento}{ex2}
   Os deslocamentos usados para definir estados coerentes não formam um grupo abeliano. Mostre que sua lei de composição é muito próxima a de um grupo abeliano, a saber,
   \begin{equation*}
      D(z_1) D(z_2) = D(z_1 + z_2) \exp\left[i \Im(z_1 z_2^*)\right].
   \end{equation*}
\end{exercício}
\begin{proof}[Resolução]
   Notemos que
   \begin{equation*}
      \commutator{z_1 \herm{a} - z_1^* a}{z_2 \herm{a} - z_2^* a} = z_1 z_2^* -  z_1^* z_2 = 2i \Im\left(z_1 z_2^*\right),
   \end{equation*}
   portanto
   \begin{align*}
      D(z_1)D(z_2) &= \exp\left[z_1 \herm{a} - z_1^* a\right]\exp\left[z_2 \herm{a} - z_2^* a\right]\\
                   &= \exp\left[\left(z_1 \herm{a} - z_1^* a\right) + \left(z_2 \herm{a} - z_2^* a\right) + \frac12\commutator{z_1 \herm{a} - z_1^* a}{z_2 \herm{a} - z_2^* a} \right]\\
                   &= \exp\left[(z_1 + z_2) \herm{a} - (z_1 + z_2)^* a + i \Im\left(z_1 z_2^*\right)\right]\\
                   &= D(z_1 + z_2) \exp\left[i \Im\left(z_1 z_2^*\right)\right]
   \end{align*}
   pela fórmula de Baker-Campbell-Hausdorff. Ainda, vemos que para cada \(\theta \in [0,2\pi)\)  temos
   \begin{align*}
      z_1, z_2 \in \Omega_\theta = \setc{z \in \mathbb{C}}{\exists \rho \in \mathbb{R} : z = \rho e^{i\theta}}
      &\implies z_1 z_2 = \rho_1 \rho_2 \in \mathbb{R}\\
      &\implies D(z_1)D(z_2) = D(z_1 + z_2).
   \end{align*}
   Assim, para cada \(\theta,\) a aplicação
   \begin{align*}
      D_\theta : \Omega_\theta &\to \mathcal{L}(\mathscr{H}, \mathscr{H})\\
                             z &\mapsto D(z)
   \end{align*}
   define um subgrupo uniparamétrico, com \(D_\theta(0) = \unity\) e \(D_\theta(z_1)D_\theta(z_2) = D_\theta(z_1 + z_2).\)
\end{proof}
