% vim: spl=pt
\begin{exercício}{Decomposição polar e estados coerentes}{ex4}
   Para tentar definir um operador de fase \(\Phi,\) escrevemos o operador de aniquilação como \(a = A e^{i \Phi},\) com \(A = \herm{A}.\) Mostre que
   \begin{equation*}
      A = \sqrt{N + 1} = \sum_{n = 0}^{\infty} \ket{n} \sqrt{n+1} \bra{n},
   \end{equation*}
   e que \(e^{i \Phi}\) não é unitário e é igual a
   \begin{equation*}
      E = \sum_{n = 0}^\infty \ket{n}\bra{n+1},
   \end{equation*}
   logo \(\Phi\) não é um operador hermitiano. Mostre que os operadores \(C\) e \(S\) definidos por
   \begin{equation*}
      C = \frac{1}{2} \left(E + \herm{E}\right)
      \quad\text{e}\quad
      S = \frac1{2i} \left(E - \herm{E}\right)
   \end{equation*}
   são análogos a \(\cos\phi\) e \(\sin\phi.\) Calcule os comutadores \(\commutator{C}{S},\) \(\commutator{C}{N},\) \(\commutator{S}{N},\) e a soma \(C^2 + S^2.\) Quais as relações de incerteza que podemos deduzir para esses operadores? Calcule os valores médios de \(C\) e \(S\) no estado coerente \(\ket{z},\) onde \(z = \abs{z} e^{i\theta}.\)
\end{exercício}
\begin{proof}[Resolução]
    
\end{proof}
