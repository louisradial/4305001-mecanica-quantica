% vim: spl=pt
\begin{exercício}{Autofunções de momento angular para o movimento em campo magnético}{ex6}
   Mostre que as autofunções dos estados estacionários de momento angular \(\mu\) para o movimento em campo magnético são
   \begin{equation*}
      \Psi_{0\mu}(\xi, \eta) = \frac1{\sqrt{2\pi \mu!}}\left(\frac{\xi + i \eta}{\sqrt{2}}\right)^\mu \exp\left(-\frac{\rho^2}{4}\right),
   \end{equation*}
   onde \(\xi = \frac{x}{\ell_B},\) \(\eta = \frac{y}{\ell_B},\) e \(\rho^2 = \xi^2 + \eta^2.\)
\end{exercício}
\begin{proof}[Resolução]
   % No gauge simétrico, temos \(A_x = -\frac12 y B\) e \(A_y = \frac12 xB,\) portanto
   % \begin{equation*}
   %    v_x = \frac{1}{m}p_x + \frac12 \omega_c y
   %    \quad\text{e}\quad
   %    v_y = \frac{1}{m}p_y - \frac12 \omega_c x,
   % \end{equation*}
   % onde \(\omega_c = \frac{e B}{m c}.\) Assim,
   % \begin{align*}
   %    \frac12m(v_x^2 + v_y^2) &= \frac{1}{2m}p_x^2 + \frac12 \omega_c y p_x + \frac12m\left(\frac12\omega_c\right)^2 y^2  + \frac{1}{2m}p_y^2 - \frac12 \omega_c x p_y + \frac12m\left(\frac12\omega_c\right)^2 x^2\\
   %                            &= \underbrace{\left[\frac1{2m} p_x^2 + \frac12m\left(\frac12\omega_c\right)^2 x^2\right]}_{H_x} + \underbrace{\left[\frac1{2m} p_y^2 + \frac12m\left(\frac12 \omega_c\right)^2 y^2\right]}_{H_y} - \frac12 \omega_c (x p_y - y p_x)\\
   %                            &= H_x + H_y - \frac12 \omega_c L_z,
   % \end{align*}
   % portanto nesse gauge temos \(H = H_x + H_y + \frac1{2m} p_z^2 - \frac12 \omega_c L_z,\) onde \(H_x\) e \(H_y\) são termos de osciladores harmônicos unidimensionais de frequências \(\omega = \frac12 \omega_c\) em suas respectivas direções. É claro que o termo \(\frac1{2m} p_z^2\) comuta com os demais, portanto restringimos a discussão ao hamiltoniano da dinâmica no plano transversal ao eixo \(z\), \(H_\perp = H_x + H_y - \frac12 \omega_c L_z.\) Notemos que 
   % \begin{align*}
   %    \commutator{L_z}{H_x} &= \commutator{xp_y - y p_x}{\frac1{2m} p_x^2 + \frac12m \omega^2x^2}&
   %    \commutator{L_z}{H_y} &= \frac{1}{i \hbar}\commutator{xp_y - y p_x}{\frac1{2m} p_y^2 + \frac12m \omega^2y^2}\\
   %                          &= \frac{i \hbar}m p_y p_x + i \hbar m \omega^2 yx&
   %                          &= - \frac{i \hbar}m p_y p_x - i \hbar m \omega^2 yx,
   % \end{align*}
   % portanto podemos diagonalizar \(H\) e \(L_z\) simultaneamente. Sabemos que o espectro de \(H\) é dado pelos níveis de Landau \(E_n = \hbar \omega_c (n + \frac12)\), portanto consideramos a base \(\ket{n,\mu}\) com 
   % \begin{equation*}
   %    H \ket{n,\mu} = E_n \ket{n,\mu}
   %    \quad\text{e}\quad
   %    L_z \ket{n,\mu} = \hbar \mu \ket{n,\mu}.
   % \end{equation*}

   Já sabemos que em qualquer calibre os operadores
   \begin{equation*}
      a = \sqrt{\frac{m}{2 \hbar \omega_c}} (v_x + i v_y)
      \quad\text{e}\quad
      \herm{a} = \sqrt{\frac{m}{2 \hbar \omega_c}} (v_x - i v_y)
   \end{equation*}
   abaixam e levantam o autovalor de energia, então pelo \cref{ex:ex5}, temos
   \begin{equation*}
      a \ket{n,\mu} = \sqrt{n} \ket{n-1, \mu + 1}\quad\text{e}\quad \herm{a}\ket{n,\mu} = \sqrt{n+1} \ket{n+1, \mu - 1}.
   \end{equation*}
   No calibre considerado, temos
   \begin{equation*}
      a = \frac{1}{\sqrt{\hbar \omega_c m}} \left(\frac{p_x + i p_y}{\sqrt{2}}\right) - i\sqrt{\frac{m \omega_c}{\hbar}} \left(\frac{x + iy}{2\sqrt{2}}\right) = \frac{\ell_B}{\hbar} \left(\frac{p_x + i p_y}{\sqrt{2}}\right) - \frac{i}{\ell_B}\left(\frac{x + i y}{2\sqrt{2}}\right) 
   \end{equation*}
   onde definimos o comprimento magnético \(\ell_B = \sqrt{\frac{\hbar}{m \omega_c}},\) portanto em termos dos operadores adimensionais
   \begin{equation*}
      \xi = \frac{1}{\ell_B}x,\quad
      \eta = \frac{1}{\ell_B}y,\quad
      p_\xi = \frac{\ell_B}{\hbar}p_x\quad\text{e}\quad
      p_\eta = \frac{\ell_B}{\hbar}p_y,
   \end{equation*}
   temos \(\commutator{\xi}{p_\xi} = i = \commutator{\eta}{p_\eta}\) e 
   \begin{equation*}
      a = \frac{p_\xi + i p_\eta}{\sqrt{2}} - i\frac{\xi + i \eta}{2\sqrt{2}}.
   \end{equation*}
   Assim, temos
   \begin{equation*}
      a\ket{0,\mu} = 0 \implies \bra{\xi'\eta'}a\ket{0,\mu} = 0 \implies \frac{1}{\sqrt{2}}\left(-i\diffp*{}{\xi'} + \diffp*{}{\eta'} -i \frac{\xi' + i \eta'}{2}\right)\braket{\xi'\eta'}{0,\mu} = 0,
   \end{equation*}
   isto é, \(\Psi_{0,\mu}(\xi', \eta') = \braket{\xi'\eta'}{0,\mu}\) satisfaz a equação diferencial
   \begin{equation*}
      \left(\diffp*{}{\xi'} + i \diffp*{}{\eta'}\right)\Psi_{0,\mu}(\xi', \eta') = -\frac{\xi' + i \eta'}{2} \Psi_{0,\mu}(\xi', \eta').
   \end{equation*}
   Consideramos \(\zeta = \xi' + i \eta',\) então
   \begin{equation*}
      \diffp{}{\xi'} = \diffp{\zeta}{\xi'} \diffp{}{\zeta} + \diffp{\zeta^*}{\xi'} \diffp{}{\zeta^*} = \diffp{}{\zeta} + \diffp{}{\zeta^*}
      \quad\text{e}\quad
      \diffp{}{\eta'} = \diffp{\zeta}{\eta'} \diffp{}{\zeta} + \diffp{\zeta^*}{\eta'} \diffp{}{\zeta^*} = i\diffp{}{\zeta} - i\diffp{}{\zeta^*},
   \end{equation*}
   logo \(\partial_{\xi'} + i \partial_{\eta'} = 2 \partial_{\zeta^*}\) e obtemos a equação diferencial
   \begin{equation*}
      \diffp{\psi(\zeta, \zeta^*)}{\zeta^*} = -\frac14 \zeta \psi(\zeta, \zeta^*)
   \end{equation*}
   onde escrevemos \(\psi(\zeta,\zeta^*) = \Psi_{0,\mu}(\xi', \eta').\) Como \(\zeta\) e \(\zeta^*\) são variáveis independentes, podemos integrar esta equação diferencial em \(\zeta^*\) com
   \begin{equation*}
      \psi(\zeta, \zeta^*) = f(\zeta)\exp\left(-\frac{\zeta \zeta^*}{4}\right)
   \end{equation*}
   para alguma função \(f(\zeta)\) que não depende de \(\zeta^*\). Para determiná-la, partimos de \(L_z \ket{0,\mu} = \hbar \mu\ket{0,\mu},\) e obtemos a equação diferencial
   \begin{align*}
      \bra{\xi'\eta'} L_z \ket{0,\mu} = \hbar \mu\braket{\xi'\eta'}{0,\mu} 
      &\implies \bra{\xi'\eta'}\hbar (\xi p_\eta - \eta p_\xi)\ket{0,\mu} = \braket{\xi'\eta'}{0,\mu}\\
      &\implies \left(-i\xi' \diffp{}{\eta'} + i\eta' \diffp{}{\xi'}\right)\Psi_{0,\mu}(\xi', \eta') = \mu \Psi_{0,\mu}(\xi', \eta')\\
                                        &\implies \eta' \diffp{\psi}{\xi'} = \xi' \diffp{\psi}{\eta'}\\
                                        &\implies \frac{\zeta + \zeta^*}{2}\left(\diffp{\psi}{\zeta} - \diffp{\psi}{\zeta^*}\right) + \frac{\zeta - \zeta^*}{2}\left(\diffp{\psi}{\zeta} + \diffp{\psi}{\zeta^*}\right) = \mu \psi(\zeta, \zeta^*)\\
                                        &\implies \zeta \diffp{\psi}{\zeta} - \zeta^* \diffp{\psi}{\zeta^*} = \mu \psi(\zeta, \zeta^*)\\
                                        &\implies \zeta \diff{f}{\zeta} - \frac14\abs{\zeta}^2 f(\zeta) + \frac14 \abs{\zeta}^2 f(\zeta) = \mu f(\zeta)\\
                                        &\implies \frac{1}{f(\zeta)} \diff{f}{\zeta} = \frac{\mu}{\zeta}.
   \end{align*}
   Notando que ambos os lados podem ser escritos como derivadas de logaritmos, obtemos
   \begin{equation*}
       f(\zeta) = \alpha \zeta^\mu,
   \end{equation*}
   com \(\alpha \in \mathbb{C}.\) Assim, 
   \begin{equation*}
      \psi(\zeta, \zeta^*) = \alpha \zeta^\mu \exp\left(-\frac14 \abs{\zeta}^2\right)
   \end{equation*}
   e como
   \begin{equation*}
      \int_{\mathbb{R}^2} \dln2{\zeta} \abs{\psi(\zeta, \zeta^*)}^2 = 2\pi\abs{\alpha}^2\int_{0}^\infty \dli{\rho} \rho^{2\mu+1} e^{-\frac12\rho^2} = (2\pi) 2^\mu \abs{\alpha}^2\int_0^{\infty} \dli{s} s^{\mu} e^{-s} = (2\pi) 2^\mu \mu! \abs{\alpha}^2,
   \end{equation*}
   concluímos que
   \begin{equation*}
      \Psi_{0,\mu}(\xi', \eta') = \frac{1}{\sqrt{2\pi \mu!}} \left(\frac{\xi' + i \eta'}{\sqrt{2}}\right)^\mu \exp\left(-\frac{{\xi'}^2 + {\eta'}^2}{4}\right)
   \end{equation*}
   é a função de onda do estado \(\ket{0,\mu}.\)
\end{proof}
